\par Le premier sous-objectif est de déterminer quelles sont les limites de la plateforme, d'un point de vue matériel (GPU, CPU, mémoire, transfert mémoire, consommation, etc.), mais aussi applicatif (entrainement, inférence). Cette phase du projet va permettre d'exécuter différents modèles déjà existants, sans  modification, en tenant compte des éléments documentés dans la littérature \cite{nguyen_mavnet_2019} \cite{jia_real-time_2020} \cite{nvidia_jetson_2019-1}. Selon le déroulement de cette étape, un ou plusieurs modèles seront sélectionnés. 
\par Un autre sous-objectif est d'optimiser ou d'adapter la plateforme, d'un point de vue matériel, mais aussi applicatif, afin d'avoir les meilleures performances et résultats possibles pendant l'entrainement et l'inférence.
\par Comme les résultats devront être disponibles en tout temps, une connexion à distance sécurisée devra être mise en place. Cette connexion permettra aussi de pouvoir prendre le contrôle du nano-ordinateur à distance et de l'administrer.
\par L'approche, les tests, et les résultats seront documentés. Il y aura beaucoup d'activités relatives à la conception et aux tests, le cheminement complet ne sera pas fourni. Une synthèse sera préférée et les informations les plus pertinentes seront incluses. Les détails de l'installation de l'environnement de développement et des applications, librairies et autres dépendances nécessaires seront inclus, ainsi que ceux de la configuration. Dans le cas où l'objectif principal n'est pas atteint, ou partiellement, la/les raison/s de l'échec seront spécifiées et des pistes de solutions potentielles proposées.
