\noindent L'objectif principal de cet essai consiste a étudier la capacité du nano-ordinateur du fabricant NVIDIA, le Jetson Nano \parencite{nvidia_jetson_2019}, à exécuter, en temps réel, une architecture de réseau de neurones pleinement connectés (\acrshort{fcnn}) entrainée à faire de la segmentation sémantique d'images et de vidéos de hautes résolutions qui sont perçues avec la caméra. L’approche, les tests, et les résultats seront documentés. Une seule classe sera extraite, celle représentant la piste multifonctionnelle. Les autres classes ne seront pas utilisées. Il semble important de préciser que l'objectif de l'essai n'est pas d'évaluer la précision des modèles produisant la segmentation sémantique, mais de déterminer, et ce en rapport avec les attentes du projet pour \acrshort{pjcci}, de la viabilité de pouvoir extraire la segmentation en temps réel à partir d'une vidéo de haute qualité avec le Jetson Nano dans un mode opérationnel 24/7, et de transmettre les délimitations de la piste multifonctionnelle à un autre programme pour détecter les conditions de la surface.
\vspace{0.5\baselineskip}
\\
\noindent Le premier objectif spécifique est de déterminer quelles sont les limites de la plateforme, d'un point de vue matériel (\acrshort{gpu}, \acrshort{cpu}s, mémoire, transfert mémoire, consommation, etc.), mais aussi applicatif, d'un point de vue inférence. Cette phase du projet va permettre d'exécuter différents modèles d'architecture déjà existants, sans les réentrainer, en tenant compte des éléments documentés dans la littérature \parencite{nguyen_mavnet_2019, zheng_real-time_2020, nvidia_jetson_2019-1}.
\vspace{0.5\baselineskip}
\\
\noindent Le second objectif spécifique est d'optimiser ou d'adapter la plateforme, d'un point de vue matériel, mais aussi applicatif, afin d'atteindre les meilleures performances et résultats possibles pendant l'inférence.
\vspace{0.5\baselineskip}
\\
\noindent D'autres objectifs spécifiques pourront être abordés si le temp le permet, tel qu'évaluer la possibilité de pouvoir réentrainer l’architecture sur le nano-ordinateur dans une perspective d’apprentissage actif et continue, réentrainer une architecture FCNN avec les images du site d’implémentation, permettre un accès à distance sécurisé au nano-ordinateur.
\vspace{0.5\baselineskip}
\\
\noindent À noter qu'il n'est pas planifié de faire des tests sur le site d'implémentation, ni de s'intégrer avec d'autres programmes du projet pour \acrshort{pjcci}, par exemple pour détecter les conditions de la surface de la piste multifonctionnelle. 
\vspace{0.5\baselineskip}