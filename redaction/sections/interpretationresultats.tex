\subsection{Performances matérielles}
\subsubsection{Stockage de données}
\par Les tests montrent que le \acrshort{ssd} interne est de 4 à 11 fois plus efficaces qu'une carte microSD, pour l'opération de lecture de données. 
\subsubsection{Performances système}
\myparagraph{Performances globales}
\par Concernant les performances globales du nano ordinateur, il est à noter que celui-ci est capable d'exécuter l'inférence en temps réel pour une durée prolongée (23 minutes dans ce cas), et rester réactif aux commandes. L'exemple qui le démontre est le démarrage du navigateur Chromium entre deux segmentations, et pendant la segmentation.
\myparagraph{Fréquence}
\par La commande "tegrastats" offre la fréquence des \acrshort{cpu}s (4 pour le nano ordinateur), le GR3D (\acrshort{gpu}) et EMC. On peut noter que l'inférence prend 100\% du GR3D pendant toute la durée. Les \acrshort{cpu}s sont tous utilisés équitablement pendant l'inférence, en dépassant rarement les 30\% d'utilisation. En fait la période qui montre une exploitation élevée des \acrshort{cpu}s est lors de l'utilisation de Chromium, ou l'ensemble des \acrshort{cpu}s sont employés entre 0\% et 90\%. 
\par Il faut donc rester vigilant quand à l'utilisation des \acrshort{cpu}s pendant l'inférence sur le long terme, au risque de perdre le système en raison d'un ralentissement progressif dû à un manque de ressources processeurs \acrshort{cpu}s.
\myparagraph{Mémoire}
\par La commande "free -m" offre l'utilisation mémoire du système en Mb. Le nano ordinateur au démarrage ne consomme qu'environ 1.5Gb de mémoire totale, et possède 4Gb de libres sur un total de 6 Gb. À la fin du test de 25 minutes, il ne reste qu'environs 3Gb de mémoire libre, un peu plus de 2Gb semble resté utilisé. De la mémoire swap a commencée à être consommée lors du démarrage de Chromium pendant la 3e segmentation, et ne semble jamais avoir été libérée. La mémoire tampon cachée est aussi sensiblement utilisée et revient un peu en dessous de son niveau original à la fin du test. 
\par De même que pour l'utilisation des processeurs, il semble être préférable de rester vigilant lors de l'utilisation opérationnelle du nano ordinateur, la segmentation consommant de la mémoire qui semble ne plus être disponible pour les autres ressources du système, comme le démontre l'état de la mémoire totale libre à la suite de l'arrêt de la 1re segmentation. 
\myparagraph{I/O}
\par La commande "iotop" offre les performances I/O du nano ordinateur pendant le test de 25 minutes. Le I/O de la segmentation est très raisonnable, de même que celle du système. Il n'y a quasiment pas d'opération visible en écriture, même la collecte des statistiques durant le test, aux secondes, n'apparait pas. Les opérations en lecture sont plus visibles, mais très ponctuelles. La période la plus occupée en lecture semble être due durant le démarrage de la segmentation la première fois: le système semble lire le modèle en mémoire, et le conserver en mémoire, car les opérations en lecture suivantes sont peu ou non visibles pendant le démarrage des segmentations suivantes.
\par Cela expliquerait l'augmentation de l'utilisation de la mémoire à la suite de la segmentation. 
\myparagraph{Température}
\par La commande "tegrastats" offre grâce à des capteurs intégrés à la carte mère la température de différents éléments matériels du nano ordinateur. La commande "sudo jetson\_clock" est démarrée manuellement dès que le système est démarré, permettant de profiter de la fréquence maximale d'utilisation supportée par le nano ordinateur. Le  succès de la commande est simple à vérifier: le ventilateur se met à ventiler aussitôt\footnote{\url{https://docs.nvidia.com/jetson/l4t/index.html#page/Tegra\%20Linux\%20Driver\%20Package\%20Development\%20Guide/power_management_nano.html}}.
\par La température dans la pièce au moment du test est de 27C. Au démarrage, on note que la température mesurée de la plupart des capteurs thermiques, sauf pour le AO ("Always on") est entre 33C et 36C. Le démarrage de la 1re segmentation fait graduellement monter la température, entre 37C et 39C, jusqu'au point d'arrêt de la segmentation, après 200 secondes approximativement, et qui diminue graduellement approximativement pendant 200 secondes vers son point d'origine lorsqu'elle est arrêtée. Le démarrage de Chromium pendant cette période semble ralentir un peu le refroidissement. L'observation lors de la seconde segmentation est identique à la première. La troisième segmentation est plus longue, 400 secondes, et voit la température se stabiliser entre 41C et 43C. L'arrêt de la segmentation voit la température baisser et revenir assez rapidement à sa température d'origine. 
\par Le capteur thermique AO ("Always on" ) est plus particulier, puisqu'il mesure une température de 10C supérieures aux autres capteurs. Selon le modérateur Trumany de NVIDIA\footnote{\url{https://forums.developer.nvidia.com/t/operating-temperature-range-on-jetson-nano/73555/10}}, "AO\_therm is used for a truly robust thermtrip and as an LP0 wake source, as other zones will cease to operate during LP0.". Mes compétences en la matière ne me permettent pas d'expliquer clairement ce renseignement, mais cela semble signifier que ce capteur est plus robuste que les autres et devient l'indicateur de référence pour gérer une surchauffe. 
\par Il est donc à noter que l'opérationnalisation constante de la segmentation aurait un impacte non négligeable sur la durée de vie du Jetson Nano. Selon la documentation de NVIDIA, une carte Jetson Xavier TX2i qui opère 24/7, selon certaines conditions, a une durée de vie théorique de 4,4 années\footnote{\url{https://docs.nvidia.com/jetson/l4t/index.html#page/Tegra\%2520Linux\%2520Driver\%2520Package\%2520Development\%2520Guide\%2Fjetson_module_support.html}}
\par Au besoin, plus d'informations peuvent être trouvées dans le guide de conception thermique du Jetson Nano \footnote{\url{https://developer.download.nvidia.com/assets/embedded/secure/jetson/Nano/docs/Jetson_Nano_Thermal_Design_Guide_TDG-09383-001_v1.3.pdf?2P65awpyl3RwXu6jWjsqFgresjNSqhO-N2uI3BPNH2Wcbp9LNh91GF3UtmC3JgEWd6MX2-BC5xoL80tY5Wpl5cEltIMR4IawEflJehkxKH3yDAgxV-HpXyOo5Ge8a32mdntMcfRzjRZZTP2-hsJlIuT5FB7G36zHkCva7uPS9ntgWDff-w1W0LBJLH5DvpE1qU-3yZM5hjSz9g9cpFM}}
\myparagraph{Consommation}
\par La commande "tegrastats" offre de visualiser la consommation du nano ordinateur, soit globale, pour les \acrshort{cpu}s et pour le \acrshort{gpu}. En mode opérationnel continue, cela peut avoir une importance sur le budget, car la consommation est clairement beaucoup plus élevée pendant la segmentation. Il peut être observé aussi qu'elle est beaucoup plus volatile avec Chromium démarrée. 
\subsection{Performances de la segmentation}
\subsubsection{Images}
La segmentation prédite pour la classe "Trail" est assez surprenante. Le \acrshort{iou} est de 89\% et de 69\% respectivement dans le cas des deux images évaluées, ce qui est très encourageant. Par contre les délimitations de la segmentation pour le chemin sont décevantes et questionnables, car le modèle retourne une résolution très faible, l'image est très grossièrement "pixelisée", de gros carrés sont utilisés pour délimiter chaque classe. C'est probablement dû au fait que l'architecture du modèle segnet18 n'utilise que 18 couches, et qu'il n'y a donc que peu de représentations possibles pour les classes. 
\subsubsection{Vidéos}
La segmentation des vidéos n'a pas pu être évaluée avec des indicateurs de performances. C'est donc subjectivement que l'on peut donner une appréciation. Comme les images se succèdent très rapidement, ce n'est pas non plus évident. Grossièrement je n'ai pas trouvé le résultat de la segmentation très bonne. Les vidéos proviennent d'un contexte autre que celui du jeu de données d'entrainement (la forêt), la surface de la piste est bien différente. 
\par Je ne sais pas trop comment la segmentation en temps réel de vidéos peut être utilisée, car il n'y a pas moyen d'évaluer la qualité de la segmentation sans avoir la vérité terrain (\acrshort{gt}).