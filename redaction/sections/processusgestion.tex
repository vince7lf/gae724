\vspace{\baselineskip}
\par L’environnement nuagique Office 365 (Office, OneDrive) sera utilisé afin de partager le rapport, les analyses, les références, les résultats et les discussions.
\par La méthodologie Agile sera utilisée afin d’être efficace et proactive dans le développement et le suivi des tâches. L’application Web "Trello" sera utilisée pour la gestion des tâches. 
\par Le code source sera conservé dans le contrôleur de source gitHub en tant que projet public. Le code pourra ainsi être facilement partagé avec le directeur et accessible publiquement et gratuitement. Le type de la licence n'a pas encore été décidé. 
\par Au niveau de la communication, Outlook365, le téléphone, les messages textes et Skype seront utilisés au besoin. Skype permettra de partager l’écran. 
\par Ayant de l’expérience professionnelle dans une grande compagnie où la collaboration avec des employés localisés à travers le Canada est très bien implantée, et les déplacements limités, les rencontres physiques avec le directeur seront peu nombreux. Il s’agira surement de se rencontrer au moment de concevoir l’architecture, puis afin de clarifier certains points et incertitudes, ou l’utilisation d’un tableau blanc sera alors grandement utile. Autrement les outils de collaboration existant en ligne remplissent amplement leur rôle.
\par Les horaires disponibles pour communiquer par téléphones seront entre 14h et 18h, entre le lundi et le vendredi. Certaines soirées de semaine seront probablement mises à profit, au besoin. 
\par L’échange de documents par courriel sera fortement déconseillé. Les commentaires pourront être échangés par courriel, par téléphone, message texte, par chat, mais il est attendu qu’ils soient intégrés directement dans les documents en ligne.
\par Je serais responsable de la gestion des outils de collaboration et de la planification des rencontres.
\par Le directeur sera responsable de savoir collaborer avec ces outils sans toutefois avoir à les maîtriser. 
\par Le projet sera étalé sur deux sessions, c’est-à-dire 38 semaines. Les efforts sont estimés entre 1 à 3 jours par semaine, l'équivalent de 40 à 90 jours d’effort (250 à 500 heures). Étant donné que je suis travailleur à temps plein (37.5/semaine), père de famille de deux jeunes enfants (8 \& 10.5) et propriétaire d’une maison, les soirs de semaines seront dédiés à l’essai, entre 21h00 et 23h00 (10 heures étalées sur 5 jours). Les fins de semaine seront principalement utilisées pour se reposer et "décrocher". Une seule semaine de pause est prévue (vacances scolaires de mars). Je planifie prendre entre cinq et douze jours (vendredi et/ou lundi) de congés durant la semanie, au besoin.
