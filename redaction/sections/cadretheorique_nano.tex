\par Voici le plan qui est utilisé pour rédiger le cadre théorique au sujet du nano-ordinateur.
\begin{itemize}
   \item historique et évolution; une brève présentation de l'historique des nano-ordinateurs, de leur apparition à leur place aujourd'hui.
   \par 
   \item usages; quelques exemples d'usages des nano-ordinateurs, dans un contexte professionnel.
   \par 
   \item architecture; brève présentation, fonctionnement et comparaison des architectures matérielles des nano-ordinateurs, leurs coûts, leurs avantages et limitations.
   \par 
\end{itemize}
\vspace{1\baselineskip}
\par 