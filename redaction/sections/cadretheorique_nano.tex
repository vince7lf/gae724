\par Le nano ordinateur de cet essai doit être compris comme étant un ordinateur miniature, ayant une taille et des capacités qui lui permettent d'être installé ("embedded system") dans une voiture, un drone, un tracteur ou être accroché à un poteau. Le terme "On the Edge" s'y approprie mieux que "\acrshort{iot}" ("\acrlong{iot}"), puisqu'étant sur le terrain il se trouve directement proche des données, ce qui lui donne l'avantage de pouvoir faire des traitements en temps réel. Selon Wikipedia, les premiers systèmes embarqués reconnus comme tels, sont ceux installés dans le missile Minuteman (1961; ref\todo{TODO}) et la navette Apollo (1960; ref\todo{TODO}). Les avancées technologiques ont permis de les rendre de plus en plus compactes et performants, et le Jetson Nano de NVIDIA est le dernier né de la compagnie NVIDIA permettant d'inférer en temps réel des modèles d'intelligence artificielle, sans ajout de périphériques. Du même constructeur, ses grands frères sont le Jetson Xavier and le Jetson TX2, plus performants, et donc plus onéreux. Son concurrent direct est le Raspberry Pi, mais il nécessite une extension USB Movidius Intel pour l'inférence de modèles \acrshort{ia}. 
\par Ce qui caractérise principalement un ordinateur miniature, est le fait qu'il soit assez petit pour pouvoir être embarqué dans un système plus gros, tel qu'un robot ou du matériel médical. Son cout est bas, en raison des performances qui sont limitées par une conception répondant à un besoin très spécifique. Tous les éléments matériels requis sont contenus sur une même carte.  Une fois installé et paramétré, le système se doit d'être fiable et opérationnel sur le long terme. Mais il doit aussi être interchangeable, au besoin, rapidement et facilement. La consommation électrique est généralement faible. Étant généralement opérationnel sur le terrain, proche des données, il est responsable d'une tâche bien particulière, qu'il doit accomplir efficacement. Il n'y a généralement pas d'interface utilisateur, et l'accès au système se fait à distance ou via une console. Il est composé de capteurs, au besoin d'une caméra. Le même système peut-être déployé en grande quantité, comme dans le contexte de notre essai, où plusieurs pairs seront déployés le long de la piste multifonctionnelle; un autre exemple est celui des constellations de nano satellites.
{
   \renewcommand*{\arraystretch}{1.4}
   \begin{table}[ht]
   \centering
   \caption{Comparaison de trois nano ordinateurs prêts pour l'\acrshort{ia}}\label{table:compare_nano}
   \vspace{0.1em} % Adjust the height of the space between caption and tabular
   \begin{tabular}{{@{}|p{12em}|p{12em}|p{12em}|@{}}}
      \hline
      \textbf{NVIDIA Jetson Nano} & \textbf{NVIDIA Jetson Xavier AGX} & \textbf{Raspberry Pi 4B + Intel NCS2}\\
      \hline  
      \centering 99USD & \centering 599USD &  134USD (55USD + 79USD) \\
      \hline
      45x69.6mm, 250gr, 5-10W & 100x87mm, 630gr, 10-15-30W & 56x85.60mm + 27x72mm, 45gr + 18.1gr, 15W\\
      \hline
      128-core NVIDIA Maxwell GPU & 512-core NVIDIA Volta GPU with 64 Tensor Cores & Intel Movidius Myriad X VPU 16 SHAVE cores \\
      \hline
      Quad-Core ARM Cortex-A57 MPCore & 8-core NVIDIA Carmel Arm v8.2 64-bit CPU 8MB L2 + 4MB L3 & Quad-core ARM Cortex-A72 64-bit @ 1.5 GHz\\
      \hline
      4 GB 64-bit LPDDR4 & 32 GB 256-bit LPDDR4 & 4GB LPDDR4\\
      \hline
      0.47 TFLOPS@FP16 & 5.5-11.5 TFLOPS@FP16; 20-32 TOPS@INT8 & 4 FLOPS@FP16, 1 TOPS@INT8 \\
      \hline
   \end{tabular}
   \end{table}
}