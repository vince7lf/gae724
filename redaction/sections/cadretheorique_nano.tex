\noindent Les nano-ordinateurs et les objets connectés, désignés aussi par l'\acrlong{iot} ou "\acrshort{iot}", \parencite{blanco-filgueira_deep_2019, sharma_history_2019} sont le résultat de la miniaturisation des systèmes informatiques. Ils permettent la détection en temps réel à des endroits, dans des situations et dans des conditions qui n'étaient pas envisageables il y a encore 10 ans \parencite{zheng_real-time_2020, bernas_edge_2017, abouzahir_iot-empowered_2017, blanco-filgueira_deep_2019}.
\vspace{0.5\baselineskip}
\\
\noindent Le nano-ordinateur de cet essai doit être compris comme étant un ordinateur miniature, ayant une taille et des capacités qui lui permettent d'être installé ("embedded system") dans une voiture, un drone, un tracteur ou être accroché à un poteau. Le terme anglais "On the Edge" (sur le bord), s'y approprie mieux que "\acrshort{iot}" ("\acrlong{iot}"), puisqu'étant sur le terrain il se trouve directement proche des données, ce qui lui donne l'avantage de pouvoir faire des traitements en temps réel. Les premiers systèmes embarqués reconnus comme tels, sont ceux installés dans le missile Minuteman (\parencite{kilby_nobel_2000}) et la navette Apollo (\parencite{kilby_nobel_2000}). Les avancées technologiques ont permis de les rendre de plus en plus compactes et performants. Les systèmes de la compagnie Campbell Scientific existent depuis les années 1974 et pemettent l'acquisition de données à distance. Le système Arduino est l'un des premiers microprocesseurs a avoir été destinés à la robotique. Le Jetson Nano de NVIDIA est le dernier né des nano-ordinateurs de la compagnie NVIDIA permettant d'inférer en temps réel des architectures d'intelligence artificielle, sans ajout de périphériques. Du même constructeur, ses grands frères sont le Jetson Xavier and le Jetson TX2, plus performants, et donc plus onéreux. Son concurrent direct est le Raspberry Pi, mais il nécessite une extension USB Movidius Intel pour l'inférence de modèles \acrshort{ia}. 
\vspace{0.5\baselineskip}
\\
\noindent Ce qui caractérise principalement un ordinateur miniature, est le fait qu'il soit assez petit pour pouvoir être embarqué dans un système plus gros, tel qu'un robot ou du matériel médical. Son coût est bas, en raison des performances qui sont limitées par une conception répondant à un besoin spécifique. Tous les éléments matériels requis sont contenus sur une même carte. Une fois installé et paramétré, le système se doit d'être fiable et opérationnel sur le long terme. Mais il doit aussi être interchangeable, au besoin, rapidement et facilement. La consommation électrique est faible, entre 5 et 10 watt. Étant généralement opérationnel sur le terrain, proche des données, il est responsable d'une tâche bien particulière, qu'il doit accomplir efficacement. Il n'y a généralement pas d'interface utilisateur, et l'accès au système se fait à distance ou via une console. Il est composé de capteurs, au besoin d'une caméra. Le même système peut être déployé en grande quantité, comme dans le contexte de notre essai, où plusieurs paires seront déployées le long de la piste multifonctionnelle; un autre exemple est celui des constellations de nano satellites.
\vspace{0.5\baselineskip}
\\
\noindent L'annexe \ref{annexe:nano_computer_samples} montrent les nano-ordinateurs qui supportent les \acrshort{sdk} pour l'\acrshort{ia}.
