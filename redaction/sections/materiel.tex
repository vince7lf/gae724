\par Voici le plan qui est utilisé pour rédiger au sujet du matériel.
\begin{itemize}
   \item Présentation du nano-ordinateur qui sera utilisé, le NVIDIA Jetson nano;
   \item Usage du nano-ordinateur durant l'essai.
\end{itemize}

\par L'objet d'étude de cet essai est un nano-ordinateur. Un nano-ordinateur est un ordinateur miniaturisé en taille, mais aussi limité en capacité. Il existe différents fabricants et modèles, de caractéristiques techniques variées, pour répondre à différents besoins. Le dernier né est le modèle "Jetson nano" du fabricant "NVIDIA", disponible depuis juin 2019 au prix très abordable de 99\$US. La compagnie NVIDIA a conçu ce matériel spécialement pour différentes applications d'inférence de modèles d'apprentissage profond sur une plateforme mobile (drone) ou proche des données ("edge" en anglais). Ce modèle a été choisi afin de répondre à l'intérêt que suscitent ses capacités et ses limites. Une image du Jetson nano et un tableau de ses caractéristiques techniques seront ajoutés. 
\par L'architecture matérielle sera étudiée et présentée avec l'aide d'images, de diagrammes et de textes explicatifs. Les éléments clés seront identifiés.
\par Afin d'optimiser les performances du Jetson nano, une recherche des périphériques les plus adaptés pour répondre aux besoins de performance (et de budget) de l'essai est essentielle, telle que l'alimentation, le stockage, la caméra. Des images des périphériques seront incluses, et les caractéristiques principales seront présentées dans des tableaux.
\par Il est à noter que le NVIDIA Jetson nano est déjà en ma possession. La liste des équipements est en cours et sera commandée par le collaborateur "Vision météo".

\paragraph{Logiciels -– 1-2 pages}
\par De même que pour les périphériques, les logiciels qui seront utilisés seront résumés dans un tableau, où il sera indiqué leur nom, le type de licence, leur version, leurs avantages et limitations, comme pour le système d'exploitation, l'environnement de développement pour l'apprentissage profond, l'inférence, les logiciels de traitements vidéos et d'images. 
\begin{itemize}
   \item Présentation du système d'exploitation qui sera installé;
   \item Présentation des frameworks d’apprentissage profond qui vont être utilisés (PyTorchm torchvision);
   \item Présentation des librairies pour d'inférence (TensorRT, onnx);
\end{itemize}
\vspace{1\baselineskip}
\par 