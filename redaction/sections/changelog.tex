\par Cette section n'est disponible que pendant la rédaction du rapport. Les changements au rapport sont documentés dans ce tableau. Cela permet de faire un suivi et permet d'aider les personnes qui révisent et commentent le rapport.
\par Les changements les plus récents apparaissent en premier. 
{
    \vspace{0.1em} % Adjust the height of the space between caption and tabular
    \begin{longtable}[t]{{@{}|p{3em}|p{6em}|p{24em}|@{}}}
        \caption{Suivi des changements}\label{table:changelog}\\
        \hline
        \textbf{Version} & \textbf{Date} & \textbf{Changement}\\
        \endfirsthead
        \hline
        \textbf{Version} & \textbf{Date} & \textbf{Changement}\\
        \hline
        \endhead
        \endfoot
        \endlastfoot
        \hline
        v1.2.1 & 27 28 29 30 juillet 2020 & \begin{itemize}
            \item Fixe les longues URL dans tout le document.
            \item Bonifie la section Résultat avec un tableau des résolutions qui ont été testées, et spécifie avec quelle carte microSD les tests de performance sont exécutés.
            \item Bonifie la section des Interprétations des résultats\char`\\Température avec le capteur AO et la durée de vie en opération constante. 
            \item Ajoute une image pour chaque microSD dans un tableau, mais le label n'est pas bien centré verticalement. (todo en cours).
            \item Bonifie la section Solutions logicielle, et ajoute un tableau des logiciels.
            \item fixe les Caption for LOF par le bon libellé pour la TOC.
            \item plusieurs clarifications et bonifications dans les sections contexte, données, interprétation des résultats, matériel logiciels, méthodologie, méthodologie/choix d'équipement, méthodologie/évaluation, méthodologie/exploration solutions logicielles, méthodologie/mise en place de solutions logicielles, méthodologie/préparation nano ordinateur, objectifs, problématique, résultats.
        \end{itemize}\\
        \hline
        v1.2.0 & 26 juillet 2020 & \begin{itemize}
            \item Modification dans l'ensemble du document de l'expression \acrshort{gt} par vérité terrain (\acrshort{gt}).
            \item Ajout dans les notes de bas de page des URLs (images du pont; github).
            \item Site d'étude: ajout des URL en note de base de page; enlève la deuxième image du pont; correction du nom de la station de métro la plus proche.  
            \item Matériel et logiciels/Le nano ordinateur : Ajout d'une référence à l'image.
            \item Méthodologie: référence à la documentation dans githug; ajout d'une sous-section Documentation; ajout des liens github dans les notes de bas de page.  
            \item Stratégie de collecte des indicateurs de performance: mise à jour majeure.
            \item Préparation du nano ordinateur/Disque SSD NVMe M.2 interne 250GB: mise à jour; ajout de notes de bas de page.
            \item Préparation du nano ordinateur/Caméra : précision de laquelle (Raspberry Pi v2)
            \item Résultats/Performances système: Précision du FCN + résolution sélectionnée lors du test.
            \item Résultats/Performances de l'inférence/Vidéo: ajout des liens vidéos en notes de bas de page ... mais trop long.
        \end{itemize}\\
        \hline
        v1.1.0 & 25 juillet 2020 & Section Résultats et Interprétation des résultats.\\
        \hline
        v1.0.2 & 19 juillet 2020 & Ajout dans le tableau des performances data read SSD vs microSD les 2 autres microSD utilisées pendant l'essai.\\
        \hline
        v1.0.1 & 18 juillet 2020 & \begin{itemize}
            \item Ajout d'une section changelog pour documenter le suivi des changements et faciliter la révision.
            \item Apport de clarification dans la section Matérielle \& logiciels (hdparm, présentation SDK, etc). 
            \item Début de la rédaction de la section résultat; inclusion d'un tableau des performances data read SSD vs microSD.
        \end{itemize}\\
        \hline
        v1.0 & 17 juillet 2020 & Première version envoyée à Michaël G. \\
        \hline
    \end{longtable}
}
\clearpage
\newpage