\noindent La segmentation sémantique d'images ou de vidéos est une technique de télédétection du domaine de la vision par ordinateur. Elle permet de délimiter (segmenter) différentes parties (sémantique) d'une image. Les méthodes de segmentation ont été améliorées ces dernières années par les récentes avancées dans le domaine de l'apprentissage profond. 
\begin{figure}[H]
   \centering
   \includegraphics[width=0.75\textwidth]{semantic_segmentation_vs_others}
   \caption[Segmentation semantic]{Segmentation semantic\parencite[p.~1]{wu_recent_2019}}
   \label{fig:semantic_segmentation_vs_others}
\end{figure}
\noindent L’apprentissage profond est un sous-domaine de celui de l'apprentissage machine qui est un sous-domaine de celui de l'intelligence artificielle. 
\begin{figure}[H]
   \centering
   \includegraphics[width=0.5\textwidth]{Deep_Learning_with_Python.pdf}
   \caption[Relation entre \acrlong{ia}, \acrlong{am} et \acrlong{ap}]{Relation entre \acrlong{ia}, \acrlong{am} et \acrlong{ap} \parencite[p.~4]{chollet_deep_2018}}
   \label{fig:ia_ml_ap}
\end{figure}
\noindent Les concepts de l'\lowercase{\acrlong{ia}} (AI) existent depuis les années 1950 \parencite[p.~4]{chollet_deep_2018} \parencite[p.~1]{alom_history_2018}, et ont continué à se développer par vague, jusqu'à leur nouvelle popularité des 15 dernières années. En effet, trois raisons principales ont permis à ce domaine de renaitre de nouveau \parencite[p.~20]{chollet_deep_2018}: 1) la capacité et la puissance des machines; 2) des jeux de données plus larges; 3) des algorithmes plus avancés. Les deux moments clés, preuves de cette renaissance, sont: 1) la possibilité d'entrainer des architectures de réseaux de neurones profonds (DNN) (2006) \parencite[p.~6]{alom_history_2018}; et 2) l'architecture du réseau de neurones convolutionels AlexNet permet de  gagner le challenge ImageNet contre les approches traditionnelles\parencite[p.~11]{alom_history_2018}. 
\vspace{\baselineskip}
\\
\noindent En 2016 \parencite[p.~14]{alom_history_2018}, l'architecture \acrshort{fcn} (\acrlong{fcn}, réseau (de neurones) convolutionnel entier) a permis aux taches réservées à la segmentation d'images d'être plus efficace que les méthodes traditionnelles de la vision par ordinateur. Cette nouvelle méthode s'applique désormais à tous les domaines connexes à l'analyse d'images, tels que l'imagerie médicale, la conduite autonome de véhicules, la robotique, la télédétection d'images satellites, la sécurité par caméra vidéo, l'agriculture de haute précision. Aujourd'hui, elle peut s'exécuter en temps réel sur des systèmes embarqués proche des données. 