\par Les images et vidéos qui seront acquises devront répondre à différents requis, afin de pouvoir adapter le modèle de réseau de neurones dans des conditions les plus proches de la réalité. 
\begin{itemize}
   \item L'objet d'intérêt pour la collecte des données est la surface de la piste multifonctionnelle du pont Jacques-Cartier, à Montréal (Québec, Canada). 
   \item L'acquisition se fera à partir de différents points d'intérêt sur le pont. Ces points d'intérêt seront déterminés par les gestionnaires du projet pour PJCCI. 
   \item L'appareil d'acquisition sera installé sur un trépied pendant la capture afin d'avoir un point de vue stable et constant. 
   \item Pour chaque point d'intérêt, la hauteur du trépied, en plus de son géoréférencement, l'angle de vision et la direction seront documentés. Les points d'intérêt seront géoréférencés afin d'être représentés sur une carte. La précision du géoréférencement devra être assez précise pour que d'autres acquisitions puissent avoir lieu. De plus, au moins trois points de référence sur les bords de l'image devront être déterminés pour garder le même angle de vision et la direction à chaque acquisition. 
   \item La capture sera faite pendant une période assez prolongée, telle que des périodes de 15 ou 30 minutes. La période d'acquisition est à la discrétion des architectes applicatifs du projet pour PJCCI.
   \item La même surface sera capturée de différents angles en simultané avec deux appareils d'acquisition. Cela permettra d'avoir deux images de la même zone au même moment, mais d'une perspective différente.
   \item Comme indiqué précédemment, plusieurs points d'intérêts de la piste multifonctionnelle seront capturés, toujours de différents angles en simultané.
   \item L'acquisition sera faite de jour.
   \item La luminosité devra être bonne, normale. 
   \item L'acquisition dans des conditions spéciales est exclue, telle que la nuit, une luminosité ou une visibilité extrême, comme aucune, ou médiocre.
   \item Des images des différentes conditions de la surface seront acquises: sèche, partiellement mouillée, trempée, partiellement recouverte de neige, recouverte de neige, glace noire, "slush", etc. Cela devrait correspondre aux classes du projet avec PJCCI et non celles de l'essai. Il existe de nombreuses références à ce sujet, par exemple \cite{cheng_road_2019}, \cite{fu_risk-based_2017}, \cite{pan_winter_nodate}.
   \item Les vidéos devront capturer du trafic: piéton, vélo, coureur, poussette, groupe, chiens en laisse, etc. Cela va permettre d'entrainer le modèle lorsqu'il y a des obstacles sur la piste et tester la fiabilité de la segmentation sémantique. 
   \item Les vidéos seront acquises en différentes résolutions: haute (1080p/i), puis de plus en plus faible 760p/i, 576p/i, 480p/i, 360p/i. Cela va permettre de tester les performances avec une perte progressive de la qualité (nombre de pixels).
   \item Les vidéos seront acquises avec un nombre d'images par seconde différente (FPS, "Frames Per Second" en anglais): élevés (30FPS), puis de moins en moins rapide, 20FPS, 10FSP, 1FPS. Cela va permettre de tester les performances et déterminer le meilleur compromis entre résolutions et FPS.
   \item Il n'est pas important que l'appareil d'acquisition soit le même que celui qui sera utilisé pendant l'essai, car les images sont adaptées par l'environnement de développement ("framework" en anglais) d'apprentissage profond au moment du prétraitement de l'entrainement.
   \item Comme il y aura une quantité non négligeable de nouvelles données acquises, qui de plus sont des données multimédias de haute résolution, un espace de rangement ("storage" en anglais) suffisant et performant sera nécessaire. Une espace de sauvegarde est de plus indispensable pour conserver les données brutes, mais aussi les différentes versions dues aux traitements. Un espace dans le nuage ("cloud") est une option non négligeable, mais la décision est à la discrétion des architectes applicatifs du projet pour PJCCI.
   \item Une nomenclature pour le nom des répertoires et des fichiers vidéos sera déterminée, à la discrétion des architectes applicatifs du projet pour PJCCI. Les données utilisées pour l'essai seront gérées à ma discrétion (espace de rangement, copies de sauvegarde, nomenclature).
\end{itemize}
