Voici les données qui ont été identifiées comme nécessaires.
\begin{itemize}
   \item jeux de données (images) d'entrainement des modèles existants;
   \item vidéos de la zone d'étude; 
   \item nouvelles images, extraites des vidéos;
\end{itemize}
\vspace{1\baselineskip}
\par Il existe déjà des modèles de réseaux de neurones dont l'application est la segmentation sémantique. Ces modèles ont été entraînés avec des jeux de données qui sont disponibles gratuitement. Il est planifié utiliser ces modèles déjà pré-entrainés, et les ré-entrainer en bonifiant les jeux de données d'entrainement avec des données locales. Ces jeux de données seront aussi utiles pour valider et tester le résultat du ré-entrainement.
\par Selon nos recherches, à ce jour, il n'existe pas de modèles de réseaux de neurones qui ont été entrainés pour répondre directement aux objectifs de l'essai. Il sera donc nécessaire de construire des jeux de données représentant la zone d'étude, le "nouveau domaine" de l'apprentissage. Ces nouvelles images permettront d'adapter à ce nouveau domaine les modèles de réseaux de neurones qui auront été sélectionnés (technique "Adaptation domain" en anglais). Les images seront extraites des vidéos acquises dans les conditions décrites ci-après (section \ref{sect:conditions}).
\par Les vidéos sont la source de données principales de l'essai: l'inférence est faite à partir de vidéos en temps réel. La résolution et le nombre d'images par seconde seront à déterminer. Les vidéos seront acquises dans les conditions décrites ci-après (section \ref{sect:conditions}).
