Voici les stratégies qui ont été identifiées pour la recherche de données.
\begin{itemize}
   \item resources NVIDIA;
   \item références en lien avec le sujet;
   \item Internet;
   \item Association des piétons et cyclistes du pont Jacques-Cartier;
   \item travaux d'étudiants de l'Université de Sherbrooke;
   \item acquisition des vidéos de la piste multifonctionnelle du pont Jacques-Cartier;
   \item discussions avec mon directeur de projet;
\end{itemize}
\vspace{1\baselineskip}
\par Les ressources mises à disposition par le constructeur du Jetson nano, NVIDIA, seront étudiées pour apprendre et tester le nano-ordinateur. Parmi les plus intéressantes, on peut citer le "Jetson Nano Developer Kit", le "NVIDIA Deep Learning Institute", la communauté Jetson, les tutoriels, les "benchmarks". Des jeux de données sont fournis gratuitement.
\par En complément des ressources de NVIDIA, deux références scientifiques seront principalement utilisées comme points de départ et comme modèles pour l'essai, car leurs études ont été faites avec le Jetson nano (\cite{nguyen_mavnet_2019} et \cite{chong_real-time_1992}). Beaucoup de références ont été publiées ces deux dernières années sur le sujet de la segmentation sémantique, ils existent donc de multiples alternatives inspirantes.
\par Internet est une mine d'information et de données. Il existe des forums et des blogues dans lesquels des utilisateurs publient leurs expérimentations de la segmentation sémantique en temps réel avec le Jetson nano (\cite{dustin_realtime_2019}), ou plus génériquement la segmentation sémantique. Des sites comme "modelzoo.co" sont des entrepôts de modèles déjà pré-entrainés. Une autre option est d'effectuer une recherche d'images ou de vidéos de la piste multifonctionnelle du pont Jacques-Cartier via les sites de recherche tels que Google. 
\par L'Association des piétons et cyclistes du pont Jacques-Cartier existe depuis de nombreuses années pour promouvoir le transport actif et conserver la piste multifonctionnelle du pont Jacques Cartier ouverte durant l'hiver. Ils fournissent, via des sites Internet, des collections de vidéos et d'images qui pourraient être utilisées. Il serait aussi possible d'entrer en contact avec l'association et leur demander de prendre de nouvelles vidéos. Voir http://pontjacquescartier365.com, et https://www.flickr.com/photos/pontjacquescartier.
\par Une autre possibilité serait d'hériter des acquisitions faites par un autre étudiant de l'université de Sherbrooke, soit déjà archivée, soit collectée prochainement. Mon directeur de projet Mickaël G. m'a informé qu'un étudiant de Sherbrooke va avoir besoin de collecter le trafic automobile sur le campus de l'Université de Sherbrooke, à Sherbrooke. 
\par Enfin il y a l'acquisition des vidéos spécifiquement pour le projet PJCCI, tel que documenté à la section \ref{sect:conditions}. Comme il n'y a aucune date de planifiée pour la capture des vidéos, l'essai devra s'arranger pour dépendre le moins possible d'elles durant la préparation et le développement, et s'attendre à les recevoir pour le ré-apprentissage et les tests, en fin d'essai.
\par Tout au long de l'essai, mon directeur Mickaël sera une ressource importante afin de vérifier que les sources de données, les prétraitements et les traitements sont adéquats aux attentes du projet pour PJCCI.
