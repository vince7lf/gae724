\myparagraph{Jetson Nano}
\vspace{\baselineskip}
\\
\noindent Le nano-ordinateur est destiné a l'inférence. NVIDIA fournit tout un système d'installation, qui est nommé JetPack, et qui contient un système d'exploitation basée sur Ubuntu, "\acrlong{l4t}" \acrshort{l4t}), la plateforme applicative et les librairies nécessaires pour l'inférence, tels que Python, pytorch, les modèles préentrainés au format \acrshort{onnx}, le compilateur CUDA, et le \acrshort{sdk} TensorRT.
\myparagraph{Calcul Québec}
\vspace{\baselineskip}
\\
\noindent Le nano-ordinateur est destiné a l'inférence, et non l'entrainement d'architectures. Il n'est pas non plus destiné a être un environnement de développement. Un autre environnement de travail est donc nécessaire pour développer, et doit posséder les capacités matérielles (\acrshort{gpu}s, mémoires, espace de stockage) et logicielles (librairies) pour entrainer une architecture. Le professeur Mickaël Germain, directeur de projet, m'a présenté l'environnement de Calcul Québec. Celui-çi fournit un espace de travail scientifiques destiné aux chercheurs et aux universitaires, qui m'a permit  de pouvoir travailler avec l'apprentissage profond, compiler un fork de torchvision, réentrainer des architetures, générer les versions \acrshort{onnx}, et ainsi contourner les limitations du nano-ordinateur. Avoir accès a cet environnement de travail a été un élément clé dans le cadre de cet essai.
\mysubparagraph{Compte Calcul Québec}
\vspace{\baselineskip}
\\
\noindent Calcul Québec mets à disposition des ressources matérielles puissantes et l'accès a des libraires de haute technologie telle que pour l'apprentissage profond, permettant d'avoir un environnement de travail professionnel et performant rapidement. Les ressources matérielles à disposition sont des grappes de serveurs, de \acrshort{cpu}s et \acrshort{gpu}s de différents types, ainsi que de l'espace de stockage. Les librairies sont disponibles via un repository privé, et lorsque certaines étaient manquantes (\acrshort{onnx} et onnxruntime), j'ai fait une demande par courriel. L'administrateur a pu rendre disponible l'une des deux (\acrshort{onnx}), la seconde (onnxruntime) étant beaucoup plus complexe a installé, pour l'avoir tenté sur le nano-ordinateur. 
\vspace{\baselineskip}
\\
\noindent L'autre avantage de l'environnement de Calcul Québec est la mise à disposition de Jupyter Notebook, afin de tester rapidement du code Python. Par contre il n'est pas conseillé d'exécuter du code nécessitant des délais, tels que l'entrainement d'une architecture. 
\vspace{\baselineskip}
\\
\noindent L'un des irritants est de ne pas pouvoir exécuter un container docker tel quel. Il faut le convertir au format Singularity. Dans le cadre du projet cela m'aurait facilité la tâche, car NVIDIA fournit des docker prêt à l'utilisation pour le réentrainement. Je n'ai malheureusement pas pris le temps et la chance de convertir un container docker au format Singularity. Je ne sais pas si c'est une activité assez simple ou complexe, mais du peu que j'ai lu cela semble assez "rapide".
\mysubparagraph{Jupyter Notebook}
\vspace{\baselineskip}
\\
\noindent Le besoin de tester du code Python est toujours nécessaire. La console Python n'étant vraiment pas conviviale, un environnement Jupyter Notebook est un compromis incontournable. Heureusement Calcul Québec fournit un accès à des notebooks depuis Internet, permettant en plus d'hériter de leur environnement de travail. Il est à noter que les notebooks n'ont pas été utilisés pour entrainer une architecture ou générer les versions \acrshort{onnx}, mais de tester du code Python simple, comme visualiser des images, transformer des tensors, et évaluer la segmentation prédite générée avec le vérité terrain (\acrshort{gt}). 
\paragraph{NVIDIA}
\mysubparagraph{Compte NVIDIA}
\vspace{\baselineskip}
\\
\noindent NVIDIA mets à disposition tout un écosystème éducatif permettant aux développeurs et aux chercheurs d'obtenir de l'aide au sujet de leur produit et librairies. Dans le cadre de l'essai, un compte NVIDIA a été créé, permettant d'accéder au forum de développeurs, et les containers docker par exemple. Il est aussi possible d'accéder à du matériel éducatif grâce à l'institut DeepLearning de NVIDIA, dont l'accès a été commandité par le professeur Mickaël Germain, directeur de projet. Le forum de développeurs a été un outil utile dans le cadre de ce projet, car le dépôt d'une question m'a permis de me débloquer. Je n'étais pas capable de regénérer la version \acrshort{onnx} à partir du code source et de la documentation fournie par NVIDIA pour une architecture \acrshort{fcn}. Le développeur principal de l'application a répondu et m'a guidé dans la résolution du problème. Les autres ressources ont eu un impact limité dans le cadre de ce projet, puisque par exemple le container docker et DIGITS n'ont pas pu être utilisé. Le code source des architectures est disponible sans nécessiter de compte, de même que les \acrshort{sdk}s Jetpack.
\mysubparagraph{NVIDIA DIGITS}
\vspace{\baselineskip}
\\
\noindent NVIDIA fournit aux développeurs un environnement visuel permettant de réentrainer les architectures \acrshort{fcn} qu'ils fournissent avec leurs propres dataset. Cet environnement se nomme DIGITS. Malheureusement il est nécessaire d'avoir son propre matériel, le système d'exploitation Ubuntu 18.04 LTS, d'avoir au moins un \acrshort{gpu}. DIGITS ne s'installe pas sur le nano-ordinateur, ni sous Windows, ni avec la version "\acrshort{wsl}" (\acrlong{wsl}). Cette option a donc été abandonnée rapidement. 
\mysubparagraph{Docker NVIDIA}
\vspace{\baselineskip}
\\
\noindent NVIDIA fournit aux développeurs des containers docker, avec tout ce qui est nécessaire pour réentrainer une architecture et regénérer une version \acrshort{onnx}, par exemple. Malheureusement la capacité du nano-ordinateur ne permet pas de travailler efficacement avec un container docker, le nano-ordinateur devient sans réponse, nécessitant un redémarrage forcé. Cette option a donc été aussi abandonnée rapidement. 
\mysubparagraph{NVIDIA DeepStream}
\vspace{\baselineskip}
\\
\noindent Durant le déroulement de l'essai, NVIDIA a mis à disposition un environnement d'apprentissage profond, nommé "DeepStream", facilitant la conception et la génération de modèles, jusqu'à l'inférence. Cet outil n'a pas été évalué, mais pourrait être un outil alternatif pour réentrainer une architecture.