\noindent L'objet d'étude de cet essai est un nano ordinateur. Un nano ordinateur est un ordinateur miniaturisé en taille, mais aussi limité en capacité. Il existe différents fabricants et modèles, de caractéristiques techniques variées, pour répondre à différents besoins. Le dernier né est le modèle "Jetson Nano" du fabricant "NVIDIA", disponible depuis juin 2019 au prix abordable de 99 \$US. La compagnie NVIDIA a conçu ce matériel spécialement pour différentes applications d'inférence de modèles d'apprentissage profond sur une plateforme mobile (drone) ou proche des données ("edge" en anglais). Ce modèle a été choisi afin de répondre à l'intérêt que suscitent ses capacités et ses limites. Une image du Jetson Nano et un tableau de ses caractéristiques techniques sont disponibles. 
\vspace{\baselineskip}
\\
\noindent L'architecture matérielle est étudiée et présentée avec l'aide d'images, de diagrammes et de textes explicatifs. Les éléments clés sont identifiés.
\vspace{\baselineskip}
\\
\noindent Afin d'optimiser les performances du nano ordinateur, une recherche des périphériques les plus adaptés pour répondre aux besoins de performance (et de budget) de l'essai est essentielle, telle que l'alimentation, le stockage, la caméra. Des images des périphériques sont incluses, et les caractéristiques principales sont présentées dans des tableaux.
\vspace{\baselineskip}
\\
\noindent Le matériel est commandé par le collaborateur de cet essai "Vision météo".