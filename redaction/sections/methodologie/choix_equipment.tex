\par L'objet d'étude de cet essai est un nano-ordinateur. Un nano-ordinateur est un ordinateur miniaturisé en taille, mais aussi limité en capacité. Il existe différents fabricants et modèles, de caractéristiques techniques variées, pour répondre à différents besoins. Le dernier né est le modèle "Jetson nano" du fabricant "NVIDIA", disponible depuis juin 2019 au prix très abordable de 99\$US. La compagnie NVIDIA a conçu ce matériel spécialement pour différentes applications d'inférence de modèles d'apprentissage profond sur une plateforme mobile (drone) ou proche des données ("edge" en anglais). Ce modèle a été choisi afin de répondre à l'intérêt que suscitent ses capacités et ses limites. Une image du Jetson nano et un tableau de ses caractéristiques techniques seront disponibles. 
\par L'architecture matérielle sera étudiée et présentée avec l'aide d'images, de diagrammes et de textes explicatifs. Les éléments clés seront identifiés.
\par Afin d'optimiser les performances du Jetson nano, une recherche des périphériques les plus adaptés pour répondre aux besoins de performance (et de budget) de l'essai est essentielle, telle que l'alimentation, le stockage, la caméra. Des images des périphériques seront incluses, et les caractéristiques principales seront présentées dans des tableaux.
\par Le matériel est commandée par le collaborateur de cet essai "Vision météo".
\vspace{1\baselineskip}
\par 