\label{metho_eval}
\begin{figure}[H]
    \centering
    \includegraphics[width=1.0\textwidth]{metho_eval_perf_round_glass_shadow}
    \caption{Éléments pour l'évaluation des performances}
    \label{fig:metho_eval}
\end{figure}
\subsubsection{Stratégie de test}
\par L'objectif principal de l'essai est de déterminer la capacité et les limites du nano ordinateur d'inférer en temps réel des modèles de réseau de neurones à convolution entier pour la segmentation sémantique de vidéos. La stratégie qui sera appliquée sera de tester avec divers modèles et divers niveaux de qualité vidéos, en espérant trouver le compromis qui répond le mieux à cet objectif.
\begin{enumerate}
   \item \label{metho:testbaseinférence} Afin de s'assurer du bon fonctionnement du nano ordinateur et d'avoir des résultats de référence propre à notre environnement, l'inférence sera testée avec des modèles existants et pré entrainés pour la segmentation sémantique, avec les images et les vidéos provenant des références, et dont les caractéristiques et les résultats sont disponibles. 
   \item \label{metho:testbaseinférencesite} En espérant que les tests de l'étape \#\ref{metho:testbaseinférence} précédente donnent les résultats documentés dans les articles de références, ils seront repris avec les mêmes modèles, mais avec les images et les vidéos du site d'étude possédant la meilleure qualité acquise (1080p/i, 30FPS). Les données sources (images et vidéos) devront subir certains prétraitements à ce effet, afin de répondre aux requis des modèles.
   \item \label{metho:testdevinférencesite} Selon les résultats de l'étape \#\ref{metho:testbaseinférencesite}, les tests se concentreront sur l'inférence avec des vidéos, en réduisant progressivement la résolution (760p/i, 576p/i, 480p/i, 360p/i) et le nombre d'images par seconde (20FPS, 10FSP, 1FPS).
   \item Les étapes intermédiaires de l'étape \#\ref{metho:testdevinférencesite} précédente seront de 1) valider les résultats de l'inférence avec des images avant de tester avec les vidéos, et 2) évaluer si les modèles de réseaux de neurones à convolution entiers doivent et/ou peuvent être adaptés facilement, en tenant compte de l'échéancier de l'essai, et ce afin de répondre à l'objectif principal.
\end{enumerate}
\subsubsection{Stratégie de collecte des indicateurs de performance}
\par La méthodologie de la collecte des indicateurs est la suivante: 
\begin{itemize}
    \item La collecte est démarrée après un démarrage frais.
    \item Elle est consolidée dans un seul fichier.
    \item La date et l'heure du début de la collecte sont précisées.
    \item Au moment précis de début et de fin de chacun des tests, une entrée spéciale est ajoutée dans le fichier de collecte des indicateurs avec la date et l'heure, un libellé et une description. Ce point de contrôle (checkpoint) est fait grâce à une commande "shell" qui vient ajouter une trace dans le fichier de collecte des indicateurs.
    \item Chaque indicateur est collecté aux 5 secondes. Ce délai est réévalué au besoin, selon les fréquences des pics observées. 
\end{itemize}
\par Ensuite tout au long des tests de validations matériels et logiciels les indicateurs sont collectés. 
\par Chaque indicateur est une paire de clé=valeur. Il existe le même nombre d'indicateurs à tout moment. La date et l'heure sont un champ, ainsi que la source (tegrastats, System Monitor, user) et le champ descriptif. 
\par Avant tout début de tests, la collecte est démarrée. Cela permet de prendre une base de référence sans aucune charge.
\par Les indicateurs de performances sont collectés avant les tests matériels et logiciels lors du montage du nano ordinateur. Cela permet d'observer le "payload" possible d'un équipement ou d'un logiciel, sans lien avec l'inférence. Par exemple Chromium. Ou la caméra avec GStreamer. 
\par Ensuite les tests avec l'inférence débutent. D'abord par des tests de régression, pour s'assurer que l'environnement est toujours fonctionnel et retourne les mêmes résultats que les tests similaires précédents, que nos attentes sont respectées. 
\par Puis par des tests ciblés, en commençant par l'inférence avec des images de diverses résolutions. Et pour finir par l'inférence avec des vidéos de diverses résolutions. 
\par Les indicateurs collectés permettent de créer des graphiques qui montrent la progression de chacun avant, pendant et après le test.  Le min, max et avg peut être extrait.
\par Les performances matérielles du Jetson Nano sont évaluées grâce au System Monitor d’Ubuntu et les métriques tegrastats fournis par NVidia.
\par Les performances de la segmentation sont évaluées grâce au IoU et au z-score pour la classe du chemin / route. Une fonction Python est utilisée. Les fonctions IoU et le z-score utilisent l'image prédite (généré par le modèle FCN) et l'image Ground Truth. Les images originales sont donc présélectionnées selon leur intérêt et l'image Ground Truth créée. L'image prédite et Ground Truth doivent utiliser la même palette de couleurs et doivent être de la même résolution. 
L'image Ground Truth est créée à la main avec l'éditeur Gimp. Comme la segmentation de l'image prédite par le modèle de NVidia est très carrée, l'image Ground Truth ne sera pas précise au pixel prêt. Le besoin est d'évaluer et non d'entrainer, l'importance de la précision de la classification est moindre. 