\noindent La recherche de références s'est concentrée autour des concepts du sujet de l'essai : la segmentation sémantique, le temps réel, et les nano-ordinateurs. Le premier objectif a été de trouver si des études avaient déjà expérimenté le nano-ordinateur, en particulier pour la segmentation de vidéos en temps réel. Pendant cette recherche, j'en ai profité pour effectuer une révision de l'évolution des réseaux de neurones convolutionnels (\acrshort{cnn} \acrlong{cnn}) et des différentes architectures, et chercher d'autres solutions de détection de la route en temps réel grâce au \acrlong{fcn} (\acrshort{fcn}). 
\vspace{0.5\baselineskip}
\\
\noindent Il a été assez compliqué de trouver des références intégrant les nano-ordinateurs. Comme l'objectif de l'essai est de valider les performances d'un nano-ordinateur bien spécifique, les mots-clés "NVIDIA Jetson Nano" font partie de la stratégie de recherche. 
\vspace{0.5\baselineskip}
\\
\noindent Les réseaux de neurones pleinement connectés (\acrshort{fcn}) sont implicitement inclus dans les résultats puisque c'est l'état de l'art actuellement pour répondre au besoin de la segmentation sémantique d'images.
\vspace{0.5\baselineskip}
\\
\noindent Plus de 75 références ont été collectées. Une quarantaine ont été sélectionnées. Cette sélection peut se décomposer en trois catégories : 1) les références se rapprochant le plus du sujet de l'essai; 2) l'histoire et les antécédents des réseaux de neurones; 3) du matériel éducatif pour étudier et manipuler les réseaux de neurones.
\vspace{0.5\baselineskip}
\\
\noindent Je me suis intéressé aux références des années les plus récentes, autour de 2020, 2019 et 2018, car les avancées dans le domaine des réseaux de neurones sont rapides. Par curiosité je suis allé aussi parfois voir dans les années bien plus éloignées, comme 1998, ou j'ai trouvé un article proposant une solution pour prédire la température de la surface de la route avec des réseaux de neurones.
\vspace{0.5\baselineskip}
\\
\noindent Je n'ai pas pu trouver de références spécifiquement pour la déduction de l'état de la surface (mouillé, gelée, etc.) d'une piste multifonctionnelle (vélo, piéton).