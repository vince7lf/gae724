\par La revue de la littérature  a débuté en Octobre-Novembre 2019, c'est à dire quelques mois après la disponibilité du nano-ordinateur (Juin 2019). La recherche s'est concentrée sur des références traitant des concepts du sujet de l'essai : la segmentation sémantique, le temps réel, et les nano-ordinateurs. Le premier objectif a été de trouver si des études avaient déjà experimentés le nano-ordinateur, en particulier pour la segmentation de vidéos en temps réel. Pendant cette recherche, j'en ai profité pour effectuer une révision de l'évolution des réseaux de neurones convolutionnels entier (FCN Fully Convolutional Network)  et des différentes architectures, et chercher d'autres solutions de détection de la route en temps réel grâce au FCN. 
\par Il a été assez compliqué de trouver des références intégrant les nano-ordinateurs. Comme l'objectif de l'essai est de valider les performances d'un nano-ordinateur bien spécifique, les mots clés "NVIDIA Jetson nano" font partie de la stratégie de recherche. 
\par Les réseaux de neurones convolutifs entiers (FCN) sont implicitement inclus dans les résultats puisque c'est le "state-of-art" actuellement pour répondre au besoin de la segmentation sémantique d'images.
\par Plus de 75 références ont été collectées. Une quarantaine ont été sélectionnées. Cette sélection peut se décomposer en trois catégories : les références se rapprochant le plus du sujet de l'essai; l'histoire et les antécédents des réseaux de neurones; du matériel éducatif pour étudier et manipuler les réseaux de neurones.
\par Je me suis intéressé aux références des années les plus récentes, autour de 2020, 2019 et 2018, car les avancées dans le domaine des réseaux de neurones sont très rapides. Par curiosité je suis allé aussi parfois voir dans les années bien plus éloignées, comme 1998, ou j'ai trouvé un article proposant une solution pour prédire la température de la surface de la route avec des réseaux de neurones.
\par Je n'ai pas pu trouver de références spécifiquement pour la déduction de l'état de la surface (mouillé, gelée, etc.) d'une piste multifonctionnelle (vélo, piéton). 
\par Il est intéressant de noter que la banque de données SCOPUS retourne plus de 11,000 documents avec l'expression "segmentation AND "real-time"". Il y en a plus de 700 uniquement pour l'année 2019. 
\vspace{1\baselineskip}
\par 