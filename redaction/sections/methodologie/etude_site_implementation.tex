\par Le nano ordinateur est destiné à être déployé sur le chemin de la piste multi-fonctionnelle du pont Jacques-Cartier. L'étude du site a permis de chercher à comprendre, parmis ses caractéristiques, les difficultés de son usage l'hiver. Il sera tenter d'expliquer les défis et les raisons, techniques, politiques, sécuritaire, de pouvoir la conserver ouverte toute l'année. Une carte du site permettra de montrer un exemple de configuration où seront installés les nano-ordinateurs, et des images de ces  zones d'intérêt permettra de "visualiser" ce qui sera interprété par le modèle. 
\par Un mot sera réservé pour citer "L'Association des piétons et cyclistes du pont Jacques-Cartier" qui est un acteur actif pour le développement du transport actif dans cette région du Québec, et dont les membres sont des usagers habituels de la piste multifonctionnelle, même l'hiver.
\vspace{1\baselineskip}
\par 