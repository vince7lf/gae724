\noindent Le nano ordinateur est destiné à être déployé sur le chemin de la piste multifonctionnelle du pont Jacques-Cartier. L'observation du site d'implémentation a permis d'étudier, parmi ses caractéristiques, les difficultés de son usage l'hiver. Il existe des défis et des raisons, techniques, politiques, sécuritaires, de pouvoir la conserver ouverte toute l'année. Une représentation cartographique du site permet de montrer un exemple de configuration où et comment seront installés les nano ordinateurs, et des images de ces zones d'intérêt permet de visualiser ce qui sera interprété par l'architecture. 
\vspace{\baselineskip}
\\
\noindent Un mot est réservé pour citer L'\acrlong{apcpontjc} qui est un acteur actif pour le développement du transport actif dans cette région du Québec, et dont les membres sont des usagers habituels de la piste multifonctionnelle, même l'hiver.