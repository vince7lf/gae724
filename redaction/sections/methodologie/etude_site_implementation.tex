\par Le nano ordinateur est destiné à être déployé sur le chemin de la piste multifonctionnelle du pont Jacques-Cartier. L'étude du site a permis de chercher à comprendre, parmi ses caractéristiques, les difficultés de son usage l'hiver. Il a été tenté de comprendre les défis et les raisons, techniques, politiques, sécuritaires, de pouvoir la conserver ouverte toute l'année. Une carte du site permet de montrer un exemple de configuration où et comment seront installés les nano ordinateurs, et des images de ces  zones d'intérêt permet de "visualiser" ce qui sera interprété par le modèle. 
\par Un mot est réservé pour citer L'\acrlong{apcpontjc} qui est un acteur actif pour le développement du transport actif dans cette région du Québec, et dont les membres sont des usagers habituels de la piste multifonctionnelle, même l'hiver.