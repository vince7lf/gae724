\par 

Les ressources mises à disposition par le constructeur du Jetson nano, NVIDIA, ont été étudiées pour apprendre et tester le nano-ordinateur. Parmi les plus intéressantes, on peut citer le "Jetson Nano Developer Kit", le "NVIDIA Deep Learning Institute", la communauté Jetson, les tutoriels, les "benchmarks". Des jeux de données sont fournis gratuitement.
En complément des ressources de NVIDIA, deux références scientifiques ont été principalement utilisées comme points de départ et comme modèles pour l'essai, car leurs études ont été faites avec le Jetson nano (\cite{nguyen_mavnet_2019} et \cite{chong_real-time_1992}). Beaucoup de références ont été publiées ces deux dernières années sur le sujet de la segmentation sémantique, ils existent donc de multiples alternatives inspirantes.
Internet est une mine d'information. Il existe des forums et des blogues dans lesquels des utilisateurs publient leurs expérimentations de la segmentation sémantique en temps réel avec le Jetson nano (\cite{dustin_realtime_2019}), ou plus génériquement la segmentation sémantique. Des sites comme "modelzoo.co" sont des entrepôts de modèles déjà pré-entrainés. 
Les architectures des modèles FCN sélectionnés pour l'essai sont résumés dans un tableau récapitulatif, incluant leur type, leur application et leurs jeux de données respectifs, précisant les différentes variantes entre résolutions et nombre d'images pas secondes (FPS).

Les ressources mises à disposition par le constructeur du Jetson nano, NVIDIA, font référence à des jeux de données qui sont disponibles publiquement.
Des sites comme "modelzoo.co" ou "kaggle.com" sont des entrepôts de données. Une autre option est d'effectuer une recherche d'images ou de vidéos de la piste multifonctionnelle du pont Jacques-Cartier via les sites de recherche tels que Google. 
L'Association des piétons et cyclistes du pont Jacques-Cartier existe depuis de nombreuses années pour promouvoir le transport actif et conserver la piste multifonctionnelle du pont Jacques Cartier ouverte durant l'hiver. Ils fournissent, via leurs sites Internet, des collections de vidéos et d'images qui pourraient être utilisées. Il serait aussi possible d'entrer en contact avec l'association et leur demander de prendre de nouvelles vidéos. Voir "http://pontjacquescartier365.com", et\\"https://www.flickr.com/photos/pontjacquescartier".
Une autre possibilité serait d'hériter des acquisitions faites par un autre étudiant de l'université de Sherbrooke, soit déjà archivée, soit collectée prochainement. Mon directeur de projet Mickaël G. m'a informé qu'un étudiant de Sherbrooke va avoir besoin de collecter le trafic automobile sur le campus de l'Université de Sherbrooke, à Sherbrooke.
\vspace{1\baselineskip}
\par 