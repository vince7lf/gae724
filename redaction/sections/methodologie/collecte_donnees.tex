\par Le jeu d'images de DeepScene est celui qui semble le plus approprié car il a été conçu pour détecter les chemins dans la forêt. De plus il existe une version du modèle qui a été entrainé avec ce jeu. Comme il possède un jeu d'images vérité terrain (\acrshort{gt}), il sera bien utile pour évaluer la segmentation prédite et un gain de temps non négligeable dans le cadre d'un essai. Le jeu d'images de CityScape est très complet pour les scènes urbaines, mais comme il est moins spécialisé dans la détection de chemins ou de piste, son utilisation ne sera pas priorisé. Il contient toutefois des images vérité terrain de routes, ce qui est avantageux dans notre contexte et le favorise par rapport au deux derniers que nous avons à notre disposition. En effet le jeu d'images et de vidéos de l'Association des piétons et cyclistes du pont Jacques-Cartier, et celui que j'ai monté en prenant des vidéos de pistes cyclables de mon quartier, sont des jeux intéressants pour tester les résultats de la segmentation avec des images ou des vidéos qui viennent du site d'études, ou similaire, que celui de chemins forestiers. De plus ces images et vidéos sont loin des conditions parfaites (luminosité, qualité du sol, angle de vue, etc).
