\noindent La compagnie \acrlong{pjcci} (\acrshort{pjcci}) désire évaluer la mise en service de la piste multifonctionnelle (vélos, piétons, etc.) du pont Jacques-Cartier, à Montréal, durant l'hiver. Pour ce faire, la piste doit rester sécuritaire et dégagée, malgré les évènements météorologiques.
\vspace{\baselineskip}
\\
\noindent L'Université de Sherbrooke, qui participe à cette initiative, propose de mettre en place sur le pont une plateforme de détection innovatrice qui consiste à installer plusieurs paires d'objets connectés ultralégers et performants (des nano ordinateurs) à différents endroits du pont. Chacun de ces nano ordinateurs possède trois différents types de capteurs: vision; son; et météorologiques (température, humidité, etc.). Chaque nano ordinateur d'une paire perçoit le même environnement, mais d'une perspective différente que son homologue: la caméra pointe vers la même surface, mais d'un autre point de vue; les sons et les données météorologiques sont captés dans le même voisinage. Les données collectées par les capteurs sont traitées en temps réel par des algorithmes de segmentation qui sont adaptés à ce type de problématiques : les réseaux de neurones, du domaine de l'intelligence artificielle. La déduction de l'état de la surface de la piste (sèche, mouillée, glacée, etc.) se fait en fusionnant les différentes perceptions (multicibles) de chaque capteur (multicapteurs).
\vspace{\baselineskip}
\\
\noindent Cet essai se concentre sur le volet vision du projet pour \acrshort{pjcci}. L'idée est de déployer rapidement et facilement et en grande quantité \todo{environs ? 25 ?} des nano ordinateurs tout le long de la piste multifonctionnelle du pont. La caméra d'un nano ordinateur pointera tout simplement vers la piste multifonctionnelle, et le modèle \acrshort{ia} devra détecter automatiquement les délimitations (segmentation) de la piste cyclable à cet emplacement, en évitant la paramétrisation ou les réglages personnalisés pour chacun des systèmes lors de leur installation, tel que l'angle de vue, la distance, la hauteur. Les délimitations de la piste pourront être ensuite transmisent en temps réel à un autre programme installé sur le nano ordinateur afin de détecter, en temps réel également, les conditions de la surface de la piste multifonctionnelle: enneigée, mouillée, présence de glace noire, partiellement sêche, etc. Les résultats de la détection seront accessibles ou transmit via un accès à distance aux responsables de \acrshort{pjcci} afin qu'ils puissent prendre les décisions adéquates en matière d'entretien et d'accès. 
\vspace{\baselineskip}
\\
\noindent La détection d'objets en temps réel est de plus en plus précise et efficace depuis que les performances des systèmes informatisés permettent l'exécution d'algorithmes exigeants, en majeure partie depuis l'utilisation des processeurs graphiques "\acrshort{gpu}" \parencite{chong_real-time_1992, dettmers_deep_2015, beam_deep_2017, jiaconda_concise_2019, zheng_real-time_2020, kurenkov_brief_2015}. 
\vspace{\baselineskip}
\\
\noindent Les nano ordinateurs et les objets connectés ("\acrlong{iot}" ou "\acrshort{iot}") \parencite{blanco-filgueira_deep_2019, sharma_history_2019} sont le résultat de la miniaturisation des systèmes informatiques. Ils permettent la détection en temps réel à des endroits, dans des situations et dans des conditions qui n'étaient pas envisageables il y a encore 10 ans \parencite{zheng_real-time_2020, bernas_edge_2017, abouzahir_iot-empowered_2017, blanco-filgueira_deep_2019}.
\vspace{\baselineskip}
\\
\noindent Les réseaux de neurones ont aussi rapidement progressé depuis 2012 \parencite{beam_deep_2017}, permettant d'offrir des alternatives aux solutions de détection et de classifications \parencite{pathak_architecturally_2019}. Les réseaux de neurones pleinement connectés ("\acrshort{fcn}" en anglais, pour "\acrlong{fcn}") sont les derniers à avoir émergé et représente l'état de l'art (en anglais "state-of-art") \parencite{zheng_real-time_2020} et à profiter au domaine de la vision et de la détection d'objets \parencite{nguyen_mavnet_2019, zheng_real-time_2020}.
\vspace{\baselineskip}
\\
\noindent La segmentation sémantique est une forme de classification d'image, pixel par pixel, qui tire profit des dernières évolutions de la classification supervisée grâce aux réseaux de neurones pleinement connectés (\acrshort{fcn}), et qui peut être réalisée en temps réel avec des nano ordinateurs \parencite{long_fully_2015, blanco-filgueira_deep_2019}. Les images doivent être de haute résolution, ce qui nécessite d'avoir à disposition un système informatique capable de fournir une puissance de calcul appropriée, particulièrement pour la manipulation de la mémoire et des nombres flottants pendant l'inférence \parencite{mody_low_2018}. Leur application par des nano ordinateurs est un défi en raison de la faible consommation d'énergie (Watts) et de la puissance de calcul limité de ces derniers \parencite{copel_whats_2016}.
\vspace{\baselineskip}
\\
\noindent Il existe différents cadres applicatifs pour l'entrainement de modèles \acrshort{ia}, tel que PyTorch ou TensorFlow. Chacun d'eux nécessite d'installer leur propre environnement de développement et d'inférence, ce qui augmente les efforts et les coûts nécessaires. Le cadre applicatif ONNX permet d'uniformiser les architectures et ne requiert la mise en place que d'un seul cadre de travail lors de l'étape de l'opérationalisation. Le cadre applicatif livré par NVIDIA avec le Jetno Nano supporte les modèles convertis au format ONNX, et offre donc la fonction d'interopérationabilité des modèles \acrshort{ia}. 
\vspace{\baselineskip}
\\
\noindent Pour \acrshort{pjcci}, les avantages d'une telle plateforme seraient multiples, et on peut en énumérer plusieurs, sans se limiter à: contrôler et mesurer l'épandage de sel; surveiller à distance les conditions de la piste multifonctionnelle; éviter le déplacement d'un spécialiste; suivre les effets du gel et du dégel; optimiser les couts des opérations d'entretien (déplacements, quantité); offrir aux usagers des conditions d'accès sécurisées et optimales même en hiver; effets environnementaux atténués; prise de décision et gestion proactive; planification.
\vspace{\baselineskip}
\\
\noindent D'un autre côté, les défis ne sont pas à sous-évaluer: la détection doit être précise, fiable et consistante, tout cela afin d'assurer aux usagers un service de qualité dans un contexte sécuritaire.