\noindent La compagnie \acrlong{pjcci} (\acrshort{pjcci}) désire évaluer la mise en service de la piste multifonctionnelle (vélos, piétons, etc.) du pont Jacques-Cartier, à Montréal, durant l'hiver. Pour ce faire, la piste doit rester sécuritaire et dégagée, malgré les évènements météorologiques.
\vspace{0.5\baselineskip}
\\
\noindent L'Université de Sherbrooke, qui participe à cette initiative, propose de mettre en place sur le pont une plateforme de détection innovatrice qui consiste à installer plusieurs paires d'objets connectés ultralégers et performants (des nano-ordinateurs) à différents endroits du pont. Chacun de ces nano-ordinateurs possède trois différents types de capteurs: vision; son; et météorologiques (température, humidité, etc.). Chaque nano-ordinateur d'une paire perçoit le même environnement, mais d'une perspective différente que son homologue: la caméra pointe vers la même surface, mais d'un autre point de vue; les sons et les données météorologiques sont captés dans le même voisinage. Les données collectées par les capteurs sont traitées en temps réel par des algorithmes de segmentation qui sont adaptés à ce type de problématiques : les réseaux de neurones, du domaine de l'intelligence artificielle. La déduction de l'état de la surface de la piste (sèche, mouillée, glacée, enneigée, etc.) se fait en fusionnant les différentes perceptions (multicibles) de chaque capteur (multicapteurs).
\vspace{0.5\baselineskip}
\\
\noindent Cet essai se concentre sur le volet vision du projet pour \acrshort{pjcci}. Il s'agit de déployer rapidement, facilement, et en grande quantité (entre 25 et 50) des nano-ordinateurs tout le long de la piste multifonctionnelle du pont. Lors de la mise en service des nano-ordinateurs, leur caméra sera simplement orientée vers la piste multifonctionnelle, et le modèle \acrshort{ia} devra détecter automatiquement dès son exécution (inférence), et d'une façon continue (opérationnalisation 24/7), les délimitations (segmentation) de la piste, sans avoir à lui fournir des paramètres ou réglages personnalisés, tels que l'angle de vue, la distance ou la hauteur. Les délimitations de la piste pourront ensuite être transmises à un autre programme installé sur le nano-ordinateur afin de détecter en temps réel ou quasi-temps réel\footnote{Temps réel signifie qu'il n'y a pas de délai entre le traitement de la donnée, et la communication du résultat. Quasi-temps réel signifie qu'un faible délai est permis entre l'occurrence de l'évènement et la communication du résultat.}, les conditions de la surface de la piste multifonctionnelle: enneigée, mouillée, présence de glace noire, partiellement sèche, etc. Les résultats de la détection seront accessibles ou transmis via un accès à distance aux responsables de \acrshort{pjcci} afin qu'ils puissent prendre les décisions adéquates en matière d'entretien et d'accès. 
\vspace{0.5\baselineskip}
\\
\noindent Pour \acrshort{pjcci}, les avantages d'une telle plateforme seraient multiples, et on peut en énumérer plusieurs, sans se limiter à: contrôler et mesurer l'épandage de sel; surveiller à distance les conditions de la piste multifonctionnelle; éviter le déplacement d'un spécialiste; suivre les effets du gel et du dégel; optimiser les couts des opérations d'entretien (déplacements, quantité); offrir aux usagers des conditions d'accès sécurisées et optimales même en hiver; effets environnementaux atténués; prise de décision et gestion proactive; planification.
\vspace{0.5\baselineskip}
\\
\noindent D'un autre côté, les défis ne sont pas à sous-évaluer: la détection doit être précise, fiable et consistante, tout cela afin d'assurer aux usagers un service de qualité dans un contexte sécuritaire.