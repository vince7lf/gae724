\par La compagnie LES PONTS JACQUES CARTIER ET CHAMPLAIN INCORPORÉE (PJCCI) désire évaluer la mise en service de la piste multifonctionnelle (vélos, piétons, etc.) du pont Jacques-Cartier, à Montréal, durant l'hiver. Pour ce faire, la piste doit rester sécuritaire et dégagée, malgré les évènements météorologiques.
\par L'université de Sherbrooke, qui participe à cette initiative, propose de mettre en place sur le pont une plateforme de détection innovatrice qui consiste à installer plusieurs paires d'objets connectés ultralégers et performants (des nano-ordinateurs) à différents endroits du pont. Chacun de ces nano-ordinateurs possède trois différents types de capteurs: vision, son, et météorologiques (température, humidité, etc.). Chaque nano-ordinateur d'une paire perçoit le même environnement, mais d'une perspective différente que son homologue: la caméra pointe vers la même surface, mais d'un autre point de vue; les sons et les données météorologiques sont captés dans le même voisinage. Les données collectées par les capteurs sont traitées en temps réel par des algorithmes de détections performants qui sont adaptés à ce type de problématiques : les réseaux de neurones, du domaine de l'intelligence artificielle. La déduction de l'état de la surface de la piste (sèche, mouillée, glacée, etc.) se fait en fusionnant les différentes perceptions (multi-cibles) de chaque capteur (multi-capteurs).
\par L'objet principal de cet essai consiste a étudier la capacité du nano-ordinateur du fabricant NVIDIA, le Jetson nano \cite{nvidia_jetson_2019}, à exécuter, en temps réel, un modèle de réseau de neurones entrainé à faire de la segmentation sémantique (classification) d'images de haute résolution qui sont perçues avec la caméra. Les résultats de cette étude permettront de déterminer le modèle de réseau de neurones le plus adapté pour répondre aux besoins du volet vision du projet pour PJCCI. 
\par La détection d'objets et de surface en temps réel est de plus en plus précise et efficace depuis que les performances des systèmes informatisés permettent l'exécution d'algorithmes exigeants, en majeure partie depuis l'utilisation des processeurs graphiques "GPU" \cite{chong_real-time_1992} \cite{dettmers_deep_2015} \cite{beam_deep_2017} \cite{jiaconda_concise_2019} \cite{jia_real-time_2020} \cite{kurenkov_brief_2015}). 
\par Les systèmes informatiques performants sont de plus en plus miniatures, on parle de nano-ordinateurs et des objets connectés ("Internet of Things" ou "IoT") \cite{blanco-filgueira_deep_2019} \cite{sharma_history_2019}. Ils permettent la détection en temps réel à des endroits, dans des situations et dans des conditions qui n'étaient pas envisageables il y a encore 10 ans (\cite{jia_real-time_2020} \cite{bernas_edge_2017} \cite{abouzahir_iot-empowered_2017} \cite{blanco-filgueira_deep_2019}).
\par Les réseaux de neurones ont aussi très rapidement progressé depuis 2012 \cite{beam_deep_2017}, permettant d'offrir des alternatives aux solutions de détection et de classifications, entre autres \cite{pathak_architecturally_2019}. Les réseaux de neurones convolutifs entiers ("FCN" en anglais, pour "Fully Convolutional Network") sont les derniers à avoir émergé ("state-of-art") \cite{jia_real-time_2020} et à profiter au domaine de la vision et de la détection d'objets (\cite{nguyen_mavnet_2019} \cite{jia_real-time_2020}).
\par La segmentation sémantique est une forme de classification d'image, pixel par pixel, qui tire profit des dernières évolutions de la classification supervisée grâce aux réseaux de neurones convolutifs entiers, et se permet d'être déduite en temps réel avec des nano-ordinateurs (\cite{long_fully_2015} \cite{blanco-filgueira_deep_2019}). Les images doivent être de très haute résolution, ce qui nécessite d'avoir à disposition un système informatique capable de fournir une puissance de calcul appropriée, particulièrement pour la manipulation de la mémoire et des nombres flottants pendant l'inférence \cite{mody_low_2018}. Leur application par des nano-ordinateurs est un défi en raison de la faible consommation d'énergie (Watts) et de la puissance de calcul limité de ces derniers \cite{copel_whats_2016}.
\par Pour PJCCI, les avantages d'une telle plateforme seraient multiples, et on peut en énumérer plusieurs, sans se limiter à: contrôler l'épandage de sel; surveiller les conditions de la piste multifonctionnelle; suivre les effets du gel et du dégel; optimiser les couts des opérations d'entretien (déplacements, quantité); offrir aux usagers des conditions d'accès sécurisées et optimales même en hiver; effets environnementaux atténués; détecter ce qu'un spécialiste humain ne pourrait pas ou aurait des difficultés à détecter; prise de décision et gestion proactive; planification.
\par D'un autre côté, les défis ne sont pas à sous-évaluer: la détection doit être précise, fiable et consistante, tout cela afin d'assurer aux usagers un service de qualité dans un contexte sécuritaire.
