% command line to generate Word docx
% pandoc -s "GAE723 - Travail 4.3 Comptes-rendus des suivis individuels - Vincent Le Falher - v1.0.tex" -o "GAE723 - Travail 4.3 Comptes-rendus des suivis individuels - Vincent Le Falher - v1.0.tex.docx"
\documentclass[12pt, letterpaper]{article}
\title{Segmentation sémantique en temps réel à partir d'un nano ordinateur: étude des performances et des limites}
\author{Vincent Le Falher \thanks{Mickaël Germains}}
\date{2020}
\usepackage[hyphens]{url}
\usepackage{relsize}
\usepackage[utf8]{inputenc}
\usepackage{graphicx}
% \usepackage{wrapfig}
\usepackage{float}
\usepackage[french]{babel}
\usepackage{todonotes}
\usepackage{comment} 
\usepackage{times}
\usepackage[T1]{fontenc} 
\usepackage[margin=2.54cm]{geometry}
\usepackage{colortbl}
\usepackage{color}
\usepackage{xcolor}
\usepackage[nottoc,numbib]{tocbibind}
\usepackage{flafter} % to force the table to be within it's section 
\usepackage[backend=biber]{biblatex}
\usepackage{lscape}
\usepackage{longtable}
\usepackage{textcomp}
\usepackage{listings}
\usepackage{subcaption}
\usepackage{pdfpages}
\usepackage{grffile}
\usepackage{makecell}
% \usepackage{draftwatermark}
\addbibresource{selection-new.bib}
\definecolor{orange}{rgb}{1,0.647,0}
\graphicspath{ {images/} }
\addto\captionsfrench{ \renewcommand*\contentsname{Table des matières} }
\addto\captionsfrench{ \renewcommand{\listfigurename}{Liste des figures} }
\addto\captionsfrench{ \renewcommand{\listtablename}{Liste des tableaux} }
\addto\captionsfrench{ \renewcommand{\bibname}{Références}}
\newcommand{\myparagraph}[1]{\paragraph{#1}\mbox{}\\\par}
\newcommand{\mysubparagraph}[1]{\paragraph{\textit{#1}}\mbox{}\\\par}
\renewcommand{\tocbibname}{Bibliographie\label{chap:bib}}
\renewcommand*{\UrlFont}{\ttfamily\smaller\relax}
\linespread{1.213}
\begin{document}
\begin{titlepage} % Suppresses headers and footers on the title page
   \centering % Centre everything on the title page
 
   \scshape % Use small caps for all text on the title page
  
   DÉPARTEMENT DE GÉOMATIQUE APPLIQUÉE\\Faculté des lettres et sciences humaines\\Université de Sherbrooke
   
   \vspace{11\baselineskip}
 
   {\LARGE Segmentation sémantique en temps réel à partir d'un nano ordinateur: étude des performances et des limites}
   \vspace{1\baselineskip}
   \linebreak
 
   \vspace*{3\baselineskip}
 
   \textit{Essai présenté pour l'obtention du grade de Maître en sciences (M.Sc.),\\cheminement géodéveloppement durable}
   
   \vspace{5\baselineskip}
    
   {\LARGE Vincent Le Falher}
     
   \vfill % Whitespace between editor names and year
 
   Longueuil\\Septembre 2020 % Publication year
   \vfill % Whitespace between editor names and year
   % \footnotesize{\textcopyright{Vincent Le Falher, 2020}}
   \tiny{\textcopyright{Vincent Le Falher, 2020}}
 
\end{titlepage}
%----------------------------------------------------------------------------------------
\thispagestyle{empty}
%----------------------------------------------------------------------------------------
\textbf{Remerciements}
{\color{red}
\todo{TODO}
\vspace{\baselineskip}
\\
\noindent Je tiens à remercier ... 
\vspace{\baselineskip}
\\
\noindent 
}
%----------------------------------------------------------------------------------------
\newpage
\pagenumbering{Roman}
\tableofcontents{}
\listoffigures{}
\listoftables{}
%----------------------------------------------------------------------------------------
\newpage
\pagenumbering{arabic}
%----------------------------------------------------------------------------------------
\section{Suivi des changements au rapport}
\par Cette section n'est disponible que pendant la rédaction du rapport. Les changements au rapport sont documentés dans ce tableau. Cela permet de faire un suivi et permet d'aider les personnes qui révisent et commentent le rapport.
\par Les changements les plus récents apparaissent en premier. 
{
    \renewcommand*{\arraystretch}{1.4}
    \begin{table}[ht]
    \centering
    \caption{Suivi des changements}\label{table:changelog}
    \vspace{0.3em} % Adjust the height of the space between caption and tabular
    \begin{tabular}{{@{}|p{3em}|p{6em}|p{12cm}|@{}}}
    % \begin{tabular}{|c|c|c|} % does not work don't know why
        \hline
        \textbf{Version} & \textbf{Date} & \textbf{Changement}\\
        \hline
        v1.2.1& 27/28 juillet 2020 & \begin{itemize}
            \item Fixe les longues URL dans tout le document
            \item Bonifie la section Résultat avec un tableau des résolutions qui ont été testées, et spécifie avec quelle carte micro-ssd les tests de perfomances sont exécutés.
            \item Bonifie la section des Interprétation des résultats\Température avec le capteur AO et la durée de vie en opération constante. 
        \end{itemize}\\
        \hline
        v1.2.0& 26 juillet 2020 & \begin{itemize}
            \item Modification dans l'ensemble du document de l'expression "ground truth" par vérité terrain ("ground truth").
            \item Ajout dans les notes de bas de page des URLs (images du pont; github).
            \item Site d'étude: ajout des URL en note de base de page; enlève la deuxième image du pont; correction du nom de la station de métro la plus proche.  
            \item Matériel et logiciels/Le nano ordinateur : Ajout d'une référence à l'image.
            \item Méthodologie: référence à la documentation dans githug; ajout d'une sous-section Documentation; ajout des liens github dans les notes de bas de page.  
            \item Stratégie de collecte des indicateurs de performance: mise à jour majeure.
            \item Préparation du nano ordinateur/Disque SSD NVMe M.2 interne 250GB: mise à jour; ajout de notes de bas de page.
            \item Préparation du nano ordinateur/Caméra : précision de laquelle (Raspberry Pi v2)
            \item Résultats/Performances système: Précision du FCN + résolution sélectionnée lors du test.
            \item Résultats/Performances de l'inférence/Vidéo: ajout des liens vidéos en notes de bas de page ... mais trop long.
        \end{itemize}\\
        \hline
        v1.1.0 & 25 juillet 2020 & Section Résultats et Interprétation des résultats.\\
        \hline
        v1.0.2 & 19 juillet 2020 & Ajout dans le tableau des performances data read SSD vs micro-sd les 2 autres micro-sd utilisées pendant l'essai.\\
        \hline
        v1.0.1 & 18 juillet 2020 & \begin{itemize}
            \item Ajout d'une section changelog pour documenter le suivi des changements et faciliter la révision.
            \item Apport de clarification dans la section Matérielle \& logiciels (hdparm, présentation SDK, etc). 
            \item Début de la rédaction de la section résultat; inclusion d'un tableau des performances data read SSD vs micro-sd.
        \end{itemize}\\
        \hline
        v1.0 & 17 juillet 2020 & Première version envoyée à Michaël G. \\
        \hline
    \end{tabular}
    \end{table}
}
\clearpage
\newpage
\section{Introduction}
\subsection{Mise en contexte}
\noindent La compagnie \acrlong{pjcci} (\acrshort{pjcci}) désire évaluer la mise en service de la piste multifonctionnelle (vélos, piétons, etc.) du pont Jacques-Cartier, à Montréal, durant l'hiver. Pour ce faire, la piste doit rester sécuritaire et dégagée, malgré les évènements météorologiques.
\vspace{\baselineskip}
\\
\noindent L'Université de Sherbrooke, qui participe à cette initiative, propose de mettre en place sur le pont une plateforme de détection innovatrice qui consiste à installer plusieurs paires d'objets connectés ultralégers et performants (des nano ordinateurs) à différents endroits du pont. Chacun de ces nano ordinateurs possède trois différents types de capteurs: vision; son; et météorologiques (température, humidité, etc.). Chaque nano ordinateur d'une paire perçoit le même environnement, mais d'une perspective différente que son homologue: la caméra pointe vers la même surface, mais d'un autre point de vue; les sons et les données météorologiques sont captés dans le même voisinage. Les données collectées par les capteurs sont traitées en temps réel par des algorithmes de segmentation qui sont adaptés à ce type de problématiques : les réseaux de neurones, du domaine de l'intelligence artificielle. La déduction de l'état de la surface de la piste (sèche, mouillée, glacée, etc.) se fait en fusionnant les différentes perceptions (multicibles) de chaque capteur (multicapteurs).
\vspace{\baselineskip}
\\
\noindent Cet essai se concentre sur le volet vision du projet pour \acrshort{pjcci}. L'idée est de déployer rapidement et facilement et en grande quantité \todo{environs ? 25 ?} des nano ordinateurs tout le long de la piste multifonctionnelle du pont. La caméra d'un nano ordinateur pointera tout simplement vers la piste multifonctionnelle, et le modèle \acrshort{ia} devra détecter automatiquement les délimitations (segmentation) de la piste cyclable à cet emplacement, en évitant la paramétrisation ou les réglages personnalisés pour chacun des systèmes lors de leur installation, tel que l'angle de vue, la distance, la hauteur. Les délimitations de la piste pourront être ensuite transmisent en temps réel à un autre programme installé sur le nano ordinateur afin de détecter, en temps réel également, les conditions de la surface de la piste multifonctionnelle: enneigée, mouillée, présence de glace noire, partiellement sêche, etc. Les résultats de la détection seront accessibles ou transmit via un accès à distance aux responsables de \acrshort{pjcci} afin qu'ils puissent prendre les décisions adéquates en matière d'entretien et d'accès. 
\vspace{\baselineskip}
\\
\noindent La détection d'objets en temps réel est de plus en plus précise et efficace depuis que les performances des systèmes informatisés permettent l'exécution d'algorithmes exigeants, en majeure partie depuis l'utilisation des processeurs graphiques "\acrshort{gpu}" \parencite{chong_real-time_1992, dettmers_deep_2015, beam_deep_2017, jiaconda_concise_2019, zheng_real-time_2020, kurenkov_brief_2015}. 
\vspace{\baselineskip}
\\
\noindent Les nano ordinateurs et les objets connectés ("\acrlong{iot}" ou "\acrshort{iot}") \parencite{blanco-filgueira_deep_2019, sharma_history_2019} sont le résultat de la miniaturisation des systèmes informatiques. Ils permettent la détection en temps réel à des endroits, dans des situations et dans des conditions qui n'étaient pas envisageables il y a encore 10 ans \parencite{zheng_real-time_2020, bernas_edge_2017, abouzahir_iot-empowered_2017, blanco-filgueira_deep_2019}.
\vspace{\baselineskip}
\\
\noindent Les réseaux de neurones ont aussi rapidement progressé depuis 2012 \parencite{beam_deep_2017}, permettant d'offrir des alternatives aux solutions de détection et de classifications \parencite{pathak_architecturally_2019}. Les réseaux de neurones pleinement connectés ("\acrshort{fcn}" en anglais, pour "\acrlong{fcn}") sont les derniers à avoir émergé et représente l'état de l'art (en anglais "state-of-art") \parencite{zheng_real-time_2020} et à profiter au domaine de la vision et de la détection d'objets \parencite{nguyen_mavnet_2019, zheng_real-time_2020}.
\vspace{\baselineskip}
\\
\noindent La segmentation sémantique est une forme de classification d'image, pixel par pixel, qui tire profit des dernières évolutions de la classification supervisée grâce aux réseaux de neurones pleinement connectés (\acrshort{fcn}), et qui peut être réalisée en temps réel avec des nano ordinateurs \parencite{long_fully_2015, blanco-filgueira_deep_2019}. Les images doivent être de haute résolution, ce qui nécessite d'avoir à disposition un système informatique capable de fournir une puissance de calcul appropriée, particulièrement pour la manipulation de la mémoire et des nombres flottants pendant l'inférence \parencite{mody_low_2018}. Leur application par des nano ordinateurs est un défi en raison de la faible consommation d'énergie (Watts) et de la puissance de calcul limité de ces derniers \parencite{copel_whats_2016}.
\vspace{\baselineskip}
\\
\noindent Il existe différents cadres applicatifs pour l'entrainement de modèles \acrshort{ia}, tel que PyTorch ou TensorFlow. Chacun d'eux nécessite d'installer leur propre environnement de développement et d'inférence, ce qui augmente les efforts et les coûts nécessaires. Le cadre applicatif ONNX permet d'uniformiser les architectures et ne requiert la mise en place que d'un seul cadre de travail lors de l'étape de l'opérationalisation. Le cadre applicatif livré par NVIDIA avec le Jetno Nano supporte les modèles convertis au format ONNX, et offre donc la fonction d'interopérationabilité des modèles \acrshort{ia}. 
\vspace{\baselineskip}
\\
\noindent Pour \acrshort{pjcci}, les avantages d'une telle plateforme seraient multiples, et on peut en énumérer plusieurs, sans se limiter à: contrôler et mesurer l'épandage de sel; surveiller à distance les conditions de la piste multifonctionnelle; éviter le déplacement d'un spécialiste; suivre les effets du gel et du dégel; optimiser les couts des opérations d'entretien (déplacements, quantité); offrir aux usagers des conditions d'accès sécurisées et optimales même en hiver; effets environnementaux atténués; prise de décision et gestion proactive; planification.
\vspace{\baselineskip}
\\
\noindent D'un autre côté, les défis ne sont pas à sous-évaluer: la détection doit être précise, fiable et consistante, tout cela afin d'assurer aux usagers un service de qualité dans un contexte sécuritaire.
\subsection{Problématique}
\noindent Dans le cadre du projet pour \acrshort{pjcci}, une plateforme technologique sera mise à la disposition des gestionnaires du pont afin de les aider à prendre les décisions les plus responsables et raisonnables possibles. Mais la mise en opération d'une solution innovante et fiable, qui concilie des algorithmes d'apprentissage profond, du temps réel, des nano ordinateurs, et des conditions climatiques variables, est complexe. Dans une certaine mesure, l'essai va contribuer à la recherche de solutions afin de répondre au défi pour le domaine du transport actif et durable d'être soutenu par des solutions technologiques fiables (opérationnelles), l'objectif étant de pouvoir offrir des services de qualité et sécuritaires sur l'ensemble des quatre saisons.
\vspace{\baselineskip}
\\
\noindent La seconde problématique que l'essai va contribuer à résoudre concerne les limites d'un nano ordinateur. L'inférence nécessite une architecture et une puissance machine différente de celle nécessaire pour l'entrainement. Les architectures de réseaux de neurones sont adaptées et optimisées pour l'inférence. L'essai va permettre de préciser les capacités du nano ordinateur pour l'inférence de diverses architectures de réseaux de neurones pleinement connectés ("\acrshort{fcn}" en anglais) et la segmentation sémantique en temps réel avec des vidéos de différentes propriétés (résolutions et nombre d'images pas seconde). Il existe des tests encourageants \parencite{nvidia_jetson_2019-1, nguyen_mavnet_2019,zheng_real-time_2020}, qui sont utilisés comme base de travail et référence, même si ceux-ci sont limités à des types d'application qui ne sont pas les mêmes que pour l'essai.
\vspace{\baselineskip}
\\
\noindent Il est difficile de trouver des jeux de données pour entrainer les réseaux de neurones pleinement connectés (\acrshort{fcn}) adaptés à la problématique. La technique de "Data augmentation" permet de démarrer d'une architecture qui a déjà appris avec un jeu d'images important (milliers d'images), et de lui faire apprendre davantage, en lui fournissant un plus petit jeu d'images (centaines d'images) de la nouvelle zone d'étude. Par exemple une architecture peut avoir appris à classifier des images de la Californie, États-Unis. Pour lui permettre de classifier des images de la Ville de Sherbrooke, il est souhaitable de lui fournir un nouveau jeu de données spécifique à cette ville afin qu'il s'adapte (ses paramètres) à cette région. Dans le contexte de cet essai, les données acquises sur le terrain sont fournies aux différents modèles qui sont évalués, et les architectures sont ré entrainées avec ce nouveau jeu d'images adapté à la zone d'étude.
\vspace{\baselineskip}
\\
\noindent La paramétrisation (des "hyper paramètres") des réseaux de neurones est "subtile" et "intuitive" et requière de l'expérience. C'est un processus d'essais-erreurs qui est couteux en temps, et risqué puisqu'il n'y a aucune garantie de succès. La technique d'apprentissage par transfert ("Transfer Learning" en anglais) permet d'hériter d'une architecture qui est déjà entrainée et configurée, et de l'adapter pour répondre à ses besoins. Cette technique permet un gain en temps et en énergie (et en argent) important puisque le temps de conception (architecture et configuration) et le temps d'entrainement, de validation et de tests sont diminués de façon non négligeable.
\begin{comment}
Par exemple l'architecture "VGG" prend 2-3 semaines d'entrainement \parencite{simonyan_very_2015} avec 4 \acrshort{gpu} Titan Black (NVIDIA), coutant 1,200 \$US (Amazon.com) chacun (pour un total de 4,800\$US, et cela juste pour les \acrshort{gpu}s, qui ne sont qu'un des éléments de l'infrastructure nécessaire). Étant donné que de multiples tentatives sont nécessaires (cycles essai-erreur), la stratégie est d'entrainer plusieurs modèles en parallèle afin d'accélérer le développement, ce qui implique un coût élevé en infrastructure.
\end{comment}
La problématique pour l'essai est de trouver l'architecture qui est la plus adaptée pour répondre au besoin, et il en existe des milliers \parencite{koh_model_2018}. La recherche dans la littérature permet heureusement de limiter les choix et donner des pistes \parencite{zheng_real-time_2020, nguyen_mavnet_2019, nvidia_jetson_2019-1}. La problématique de la conception existe toujours, car l'architecture a besoin d'être étudiée, adaptée et ré entrainée, jusqu'à l'obtention de résultats probants. Mais la paramétrisation des hyper paramètres n'est plus nécessaire (supposément), ce qui est avantageux.
\vspace{\baselineskip}
\\
\noindent Même si les tests du modèle donnent des résultats satisfaisants en théorie, la réalité du terrain peut surprendre. Le test du modèle doit se faire dans des conditions réelles avec de nouvelles données (images), celles qui sont captées par le système hôte sur le terrain d'implantation: dans le jargon de l'intelligence artificielle, c'est l'"inférence" \parencite{copel_whats_2016, nvidia_jetson_2019-1}. Il est assez probable que l'architecture doive retourner à une phase de ré entrainement. De plus, le système hôte, dans notre cas le nano ordinateur NVIDIA Jetson Nano, est conçu avec une architecture matérielle limitée (\acrshort{gpu}, \acrshort{cpu}s, mémoire, taux de transfert, alimentation) et verra, au besoin, son architecture matérielle adaptée et remise en question.
\subsection{Objectifs}
\noindent L'objectif principal de cet essai consiste a étudier la capacité du nano ordinateur du fabricant NVIDIA, le Jetson Nano \parencite{nvidia_jetson_2019}, à exécuter, en temps réel, une architecture de réseau de neurones pleinement connectés (\acrshort{fcnn}) entrainée à faire de la segmentation sémantique d'images et de vidéos de hautes résolutions qui sont perçues avec la caméra. Une seule classe sera extraite, celle représentant la piste multifonctionnelle. Les autres classes ne seront pas utilisées. Il semble important de préciser que l'objectif de l'essai n'est pas d'évaluer la précision des modèles (\acrshort{iou},  F1 score) produisant la segmentation sémantique, mais de déterminer, et ce en rapport avec les attentes du projet pour \acrshort{pjcci}, de la viabilité de pouvoir extraire la segmentation en temps réel à partir d'une vidéo de haute qualité avec le Jetson Nano dans un mode opérationnel 24/7, et de transmettre les délimitations de la piste multifonctionnelle à un autre programme pour détecter les conditions de la surface.
\vspace{\baselineskip}
\\
\noindent Les sous-objectifs sont les suivants: 
\begin{itemize}
   \item Évaluer les limites de la plateforme, matérielle et applicative.
   \item Évaluer les moyens d'optimiser la plateforme d'un point de vue matériel et applicatif. 
   \item Évaluer la possibilité de pouvoir ré entrainer l'architecture sur le nano ordinateur dans une perspective d'apprentissage actif et continue.
   \item Ré entrainer une architecture \acrshort{fcnn} avec les images du site d'implémentation.
   \item Permettre un accès à distance sécurisé au nano ordinateur.
   \item Documenter l'approche, les tests, et les résultats;
\end{itemize}
\vspace{\baselineskip}
\noindent Il n'est pas planifier de faire des tests sur le site d'implémentation, ni s'intégrer avec d'autres programmes du projet pour \acrshort{pjcci}, par exemple pour détecter les conditions de la surface de la piste multifonctionnelle. 
\vspace{\baselineskip}
\\
\noindent Le premier sous-objectif est de déterminer quelles sont les limites de la plateforme, d'un point de vue matériel (\acrshort{gpu}, \acrshort{cpu}s, mémoire, transfert mémoire, consommation, etc.), mais aussi applicatif, d'un point de vue inférence. Cette phase du projet va permettre d'exécuter tel quel différents modèles d'architecture déjà existants, sans les ré entrainer, en tenant compte des éléments documentés dans la littérature \parencite{nguyen_mavnet_2019, zheng_real-time_2020, nvidia_jetson_2019-1}.
\vspace{\baselineskip}
\\
\noindent Un autre sous-objectif est d'optimiser ou d'adapter la plateforme, d'un point de vue matériel, mais aussi applicatif, afin d'avoir les meilleures performances et résultats possibles pendant l'inférence.
\vspace{\baselineskip}
\\
\noindent L'un des intérêts de l'\acrshort{ia} est de pouvoir améliorer constamment les modèles grâce au ré entrainment continue. L'essai va évaluer la possibilité de bénéficier de cet avantage directement sur le nano ordinateur en tentant de ré entrainer activement l'architecture avec des images de la piste multifonctionnelle re segmentées par un expert, et re générer un modèle plus précis, tout ceci en concurence avec l'inférence en temps réel. 
\vspace{\baselineskip}
\\
\noindent Comme les résultats devront être disponibles en tout temps, une connexion à distance sécurisée devra être mise en place. Cette connexion permettra aussi de pouvoir prendre le contrôle du nano ordinateur à distance et de l'administrer. En effet, le nano ordinateur sera déployé sur le site d'implémentation sans les périphériques standards, tel qu'un clavier, souris ou écran. Le type de réseau adéquat, soit Ethernet ou cellulaire (carte SIM réseau 3g/4g), sera évalué.
\vspace{\baselineskip}
\\
\noindent L'approche, les tests, et les résultats sont documentés. Il y aura beaucoup d'activités relatives à la conception et aux tests, le cheminement complet n'est pas fourni. Une synthèse est préférée et les informations les plus pertinentes sont incluses. Les détails de l'installation de l'environnement de développement et des applications, librairies et autres dépendances nécessaires sont inclus, ainsi que ceux de la configuration. Dans le cas où l'objectif principal n'est pas atteint, ou partiellement, la/les raison/s de l'échec sont spécifiées et des pistes de solutions potentielles proposées.
%----------------------------------------------------------------------------------------
\section{Cadre théorique (état des connaissances, revue de la littérature)}
Il y a deux sections, la première qui concerne le nano ordinateur et ensuite la seconde, l'apprentissage profond et la segmentation sémantique.
\subsection{Cadre théorique au sujet du nano ordinateur}
\noindent Le nano ordinateur de cet essai doit être compris comme étant un ordinateur miniature, ayant une taille et des capacités qui lui permettent d'être installé ("embedded system") dans une voiture, un drone, un tracteur ou être accroché à un poteau. Le terme anglais "On the Edge" (sur le bord), s'y approprie mieux que "\acrshort{iot}" ("\acrlong{iot}"), puisqu'étant sur le terrain il se trouve directement proche des données, ce qui lui donne l'avantage de pouvoir faire des traitements en temps réel. Les premiers systèmes embarqués reconnus comme tels, sont ceux installés dans le missile Minuteman (1961; ref\todo{TODO}) et la navette Apollo (1960; ref\todo{TODO}). Les avancées technologiques ont permis de les rendre de plus en plus compactes et performants. Les systèmes de la compagnie Campbell Scientific existent depuis les années 1974 et pemettent l'acquisition de données à distance. Le système Arduino est l'un des premiers microprocesseurs a avoir été destinés à la robotique. Le Jetson Nano de NVIDIA est le dernier né des nano ordinateurs de la compagnie NVIDIA permettant d'inférer en temps réel des architectures d'intelligence artificielle, sans ajout de périphériques. Du même constructeur, ses grands frères sont le Jetson Xavier and le Jetson TX2, plus performants, et donc plus onéreux. Son concurrent direct est le Raspberry Pi, mais il nécessite une extension USB Movidius Intel pour l'inférence de modèles \acrshort{ia}. 
\vspace{\baselineskip}
\\
\noindent Ce qui caractérise principalement un ordinateur miniature, est le fait qu'il soit assez petit pour pouvoir être embarqué dans un système plus gros, tel qu'un robot ou du matériel médical. Son coût est bas, en raison des performances qui sont limitées par une conception répondant à un besoin spécifique. Tous les éléments matériels requis sont contenus sur une même carte. Une fois installé et paramétré, le système se doit d'être fiable et opérationnel sur le long terme. Mais il doit aussi être interchangeable, au besoin, rapidement et facilement. La consommation électrique est faible, entre 5 et 10 watt. Étant généralement opérationnel sur le terrain, proche des données, il est responsable d'une tâche bien particulière, qu'il doit accomplir efficacement. Il n'y a généralement pas d'interface utilisateur, et l'accès au système se fait à distance ou via une console. Il est composé de capteurs, au besoin d'une caméra. Le même système peut-être déployé en grande quantité, comme dans le contexte de notre essai, où plusieurs paires seront déployées le long de la piste multifonctionnelle; un autre exemple est celui des constellations de nano satellites.
\vspace{\baselineskip}
\\
\noindent L'annexe \ref{annexe:nano_computer_samples} montrent les nano ordinateurs qui supportent les \acrshort{sdk} pour l'\acrshort{ia}.

\subsection{Cadre théorique au sujet de l'apprentissage profond et de la segmentation sémantique}
\noindent La segmentation sémantique d'images ou de vidéos est une technique de télédétection du domaine de la vision par ordinateur. Elle permet de délimiter (segmenter) différentes parties (sémantique) d'une image. Les méthodes de segmentation ont été améliorées ces dernières années par les récentes avancées dans le domaine de l'apprentissage profond. 
\begin{figure}[H]
   \centering
   \includegraphics[width=0.75\textwidth]{semantic_segmentation_vs_others}
   \caption[Segmentation semantic]{Segmentation semantic\parencite[p.~1]{wu_recent_2019}}
   \label{fig:semantic_segmentation_vs_others}
\end{figure}
\noindent L’apprentissage profond est un sous-domaine de celui de l'apprentissage machine qui est un sous-domaine de celui de l'intelligence artificielle. 
\begin{figure}[H]
   \centering
   \includegraphics[width=0.5\textwidth]{Deep_Learning_with_Python.pdf}
   \caption[Relation entre \acrlong{ia}, \acrlong{am} et \acrlong{ap}]{Relation entre \acrlong{ia}, \acrlong{am} et \acrlong{ap} \parencite[p.~4]{chollet_deep_2018}}
   \label{fig:ia_ml_ap}
\end{figure}
\noindent Les concepts de l'\lowercase{\acrlong{ia}} (AI) existent depuis les années 1950 \parencite[p.~4]{chollet_deep_2018} \parencite[p.~1]{alom_history_2018}, et ont continué à se développer par vague, jusqu'à leur nouvelle popularité des 15 dernières années. En effet, trois raisons principales ont permis à ce domaine de renaitre de nouveau \parencite[p.~20]{chollet_deep_2018}: 1) la capacité et la puissance des machines; 2) des jeux de données plus larges; 3) des algorithmes plus avancés. Les deux moments clés, preuves de cette renaissance, sont: 1) la possibilité d'entrainer des architectures de réseaux de neurones profonds (DNN) (2006) \parencite[p.~6]{alom_history_2018}; et 2) l'architecture du réseau de neurones convolutionels AlexNet permet de  gagner le challenge ImageNet contre les approches traditionnelles\parencite[p.~11]{alom_history_2018}. 
\vspace{\baselineskip}
\\
\noindent En 2016 \parencite[p.~14]{alom_history_2018}, l'architecture \acrshort{fcn} (\acrlong{fcn}, réseau (de neurones) convolutionnel entier) a permis aux taches réservées à la segmentation d'images d'être plus efficace que les méthodes traditionnelles de la vision par ordinateur. Cette nouvelle méthode s'applique désormais à tous les domaines connexes à l'analyse d'images, tels que l'imagerie médicale, la conduite autonome de véhicules, la robotique, la télédétection d'images satellites, la sécurité par caméra vidéo, l'agriculture de haute précision. Aujourd'hui, elle peut s'exécuter en temps réel sur des systèmes embarqués proche des données. 
%----------------------------------------------------------------------------------------
\section{Matériel et méthodes}
\subsection{Site d'étude}
\par Voici le plan qui est utilisé pour rédiger au sujet du site d'étude.
\begin{itemize}
   \item Brève présentation du pont Jacques-Cartier, et de la piste multifonctionnelle;
   \item Présentation des difficultés de l'usage de la piste l'hiver et des défis et raisons (technique, politique, sécurité) de pouvoir la conserver ouverte toute l'année, en lien avec les objectifs de l'essai 
\end{itemize}
\vspace{1\baselineskip}
\par 
\subsection{Modèles et jeux de données}
\myparagraph{Données}\label{metho:data}
\todo{TODO review}
\par Les ressources mises à disposition par le constructeur du Jetson nano, NVIDIA, font référence à des jeux de données qui sont disponibles publiquement.
\par En complément des ressources de NVIDIA, deux références scientifiques seront principalement étudiées, car leurs recherches ont été faites avec le Jetson nano (\cite{nguyen_mavnet_2019} et \cite{chong_real-time_1992}). Beaucoup de références ont été publiées ces deux dernières années sur le sujet de la segmentation sémantique, ils existent donc de multiples alternatives inspirantes.
\par Internet est une mine de données : il existe des forums et des blogues dans lesquels des utilisateurs publient leurs expérimentations de la segmentation sémantique en temps réel avec le Jetson nano (\cite{dustin_realtime_2019}), ou plus génériquement la segmentation sémantique; des sites comme "modelzoo.co" ou "kaggle.com" sont des entrepôts de données et de modèles FCN prêts à être utilisés; une autre option a été d'effectuer une recherche d'images ou de vidéos de la piste multifonctionnelle du pont Jacques-Cartier via les sites de recherche tels que Google. 
\par L'Association des piétons et cyclistes du pont Jacques-Cartier existe depuis de nombreuses années pour promouvoir le transport actif et conserver la piste multifonctionnelle du pont Jacques Cartier ouverte durant l'hiver. Ils fournissent, via leurs sites Internet, des collections de vidéos et d'images qui seront utilisées après leur avoir demandé leur autorisation verbale et écrite. \cite{association_des_pietons_et_cyclistes_du_pont_jacques-cartier_pontjacques-cartier365com_2020} \cite{association_des_pietons_et_cyclistes_pont_jacques-cartier_flickr_2020}
\myparagraph{Approche prévue pour le traitement des données}
\par Il y a deux phases à cet essai: l'inférence avec des modèles déjà prêts et l'inférence avec des modèles ré entrainés. Les données utilisées par l'inférence sont des vidéos (d'une certaine résolution et d'un certain nombre d'images pas seconde), et celles pour l'entrainement sont des images. Dans les deux cas, les images pour l'entrainement ou l'inférence doivent être d'une taille bien précise, celles avec lesquelles le modèle a été, ou sera, entrainées. La résolution et la qualité de l'image vidéo seront nivelées vers le bas afin de déterminer la limite inférieure acceptable pour la détection la plus efficace et fiable possible. La résolution et le nombre d'images par seconde de la vidéo sont contrôlés par le logiciel ("driver" en anglais) de la caméra, et sont configurables. 
\par Tout cela signifie que les vidéos ou nouvelles images devront être traités pour répondre à une certaine taille et résolution requise par le modèle, tout en conservant une qualité élevée (nombre de pixels, niveaux de couleurs). De nouvelles images pour l'entrainement seront extraites des vidéos, et annotées. 
\par Certains cadres d'application logicielle ("framework") d'apprentissage profond (par exemple "Keras") offrent l'option d'augmenter automatiquement le jeu de données avec des techniques d'augmentation de données (par exemple la rotation, le redimensionnement, l'effet miroir), ce qui est très utile et non négligeable.
\par Voici le tableau de synthèse des données, incluant la référence avec leurs réseaux de neurones.
{
   \clearpage 
   \newpage
   \begin{landscape}
   \newcounter{magicrownumbers}
   \newcommand\rownumber{\stepcounter{magicrownumbers}\arabic{magicrownumbers}}
   \vspace{0.3em} % Adjust the height of the space between caption and tabular
   \begin{longtable}[t]{@{}p{1em}|p{15em}p{35em}@{}} % p{15em}p{35em} with landscape
      \caption{Tableau des données}\label{tab:datasets}\\
      & \textbf{Spécification} & \textbf{Description}\\
      \hline
      \endfirsthead
      & \textbf{Spécification} & \textbf{Description}\\
      \hline
      \endhead
      \endfoot
      \endlastfoot
      \hline
      \rownumber & \begin{tabular}[t]{@{}p{15em}@{}}
         réseau: SegNet\\jeu de données: CamVid\\vidéo: 10 minutes\\résolution/s: HD
      \end{tabular} & \begin{tabular}[t]{@{}p{35em}@{}}
         SegNet est un réseau qui a été créé pour la segmentation sémantique de vidéos. Il a été entrainé avec le jeu de données de CamVid, qui procurent des vidéos de la route avec la même perspective que le conducteur du véhicule. Un modèle entrainé est disponible pour le Jetson nano.\\
         \url{https://github.com/PengKiKi/camvid}\\
      \end{tabular}\\
      \hline
      \rownumber & \begin{tabular}[t]{@{}p{15em}@{}}
         réseau: MFANet\\jeu de données: Cityscapes\\nombre d'images: 5000\\résolutions: 1280x1024
      \end{tabular} & \begin{tabular}[t]{@{}p{35em}@{}}
         MFANet est un réseau qui a été créé en 2019 pour la segmentation sémantique sur des appareils tel que le Jetson nano. Il a été entrainé avec le jeu de données de Cityscapes, qui procurent des images de scènes urbaines. Différentes stratégies d'augmentation de données sont utilisées. Des tests ont été faits avec le Jetson nano.\\
         leejy@ustb.edu.cn\\
      \end{tabular}\\
      \hline
      \rownumber & \begin{tabular}[t]{@{}p{15em}@{}}
         réseau: RESNet18\\jeu de données: Cityscapes\\nombre d'images: 25 000\\résolutions: 360x720, 512x256, 1024x512, 2048x1024
      \end{tabular} & \begin{tabular}[t]{@{}p{35em}@{}}
         Cityscapes est un jeu de données qui fournit des images de rues spécifiquement destinées pour la segmentation sémantique. Il peut être utilisé par de nombreux réseaux. RESNet18 a été entrainé avec ce jeu et est disponible en diverses résolutions pour le Jetson Nano.\\
         \url{https://github.com/tynguyen/MAVNet/tree/master/data/perch_drone}\\
      \end{tabular}\\
      \hline
      \rownumber & \begin{tabular}[t]{@{}p{15em}@{}}
         réseau: RESNet18\\jeu de données: DeepScenes\\nombre d'images: 15 000\\résolutions: 576x320, 864x480 
      \end{tabular} & \begin{tabular}[t]{@{}p{35em}@{}}
         DeepScene propose un modèle et un jeu de données. Le modèle est entrainé avec différents jeux de données, comme Cityscpapes, SUN-RGBD, Synthia. Le jeu de données fournit des images de forêt, qui est destinée pour la segmentation sémantique. RESNet18 a été entrainé avec ce jeu et est disponible en deux  résolutions pour le Jetson Nano.\\
         \url{http://deepscene.cs.uni-freiburg.de}\\
      \end{tabular}\\
      \hline
      \rownumber & \begin{tabular}[t]{@{}p{15em}@{}}
         réseau: DeepScene\\jeu de données: Synthia\\nombre d'images: 220 000\\résolutions: 1280x760
      \end{tabular} & \begin{tabular}[t]{@{}p{35em}@{}}
         Le jeu de données Synthia fournit des images (et vidéos) de scènes de rue comme celui de Cityscapes, et qui est destiné pour la segmentation sémantique. DeepScene a été entrainé avec ce jeu. Il n'a pas été testé avec le Jetson Nano.\\
         \url{http://3dvision.princeton.edu/datasets.html}\\
      \end{tabular}\\
      \hline
      \rownumber & \begin{tabular}[t]{@{}p{15em}@{}}
         jeu de données: Association des piétons et cyclistes pont Jacques-Cartier\\nombre d'images: 313\\résolutions: variées
      \end{tabular} & \begin{tabular}[t]{@{}p{35em}@{}}
         L'Association des piétons et cyclistes du pont Jacques-Cartier a une collection d'images et de vidéos de la piste multifonctionnelle du pont Jacques-Cartier. Ce n'est pas un jeu de données qui est prêt à être utilisé pour l'apprentissage tel quel, il doit être préparé. Mais c'est une source de données qui est très importante pour l'essai. Il est envisagé de contacter l'association au besoin afin de leur demander leur collaboration pour la collecte d'autres d'images ou vidéos.\\
         \url{https://www.flickr.com/photos/pontjacquescartier}\\
         \url{http://pontjacquescartier365.com/videos-pont-jacques-cartier}\\
      \end{tabular}\\
      \hline
      \rownumber & \begin{tabular}[t]{@{}p{15em}@{}}
         jeu de données: images et vidéo sur Internet\\nombre d'images: entre 30-50\\résolutions: variées
      \end{tabular} & \begin{tabular}[t]{@{}p{35em}@{}}
         Internet est une source de données non négligeable en termes de données. Quelques images et vidéos de la piste multifonctionnelle du pont Jacques-Cartier, autre que celles fournies par L'Association des piétons et cyclistes du pont Jacques-Cartier, sont disponibles. Ce n'est pas un jeu de données qui est prêt à être utilisé pour l'apprentissage tel quel, il doit être préparé. Mais c'est une source de données qui est très importante pour l'essai.\\
         \url{https://google.ca}\\
      \end{tabular}\\
      \hline
      \rownumber & \begin{tabular}[t]{@{}p{15em}@{}}
         jeux de données: KITI Road/Lane Detection
      \end{tabular} & \begin{tabular}[t]{@{}p{35em}@{}}
         Ce jeu de données contient 289 images d'entrainement et 290 images de tests d'image de routes urbaines. Il existe une grande multitude de modèles qui sont entrainés avec ce jeu de données.\\
         \url{http://www.cvlibs.net/datasets/kitti/eval_road.php}\\
      \end{tabular}\\
      \hline
   \end{longtable}
   \end{landscape}
   \clearpage
   \newpage
}
\subsection{Matériel et logiciels}
\myparagraph{Le nano ordinateur}
\begin{figure}[H]
    \centering
    \includegraphics[width=1.0\textwidth]{jetson_nano_lego}
    \caption[Carte mère Jetson Nano de NVIDIA]{Carte mère Jetson Nano de NVIDIA, représenté avec des Lego pour démontrer sa petite taille}
    \label{fig:jetson_nano_lego}
\end{figure}
\par L'objet d'étude de cet essai est le nano ordinateur "Jetson nano" du fabricant "NVIDIA" (figure \ref{fig:jetson_nano_lego}). Ce modèle a été choisi car il a été conçu par la compagnie NVIDIA spécifiquement pour répondre au besoin d'inférence en temps réelle sur le terrain, afin d'éviter le transfer de données et le traitement à distance et différé. 
\par L'architecture du nano ordinateur est ARM 64 bits (aarch64), ce qui le limite pour certaines portabilités de librairies, surtout dans le domaine assez restreind de la recherche, ou l'architecture la plus populaire et portable est x86-64. 
\par Il est composé d'un quad-core ARM Cortex-A57, qui est conçu pour ce genre de nano ordinateur, comme le Raspberry Pi.
\par Les performances GPU sont faibles, 0.5 TFLOPs (16FP; 16bits/2 bytes floating points). Par comparaison la PlayStation 4 Pro (2016) supporte +4 TFLOPs. 
\par La mémoire est limitée à 6GB. 
\par Les autres caractéristiques à considérer sont le port pour une carte microSD, un port Ethernet 10/100/1000Mbs, un port HDMI, un hub USB 4 ports 3.0, un connecteur pour une caméra, et un port PCIe.
\myparagraph{Logiciels}
\par De même que pour les périphériques, les solutions logiciels principales qui seront utilisés dans le cadre de l'essai sont résumés dans le tableau suivant, où il est indiqué leur nom, le type de licence, leur version, leurs rôles et responsabilités, comme pour le système d'exploitation, l'environnement de développement pour l'apprentissage profond, l'inférence, les logiciels de traitements vidéos et d'images. 
\par Pour tester les performances de la microSD et du disque SDD interne M.2 NVMe, l'utilitaire "hdparm" a été utilisé. Il est nécessaire de l'installer (`sudo apt-get install hdparm`) car il n'est pas inclus de base avec le système L4T.
\par Le SDK qui sera utilisé avec le nano ordinateur sera celui fourni par NVIDIA et qui se nomme "JetPack"\footnote{\url{https://developer.nvidia.com/embedded/jetpack}} \footnote{\url{https://docs.nvidia.com/jetson/jetpack/introduction/index.html}}. La version 4.4\footnote{\url{https://developer.nvidia.com/embedded/jetpack-archive}} sera celle avec laquelle les tests de performance ont été exeécuté. Il contient le système d'exploitation Linux For Tegra (L4T)\footnote{\url{https://developer.nvidia.com/embedded/linux-tegra}} (version L4T 32.4.3), qui est une version de la distribution Linux Ubuntu 18.04 mise à la saveur de NVIDIA. Jetpack contient aussi d'autres librairies qui sont nécessaires pour l'inférence, tel que Cuda, CuDNN et TensorRT.
\begin{figure}[H]
    \centering
    \includegraphics[width=1.0\textwidth]{jetpack_architecture}
    \caption[Diagramme de l'architecture du NVIDIA JetPack]{Diagramme de l'architecture du NVIDIA JetPack\protect\footnotemark}
    \label{fig:jetpack_architecture}
\end{figure}
\footnotetext{\url{https://docs.nvidia.com/jetson/l4t/index.html#page/Tegra\%2520Linux\%2520Driver\%2520Package\%2520Development\%2520Guide\%2Foverview.html\%23}}
\par Python et le C++ sont les languages utilisés par le framework de DeepLearning de NVIDIA. Python est utilisé comme language accessible et appelle les extensions écritent en C++ et qui optimisent les accès aux ressources systèmes tel que les CPUs et GPUS, les traitement des images et vidéos, les boucles et les traitements mémoires intensifs.
\par La librairie d’apprentissage profond qui sera utilisée est PyTorch, bonifié avec une version adaptée par NVIDIA de torchvision, qui fournie des modèles d'architecture et des utilitaires pour la vision par ordinateur (computer vision). Des versions bien spécifiques sont nécessaires et il est important de s'y conformer au risque de tomber dans une investigation bien coûteuse en temps et énergie\footnote{\url{https://forums.developer.nvidia.com/t/trying-to-regenerate-onnx-for-jetson-nano/125494?u=vincelf}}.
\par Le nano ordinateur inclut un GPU qui est mis à contribution lors de l'inférence. Le compilateur de NVIDIA pour GPU 'cuda' est nécessaire pour regénérer le .onnx lors de la phase d'adaption. La version doit concorder avec la bonne version de PyTorch. La version adaptée (fork) de torchvision doit être recompilée avec la bonne version de pytorch et cuda. 
\par Enfin pour régénérer le .onnx lors de la phase d'adaptation, les librairies TensorRT et ONNX ont été utilisées, en compagnie de l'utilitaire `trtexec` qui permet de valider et tester le fichier .onnx généré.
\par Lors de la phase d'évaluation des performances systèmes, les utilitaires tegrastats, free, iotop ont été utilisés.
{
    \vspace{0.3em} % Adjust the height of the space between caption and tabular
    \begin{longtable}[t]{{@{}|p{5em}|p{3em}|p{3em}|p{24em}|@{}}} % p{15em}p{35em} with landscape
        \caption{Solutions logicielles de l'essai}\label{table:table_sol_logiciel}\\
        \hline
        \textbf{Language} & \textbf{Version} & \textbf{Licence} & \textbf{Rôles et responsabilités} \\
        \endfirsthead
        \hline
        \textbf{Language} & \textbf{Version} & \textbf{Licence} & \textbf{Rôles et responsabilités} \\
        \hline
        \endhead
        \endfoot
        \endlastfoot
        \hline
        JetpPack & 4.4 & NVIDIA & Kit de développement de logiciels incluant le système d'exploitation L4T, et les librairies et utilitaires nécessaires pour l'inférence avec le nano ordinateur.\\
        \hline
        L4T & 32.4.3 & NVIDIA & Le sytème d'exploitation "Linux For Tegra" conçut par NVIDIA pour leurs solutions d'inférence légères, comme pour le nano ordinateur.\\
        \hline
        Python & 2.7 & GPL & Language plus accessible que le C++.\\
        \hline
        C++ & GCC 7.3.1\footnote{\url{https://developer.nvidia.com/embedded/linux-tegra}} & GPL & Certaines extensions du cadre applicatif de NVIDIA pour l'inférence sont écrites en C++, pour des raisons d'optimisation.\\
        \hline
        pytorch & 1.1.0 & BSD 3-Clause & Cadres d'application logicielle ("framework") pour l'apprentissage machine et profond.\\
        \hline
        torchvision & & BSD 3-Clause & Branche de torchvision adaptée par NVIDIA\footnote{\url{https://github.com/dusty-nv/vision.git} et ensuite branche v0.3.0}; Doit être recompilée avec la version de pytorch 1.1.0 et cuda 10.0.\\
        \hline
        cuda & 10.0 & NVIDIA & Compilateur de code C++ pour GPU.\\
        \hline
        TensorRT & 6.0.1.5 & NVIDIA & SDK pour générer des modèles au format ONNX, optimisés et interopérables, pour l'inférence.\\
        \hline
        ONNX & & MIT & Permet lde générer un format interopérable pour l'inférence de modèles d'architecture construit avec différent framework de machine learning (Caffe, PyTorch, TensorFlow, etc).\\
        \hline
        trtexec & & NVIDIA & Utilitaire qui a permis de tester le .onnx qui a été regénéré.\\
        \hline
        gstreamer & 1.1 & LGPL & Utilitaire qui a permis d'alimenter le modèle de la segmentation avec la vidéo.\\
        \hline
        v4l2loopback & & GPL & Utilitaire qui a permis de créer un matériel vidéo virtuelle permettant de remplaçer la caméra, permettant ainsi au modèle d'être alimenté par une vidéo et non la caméra.\\
        \hline
        hdparm & & GPL & Utilitaire permettant de tester la capacité de lecture d'une unité de stockage, tel que'un SSD NVMe et différentes cartes microSD.\\
        \hline
        tegrastats & & NVIDIA & La commande offre différents indicateurs système tel que l'utilisation des processeurs, la température, la consommation, et qui sont utiles pour observer le comportement du système lors des tests de performance de la segmentation.\\
        \hline
        free & & GPL & La commande offre le status de la mémoire totale, utilisée, libre, swap, cachée, etc. Elle est utile pour observer le comportement de la mémoire du nano ordinateur lors des tests de performance de la segmentation.\\
        \hline
        iotop & & GPL & La commande offre le status des opérations "I/O" de lecture \& écriture sur le disque, totale ou pour le processus de segmentation. Elle est utile pour observer le comportement des opérations sur le disque du nano ordinateur lors des tests de performance de la segmentation.\\
        \hline
    \end{longtable}
}
\clearpage
\newpage
\subsection{Méthodologie}
\par Voici à très haut niveau les grandes étapes de cet essai:
\label{methodologie_haut_niveau}
\begin{figure}[H]
    \centering
    \includegraphics[width=0.75\textwidth]{methodologie_haut_niveau}
    \caption{Organigramme de la méthodologie à haut niveau}
    \label{fig:methodologie_haut_niveau}
\end{figure}
\par Pour y parvenir, la méthodologie suivante a été suivie et permet d'évaluer les performances de base de la segmentation sémantique avec le nano ordinateur.
\label{methodologie_simple}
\begin{figure}[H]
    \centering
    \includegraphics[width=0.65\textwidth]{methodologie_simple}
    \caption{Organigramme de la méthodologie pour évaluer les performances}
    \label{fig:methodologie_simple}
\end{figure}
\par Si chacun des blocs est explosé, chacun d'eux s'organise autour des activités suivantes: 
\label{methodologie_simple_details}
\begin{figure}[H]
    \centering
    \includegraphics[width=1.0\textwidth]{methodologie_simple_details}
    \caption{Organigramme des détails de la méthodologie pour évaluer les performances}
    \label{fig:methodologie_simple_details}
\end{figure}
\par Si l'évaluation est probante, la méthodologie se verra bonifiée par des étapes d'adaptation et de traitement. 
\label{methodologie_complexe}
\begin{figure}[H]
    \centering
    \includegraphics[width=1.0\textwidth]{methodologie_complexe}
    \caption{Organigramme de la méthodologie pour évaluer les performances après une phase d'adaptation théorique}
    \label{fig:methodologie_complexe}
\end{figure}
\par Mais dans la réalité la méthodologie ressemblera plus à celle-ci:
\label{methodologie_complexe_realiste}
\begin{figure}[H]
    \centering
    \includegraphics[width=0.50\textwidth]{methodologie_complexe_realiste}
    \caption{Organigramme de la méthodologie pour évaluer les performances après une phase d'adaptation réaliste}
    \label{fig:methodologie_complexe_realiste}
\end{figure}
\par Si chacun des blocs est explosé, chacun d'eux s'organise autour des activités suivantes: 
\label{methodologie_complexe_details}
\begin{figure}[H]
    \centering
    \includegraphics[width=1.0\textwidth]{methodologie_complexe_details}
    \caption{Organigramme des détails de la méthodologie pour évaluer les performances après une phase d'adaptation}
    \label{fig:methodologie_complexe_details}
\end{figure}
\par Les phases de la méthodologie présentée dans l'organigramme de la figure \ref{fig:methodologie_complexe_details} peuvent être résumées de la façon suivante:
\begin{itemize}
   \item Recherche des références, des modèles et des données, ainsi que l'équipement pour le nano ordinateur et des logiciels nécessaires.
   \item Installation sur le nano ordinateur le système d'exploitation, l'environnement de développement et de tests pour l'inférence.
   \item Itération entre les étapes suivantes:
   \begin{itemize}
      \item Inférence avec le nano ordinateur en utilisant les modèles et les sources de données sélectionnées.
      \item Adaptation des modèles à différentes résolutions d'images et à la zone d'étude.
      \item Traitement des données afin de les adapter au requis des modèles.
   \end{itemize}
\end{itemize}
\par Finalement, il est à noter  que le cheminement de l'essai a été entièrement documenté pendant tout le déroulement de l'essai.
\subsection{Documentation}
\par La méthodologie a été entièrement documentée pendant tout le déroulement de l'essai. Elle se retrouve pour référence dans un blogue public sur le site de "github.io" \footnote{\url{https://vince7lf.github.io/}}. Cette méthodologie de documentation permet entre autres, de très facilement documenter, de ne pas perdre des notes très importantes, de suivre le cheminement, de pouvoir retrouver des notes, mêmes si elles ont été effacées ou modifiées, puisque toute modification est sauvegardée dans un repository Git.
\par Par ailleurs, tous les documents de rédaction LaTeX, les images, les scripts et code source qui ont été utiles et utilisés durant l'essai ont été géré dans un repository Git public avec "github.com" \footnote{\url{https://github.com/vince7lf/gae724}}. 
\par Ces sources d'information viennent bonifier grandement ce rapport et il est même recommandé de s'y référer pour atteindre un certain niveau détails et de compréhension. 
\subsection{Recherche}
% \begin{wrapfigure}{R}{0.25\textwidth}
%     \includegraphics[width=0.25\textwidth]{metho_recherche}
% \end{wrapfigure}
\subsubsection[Revue de littérature]{Revue de littérature\footnote{La revue de la littérature a débuté en octobre-novembre 2019, c'est-à-dire quelques mois après la disponibilité du nano ordinateur (juin 2019).}}
\par La revue de la littérature  a débuté en octobre-novembre 2019, c'est à dire quelques mois après la disponibilité du nano-ordinateur (Juin 2019). La recherche s'est concentrée sur des références traitant des concepts du sujet de l'essai : la segmentation sémantique, le temps réel, et les nano-ordinateurs. Le premier objectif a été de trouver si des études avaient déjà experimentés le nano-ordinateur, en particulier pour la segmentation de vidéos en temps réel. Pendant cette recherche, j'en ai profité pour effectuer une révision de l'évolution des réseaux de neurones convolutionnels entier (FCN Fully Convolutional Network)  et des différentes architectures, et chercher d'autres solutions de détection de la route en temps réel grâce au FCN. 
\par Il a été assez compliqué de trouver des références intégrant les nano-ordinateurs. Comme l'objectif de l'essai est de valider les performances d'un nano-ordinateur bien spécifique, les mots clés "NVIDIA Jetson nano" font partie de la stratégie de recherche. 
\par Les réseaux de neurones convolutifs entiers (FCN) sont implicitement inclus dans les résultats puisque c'est le "state-of-art" actuellement pour répondre au besoin de la segmentation sémantique d'images.
\par Plus de 75 références ont été collectées. Une quarantaine ont été sélectionnées. Cette sélection peut se décomposer en trois catégories : les références se rapprochant le plus du sujet de l'essai; l'histoire et les antécédents des réseaux de neurones; du matériel éducatif pour étudier et manipuler les réseaux de neurones.
\par Je me suis intéressé aux références des années les plus récentes, autour de 2020, 2019 et 2018, car les avancées dans le domaine des réseaux de neurones sont très rapides. Par curiosité je suis allé aussi parfois voir dans les années bien plus éloignées, comme 1998, ou j'ai trouvé un article proposant une solution pour prédire la température de la surface de la route avec des réseaux de neurones.
\par Je n'ai pas pu trouver de références spécifiquement pour la déduction de l'état de la surface (mouillé, gelée, etc.) d'une piste multifonctionnelle (vélo, piéton). 
\par Il est intéressant de noter que la banque de données SCOPUS retourne plus de 11,000 documents avec l'expression "segmentation AND "real-time"". Il y en a plus de 700 uniquement pour l'année 2019. 
\subsubsection{Étude du site d'implantation}
\par Le nano ordinateur est destiné à être déployé sur le chemin de la piste multi-fonctionnelle du pont Jacques-Cartier. L'étude du site a permis de chercher à comprendre, parmis ses caractéristiques, les difficultés de son usage l'hiver. Il sera tenter d'expliquer les défis et les raisons, techniques, politiques, sécuritaire, de pouvoir la conserver ouverte toute l'année. Une carte du site permettra de montrer un exemple de configuration où seront installés les nano-ordinateurs, et des images de ces  zones d'intérêt permettra de "visualiser" ce qui sera interprété par le modèle. 
\par Un mot sera réservé pour citer "L'Association des piétons et cyclistes du pont Jacques-Cartier" qui est un acteur actif pour le développement du transport actif dans cette région du Québec, et dont les membres sont des usagers habituels de la piste multifonctionnelle, même l'hiver.
\vspace{1\baselineskip}
\par 
\subsubsection{Sélection des données et des modèles de réseaux de neurones}
\par Les ressources mises à disposition par le constructeur du Jetson nano, NVIDIA, ont été étudiées pour apprendre et tester le nano ordinateur. Parmi les plus intéressantes, on peut citer le "Jetson Nano Developer Kit", le "NVIDIA Deep Learning Institute", la communauté Jetson, les tutoriels, les "benchmarks". Des jeux de données sont fournis gratuitement.
\par En complément des ressources de NVIDIA, deux références scientifiques ont été principalement utilisées comme points de départ et comme modèles pour l'essai, car leurs études ont été faites avec le Jetson nano (\cite{nguyen_mavnet_2019} et \cite{chong_real-time_1992}). Beaucoup de références ont été publiées ces deux dernières années sur le sujet de la segmentation sémantique, ils existent donc de multiples alternatives inspirantes.
\par Internet est une mine d'information. Il existe des forums et des blogues dans lesquels des utilisateurs publient leurs expérimentations de la segmentation sémantique en temps réel avec le Jetson nano (\cite{dustin_realtime_2019}), ou plus génériquement la segmentation sémantique. Des sites comme "modelzoo.co" et "kaggle.com" sont des entrepôts de modèles déjà entrainés. 
\par Une autre option est d'effectuer une recherche d'images ou de vidéos de la piste multifonctionnelle du pont Jacques-Cartier via les sites de recherche tels que Google. 
\par L'Association des piétons et cyclistes du pont Jacques-Cartier existe depuis de nombreuses années pour promouvoir le transport actif et conserver la piste multifonctionnelle du pont Jacques Cartier ouverte durant l'hiver. Ils fournissent, via leurs sites Internet, des collections de vidéos et d'images qui pourraient être utilisées. Il serait aussi possible d'entrer en contact avec l'association et leur demander de prendre de nouvelles vidéos. \cite{association_des_pietons_et_cyclistes_du_pont_jacques-cartier_pontjacques-cartier365com_2020} \cite{association_des_pietons_et_cyclistes_pont_jacques-cartier_flickr_2020}
\par Les architectures des modèles FCN sélectionnés pour l'essai sont résumés dans un tableau récapitulatif, incluant leur type, leur application et leurs jeux de données respectifs, précisant les différentes variantes entre résolutions et nombre d'images pas secondes (FPS).
\subsubsection{Choix de l'équipement pour le nano ordinateur}
\par L'objet d'étude de cet essai est un nano ordinateur. Un nano ordinateur est un ordinateur miniaturisé en taille, mais aussi limité en capacité. Il existe différents fabricants et modèles, de caractéristiques techniques variées, pour répondre à différents besoins. Le dernier né est le modèle "Jetson nano" du fabricant "NVIDIA", disponible depuis juin 2019 au prix très abordable de 99\$US. La compagnie NVIDIA a conçu ce matériel spécialement pour différentes applications d'inférence de modèles d'apprentissage profond sur une plateforme mobile (drone) ou proche des données ("edge" en anglais). Ce modèle a été choisi afin de répondre à l'intérêt que suscitent ses capacités et ses limites. Une image du Jetson nano et un tableau de ses caractéristiques techniques seront disponibles. 
\par L'architecture matérielle sera étudiée et présentée avec l'aide d'images, de diagrammes et de textes explicatifs. Les éléments clés seront identifiés.
\par Afin d'optimiser les performances du nano ordinateur, une recherche des périphériques les plus adaptés pour répondre aux besoins de performance (et de budget) de l'essai est essentielle, telle que l'alimentation, le stockage, la caméra. Des images des périphériques seront incluses, et les caractéristiques principales seront présentées dans des tableaux.
\par Le matériel est commandé par le collaborateur de cet essai "Vision météo".
\subsubsection{Exploration des solutions logicielles}
\par De même que pour les périphériques, les solutions logicielles logiciels nécessires sont résumés dans un tableau, où il sera indiqué leur nom, leur version, les avantages et limitations, comme le système d'exploitation, l'environnement de développement, l'inférence, les logiciels de traitements vidéos et d'images. 
\vspace{1\baselineskip}
\par 
\subsection{Environnement de travail}
\input{sections/methodologie/environnement_de_travail}
\subsubsection{Préparation du nano ordinateur}
\label{preparation_nano_ordinateur}
\par L'organigramme de la figure \ref{fig:preparation_nano_ordinateur} présente les activités qui composent la préparation du nano-ordinateur. 
\begin{figure}
    \centering
    \includegraphics[width=1.0\textwidth]{preparation_nano_ordinateur}
    \caption{Préparation du nano-ordinateur}
    \label{fig:preparation_nano_ordinateur}
\end{figure}
\myparagraph{Montage}
\par Le nano ordinateur est une carte mère livrée sans aucun périphérique ni même boîtier. Vu que les performances logicielles dépendent des performances matériels, surtout pour une unité tel qu 'un nano-ordinateur où les capacités matérielles sont très limités, la première partie de l'essai a été allouée à la sélection des accessoires et périphériques qui vont permettre d'augmenter les performances, protéger et utiliser confortablement le nano-ordinateur. 
\label{montage_nano_ordinateur}
\par L'organigramme de la figure \ref{fig:montage_nano_ordinateur} présente les activités qui composent le montage du nano-ordinateur. 
\begin{figure}
    \centering
    \includegraphics[width=1.0\textwidth]{montage_nano_ordinateur}
    \caption{Montage du nano-ordinateur}
    \label{fig:montage_nano_ordinateur}
\end{figure}
\mysubparagraph{Préparation de la carte mère Jetson Nano}
\begin{figure}
    \centering
    \includegraphics[width=1.0\textwidth]{jetson_nano}
    \caption{Carte mère du nano-ordinateur}
    \label{fig:jetson_nano}
\end{figure}
\par Le nano-ordinateur qui est livré dans sa boîte est uniquement une carte mère, sans unité de stockage, ni boîtier, clavier, souris, écran, capacité wifi, ou caméra. Il est uniquement livré avec un cable micro-usb qui lui permet d'être démarré avec une alimitation minimale de 5 Volt/2Amp et ne consommer que 5 Watt. Aucun système d'exploitation n'est livré non plus. Vu que de l'objectif de l'essai est de tester les capacités du nano-ordinateur et que la consommation sera de plus de 5Watt dues aux branchements de multiples périphérques, certaines "broches" sur la carte mère doivent être activées:  la broche J48 permet de brancher un adapteur d'alimentation de 5Volt 4Amp au lieu de l'alimentation micro-usb; et la broche J38 permet d'activer le PoE (Power-Over-Ethernet) afin d'hériter de l'alimentation du cable Ethernet. Aucune autre préparation sur la carte n'est nécessaire.
\mysubparagraph{Alimentation}
\begin{figure}
    \centering
    \includegraphics[width=1.0\textwidth]{alimentation}
    \caption{Adapteur 5 volt 4 amp}
    \label{fig:alimenation}
\end{figure}
\par L'alimentation du nano-ordinateur est l'élément matériel le plus important du système. De base le nano-ordinateur est livré avec un cable micro-USB, lui permettant d'être alimenté en 5Volt 2Amp. Mais le besoin en énergie augmente avec les périphériques qui s'accumulent, tel qu'une caméra. Il est prudent de choisir un adapteur 5Volt 4Amp d'un fournisseur recommandé par NVIDIA, car un changement de puissance sensible en entrée impacte le fonctionnement opérationnel du nano-ordinateur. Deux adapteurs ont été utilisés, l'un recommandé, et l'autre non, afin de tester leur performance. 
\par Dans le cadre de l'essai, l'alimentation du nano-ordinateur est utilisée pour alimenter la carte mère, qui comporte entre autre les CPUs, le GPU, le Hub USB 3.0 interne, le controleur Ethernet et le port HDMI. Mais aussi la caméra et  le ventilateur et optionellement une carte d'extension M.2 NVMe. Afin d'assister l'adapteur, un hub USB 3.0 externe a été utilisé pour brancher la souris, le clavier, et à un moment donné le dongle Wifi.
\mysubparagraph{Boîtier}
\begin{figure}
    \centering
    \includegraphics[width=1.0\textwidth]{boitier}
    \caption{Boitier pour le nano-ordinateur}
    \label{fig:boitier}
\end{figure}
\begin{figure}
    \centering
    \includegraphics[width=1.0\textwidth]{boitier_back}
    \caption{Vue arrière du boitier pour le nano-ordinateur}
    \label{fig:boitier_arriere}
\end{figure}
\par Afin de protéger le nano-ordinateur durant l'essai et l'utiliser dans les conditions les plus proches de son futur mode d'opération, il a été installé dans un boîtier en métal. Le boîtier a été choisi en tenant compte qu'une carte d'extension pour un SSD interne sera installée, ainsi qu'une caméra et un ventilateur. Durant l'essai le nano-ordinateur sera manipulé très fréquemment en raison d'un manque d'espace réservé dans la maison. Le boîtier permet donc d'éviter de manipuler le matériel et les connecteurs, les protège, évitant de risquer de les briser, et donc ajouter des délais à l'essai. 
\mysubparagraph{SSD interne Nvme M.2 avec carte d'extension M.2 vers USB3.0}
\begin{figure}
    \centering
    \includegraphics[width=1.0\textwidth]{Samsung 970 EVO Plus 250GB M.2 NVMe Internal Solid State Drive}
    \caption{Disque SSD NVMe M.2 interne 250GB}
    \label{fig:disquessd}
\end{figure}
\par Un disque SSD est entre 50 et 100 fois plus performant qu'une carte micro-SD. Il est aussi plus adapté pour manipuler les petits fichiers et héberger un système d'exploitation. Il est aussi plus résilient à long terme. C'est donc une option qui ne doit pas être négligée dans le contexte de tests de performance, encore plus avec un nano-ordinateur dont les capacités matériels sont limités. Néanmoins, il y a un contre-parti important dans la situation d'un nano-ordinateur: la consommation d'énergie. Un SSD interne va demander plus d'énergie qu'une carte micro-SD, et si le nano-ordinateur n'est pas capable de gérer correctement les besoins en énergie de ses extensions matériels, le SSD interne risque d'échouer en pleine opération et le nano-ordinateur devenir non fonctionnel soudainement.
\par Il y a deux choix qui ont été retenu pendant l'essai pour brancher un SSD interne au nano-ordinateur: soit via une carte d'extension M.2 MVMe, et connecté via le Hub USB, soir via une carte d'extension M.2 NVMe connecté au port PCIe interne du nano-ordinateur, normalement destinée à une carte d'extension Wifi.
\par Concernant le disque SSD M.2 NVMe connecté à la carte d'extension M.2 via le Hub USB 3.0 interne, le système L4T de NVidia (Ubuntu 18 mis à la saveur NVidia) ne supporte pas les SSD M.2 NVMe connecté au port USB. Il n'est pas reconnu / détecté, il est donc impossible de le formatter, de le partitionner, de l'utiliser. Comme il serait risqué pour l'essai de se lancer dans la recompilation du kernel du L4T, une alternative trouvée sur le développeur forum de NVidia est de passer par un adapteur M.2 MVMe connecté au port PCIe interne.
\par Malheureusement cette alternative a rapidement été abandonnée. Il a été possible de booter et installer le système d'opération sur le SSD M.2, et faire quelques tests, mais pour une raison inconnue, le système n'était pas stable et devenait non opérationnel assez rapidement, le système perdant la connexion au SSD. La durée la plus longue de stabilité observée a été de moins 30 minutes. Une hypothèse est une baisse d'énergie qui survient à un moment et qui impacte l'alimentation du SSD, chaque volt et milliampère étant important pour la stabilité du nano-ordinateur. De plus, le raccordement du cable de la carte d'extension M.2 NVMe PCIe avec le SSD M.2 NVMe est très compliquée et risqué pour le cable lui même. Une autre limitation importante est que cette solution ne permet pas d'utiliser le boîtier car le SSD M.2 ne rentre pas et ne peut même pas être fixé. 
\par Différentes options pour optimiser l'alimentation ont été explorées: utiliser un HUB USB externe et auto-alimenté; brancher un cable Ethernet au lieu d'utiliser un Dongle Wifi; allumer le ventilateur dés le démarrage du nano-ordinateur; et l'options de fournir 6Amp directement supportée par la carte mère via les pins; explorer les solutions sur les forums de discussion. 
{color{red}
Ref: https://www.kingston.com/en/community/articledetail/articleid/48543
https://geekworm.com/products/nvidia-jetson-nano-nvme-m-2-ssd-shield-t100-v1-1
\par À noter que la carte T100 est discontinuée et remplacée par T130
}
\mysubparagraph{Caméra}
\begin{figure}
    \centering
    \includegraphics[width=1.0\textwidth]{camera}
    \caption{Caméra}
    \label{fig:camera}
\end{figure}
\par L'objectif du nano-ordinateur est d'être utilisé pour détecter continuellement les délimitations de la piste cyclable. Il est évident qu'une caméra doit donc faire partie du système et faire partie de l'évaluation des performances. Néanmoins, durant le déroulement de l'essai, la caméra sera très peu utilisée. En effet il n'est pas évident d'être dans un mode de développement directement sur le terrain. Un matériel vidéo virtuel sera utilisé pour simuler la caméra et alimenter l'inférence avec des vidéos pré-enregistrées, permettant ainsi d'évaluer les performances de l'inférence avec des vidéos, même si d'un point de vue performance matérielle l'utilisation ne sera pas équivalente. Les performances matérielles de l'inférence en temps réel seront évaluée avec la caméra, même si la vue de la caméra n'est pas la piste cyclable, ce qui n'est pas important pour ce test, peu importe ce qui est détecté.
\mysubparagraph{Fan}
\begin{figure}
    \centering
    \includegraphics[width=1.0\textwidth]{fan}
    \caption{Ventilateur}
    \label{fig:fan}
\end{figure}
\par Un système informatique a besoin d'un ventilateur pour évacuer la chaleur produite par ses processeurs et les autres élément électroniques, et éviter une faute opérationnelle et des bris de matériel. L'objectif du nano-ordinateur étant d'être opérationnel continuellement, et ses éléments étant contenus dans un boîtier, il est encore plus indispensable d'installer un ventilateur. Le ventilateur choisi a pu être installé dans le boitier, même si le boîtier ne possède de support pour le fixer. Le ventilateur est capable de démarrer automatiquement au besoin, mais il est volontairement démarré manuellement dés que le nano-ordinateur est démarré. Cela évite que la chaleur ne s'accumule, qu'elle soit tout de suite ventilée à l'extérieure, évitant un risque de surchauffe, la capacité du ventilateur étant tout de même limité (petit modèle).
\mysubparagraph{Hub USB Externe 3.0 4 ports}
\begin{figure}
    \centering
    \includegraphics[width=1.0\textwidth]{Powered USB Hub 3.0, Atolla 7-Port USB Data Hub Splitter with One Smart Charging}
    \caption{Hub USB 3.0 externe autoalimenté}
    \label{fig:hubusb}
\end{figure}
\par Le nano-ordinateur comprends un hub USB 3.0 4 ports interne, les 4 ports étant connectées via le même controleur. Ce hub consomme de l'énergie pour alimenter les périphériques qui y sont connectés, comme un SSD interne ou un dongle Wifi, et gérer le échanges de données. Afin de minimiser les besoins en alimenation et optimiser le plus possible le transfer de données, la souris, le clavier et le dongle USB ont été branchées a un hub USB 3.0 externe autoalimenté. Malheureusement cette option complexifie le déploiement sur le terrain du nano ordinateur. L'alternative pour s'en passer est d'utiliser un cable Ethernet, PoE préférablement, à la place d'un dongle Wifi qui est très gourmant en terme de besoin en alimentation, et chauffe rapidement.
\mysubparagraph{Réseau Internet}
\par Le nano-ordinateur comprends un controleur Ethernet pour brancher un cable réseau et se brancher sur Internet. Selon la configuration de la carte mère, le nano-ordinateur peut hériter de l'alimentation via Ethernet (PoE), via la broche J38. Il comprends aussi aussi un port PCIe interne qui permet de brancher une carte d'extension Wifi. L'autre alternative étant de passer par un dongle USB Wifi, ou un périphérique Wifi externe connecté au port USB. 
\par Dans le cadre de cet essai, le périphérique Wifi externe USB a été utilisé en premier puisque déjà disponible. Malheureusement les performances étaient assez décevantes, le réseau Wifi à la maison n'étant pas non plus très performant dans la pièce ou le nano ordinateur était installé (table de la cuisine). Un débit d'environs 5Mbits était disponible. Par curiosité un dongle USB Wifi a été acquis, mais autant décevant. La meilleure alternative pour améliorer le déroulement de l'essai a été de tirer un cable Ethernet et d'installer un router secondaire, et de brancher le nano-ordinateur a ce nouveau router. L'accès internet a été plus stable et de bien meilleure qualité, la connexion étant d'environs 11Mbs. 
\par Le PoE n'a pas été évalué. 
\myparagraph{Préparation de l'unité de stockage}
\par Le nano-ordinateur est conçu pour fonctionner avec un système  d'exploitation hébergé sur une carte micro-SD. Il existe différentes cartes micro-SD, et certaines sont  beaucoup plus performantes que les autres. Malheureusement les cartes micro-SD ne sont pas destinées à exécuter un système d'exploitation à temps plein, et leur espérance de vie reste très limitée.  Étant donné que l'objectif du nano-ordinateur est d'être en opération continuelle à l'extérieure, l'utilisation un disque SSD interne comme alternative semble logique.
\mysubparagraph{Carte micro-sd}
\par Il existe différentes cartes micro-SD, de multiples constructeurs, et pour différents usages, mais généralement destiné pour stocker des images et vidéos directement par les appareils multimédias. Leur conception est faîte pour la manipulation de gros block de données, et non des petits fichiers. Trois cartes micro-SD 
seront évaluées: la carte micro-SD 64Gb EVO Plus (rouge; Samsung), 64Gb EVO Select (verte; Samsung), 32Gb Ultra (blanche; ScanDisk).
\mysubparagraph{Disque SSD}
\par Pour un appareil destiné a être continuellement en opération et à l'extérieure, l'unité de stockage doit être non seulement performante mais aussi endurante. Un disque SSD interne pour un nano ordinateur est soit une carte d'extension M.2 NVMe ou SATA (selon la carte d'extension), connecté au port PCIe ou USB. Les SSD internes 
Samsung 970 EVO 250GB NVMe M.2 et Samsung 860 EVO M.2 500GB SATA seront évalués. À noter qu'une carte micro-SD est tout de même nécessaire pour "bootstrapper" le système d'exploitation. Il n'est pas nécessaire d'avoir une carte micro-SD performante puisqu'elle n'est utilisée que pour démarrer le système qui se trouve sur le SSD interne. 
\myparagraph{Configuration du système d'exploitation}
\par La première fois que le système démarre, le système Ubuntu Linux For Tegra (L4T) doit être configurée avec toutes les options personnalisées (langue, clavier, timezone, etc).
\myparagraph{Installation \& compilation des librairies pour l'inférence}
\par Les librairies pour la segmentation sémantique 
d'images et de vidéos via l'inférence de modèles déjà préparées sont mises à disposition par NVIDIA via un projet dans GitHub. La documentation pour l'installation et l'inférence est disponible directement dans la page GitHub. 
\myparagraph{Installation d'un matériel vidéo virtuel 'v4l2loopback'}
\par L'inférence fournie par NVIDIA est conçue pour utiliser la caméra du nano-ordinateur. Ce qui n'est pas forcément "pratique" pour évaluer la segmentation sémantique d'une vidéo d'une piste cyclable. Heureusement un matériel vidéo virtuel permet de simuler la caméra et d'alimenter l'inférence avec une vidéo enregistrée, au lieu de la caméra. Le contre-partie concerne l'évalutation des performances:  en effet la caméra demande plus de puissance au nano-ordinateur que le simulateur logiciel.
\myparagraph{Tests}
\par Afin de s'assurer que le nano-ordinateur est prêt pour être évalué, des tests matériels et logiciels sont  effectuées une fois le système monté et stabilisé. Les résultats des tests servent de référence pour évaluer l'état de santé du nano-ordinateur. 

\subsubsection{Collecte des données}
\noindent Le jeu d'images de DeepScene est celui qui semble le plus approprié car il a été conçu pour détecter les chemins dans la forêt. De plus, il existe une version de l'architecture qui a été entrainée avec ce jeu. Comme un jeu d'images vérité terrain est disponible, cela procure un gain de temps non négligeable dans le cadre d'un essai. Le jeu d'images de CityScape est complet pour les scènes urbaines, mais comme il est moins spécialisé dans la détection de chemins ou de piste, son utilisation n'est pas priorisée. Il contient toutefois des images vérité terrain de routes, ce qui est avantageux dans notre contexte et le favorise par rapport au deux derniers que nous avons à notre disposition. En effet le jeu d'images et de vidéos de l'\acrshort{apcpontjc}, et celui que j'ai monté en prenant des vidéos de pistes cyclables de mon quartier, sont des jeux intéressants pour tester les résultats de la segmentation avec des images ou des vidéos qui viennent du site d'études, ou similaire, que celui de chemins forestiers. De plus ces images et vidéos sont loin des conditions parfaites (luminosité, qualité du sol, angle de vue, etc.).
\label{section:collecte_donnees}
\subsubsection{Mise en place des solutions logicielles}
\myparagraph{Jetson Nano}
\par Le nano ordinateur est destiné a l'inférence. NVIDIA fournit tout un système d'installation, qui est nommé JetPack, et qui contient un système d'exploitation basée sur Ubuntu, Linux For Tegra L4T), le cadre applicatif ('framework') et les librairies nécessaires pour l'inférence, tel que Python, pytorch, les modèles pré-entrainés au format ONNX, le compilateur CUDA, et le SDK TensorRT.
\myparagraph{Compute Canada}
\par Le nano ordinateur est destiné a l'inférence, et non l'entrainement de modèles. Il n'est pas non plus destiné a être un environnement de développement. Un autre environnement de travail est donc nécessaire pour développer, et doit posséder les capacités matérielles (GPUs, mémoires, espace de stockage) et logicielles (librairies) pour entrainer un modèle. Heureusement mon directeur de projet m'a introduit à Compute Canada, ou Calcul Québec. Compute Canada fournit un espace de travail puissant aux chercheurs et aux universitaires. Il n'est pas évident de posséder à la maison un environnement permettant de faire de l'apprentissage profond. Ce que je ne pouvais faire avec le nano ordinateur, j'ai pu le faire dans l'environnement de Compute Canada, tel que compiler un fork de torchvision, réentrainer des modèles, générer des onnx. Avoir accès a cet environnement de travail a été un élément déterminant dans le cadre de cet essai.
\mysubparagraph{Compte Compute Canada}
\par Compute Canada mets à disposition des ressources matérielles puissantes et l'accès a des libraires de haute technologie telle que pour l'apprentissage profond, permettant d'avoir un environnement de travail professionnel et performant rapidement. Les ressources matérielles à disposition sont des grappes de serveurs, de CPUs et GPUs de différents types, ainsi que de l'espace de stockage. Les librairies sont disponibles via un repository privé, et lorsque certaines étaient manquantes (onnx et onnxruntime), j'ai fait une demande par courriel. L'administrateur a pu rendre disponible l'une des deux (onnx), la seconde (onnxruntime) étant beaucoup plus complexe a installé, pour l'avoir tenté sur le nano ordinateur. 
\par L'autre avantage de l'environnement de Compute Canada est la mise à disposition de Jupyter Notebook, afin de tester rapidement du code Python. Par contre il n'est pas conseillé d'exécuter du code nécessitant des délais, tels que l'entrainement d'un modèle. 
\par L'un des irritants est de ne pas pouvoir exécuter un container docker tel quel. Il faut le convertir au format Singularity. Dans le cadre du projet cela m'aurait facilité la tâche, car NVIDIA fournit des docker prêt à l'utilisation pour le réentrainement. Je n'ai malheureusement pas pris le temps et la chance de convertir un container docker au format Singularity. Je ne sais pas si c'est une activité assez simple ou complexe, mais du peu que j'ai lu cela semble assez "rapide".
\mysubparagraph{Jupyter Notebook}
\par Le besoin de tester du code Python est toujours nécessaire. La console Python n'étant vraiment pas conviviale, un environnement Jupyter Notebook est un compromis incontournable. Heureusement Compute Canada fournit un accès à des notebooks depuis Internet, permettant en plus d'hériter de leur environnement de travail. Il est à noter que les notebooks n'ont pas été utilisés pour entrainer un modèle ou générer des onnx, mais de tester du code Python simple, comme visualiser des images, transformer des tensors, et évaluer la segmentation prédite générée avec le vérité terrain (\acrshort{gt}). 
\paragraph{NVIDIA}
\mysubparagraph{Compte NVIDIA}
\par NVIDIA mets à disposition tout un écosystème éducatif permettant aux développeurs et aux chercheurs d'obtenir de l'aide au sujet de leur produit et librairies. Dans le cadre de l'essai, un compte NVIDIA a été créé, permettant d'accéder au forum de développeurs, et les containers  docker par exemple. Il est aussi possible d'accéder à du matériel éducatif grâce à l'institut DeepLearning de NVIDIA, dont l'accès a été commandité par mon directeur de projet. Le forum de développeurs a été un outil très utile dans le cadre de ce projet, car le dépôt d'une question m'a permis de me débloquer. Je n'étais pas capable de re générer l'ONNX à partir du code source et de la documentation fournie par NVIDIA pour un modèle FCN. Le développeur principal de l'application a répondu et m'a guidé dans la résolution du problème. Les autres ressources ont eu un impacte limité dans le cadre de ce projet, puisque par exemple le container docker et DIGITS n'ont pas pu être utilisé. Le code source des modèles est disponible sans nécessiter de compte, de même que les SDKs Jetpack.
\mysubparagraph{NVIDIA DIGITS}
\par NVIDIA fournit aux développeurs un environnement visuel permettant de réentrainer les modèles FCN qu'ils fournissent avec leurs propres dataset. Cet environnement se nomme DIGITS. Malheureusement il est nécessaire d'avoir son propre matériel, le système d'exploitation Ubuntu 18.04 LTS, et très recommandé d'avoir au moins un GPU et un ordinateur performant. Ce qui n'est malheureusement pas mon cas. DIGITS ne s'installe pas sur le nano ordinateur, ni sous Windows, ni même un Ubuntu sous windows (WSL). Cette option a donc été abandonnée rapidement. 
\mysubparagraph{Docker NVIDIA}
\par NVIDIA fournit aux développeurs des containers docker, avec tout ce qui est nécessaire pour réentrainer un modèle et re générer un ONNX, par exemple. Malheureusement la capacité du nano ordinateur ne permet pas de travailler efficacement avec un container docker, le nano ordinateur devient sans réponse, nécessitant un redémarrage forcé ("hard-reboot"). Cette option a donc été aussi abandonnée rapidement. 
\mysubparagraph{NVIDIA DeepStream}
\par Durant le déroulement de l'essai, NVIDIA a mis à disposition un environnement d'apprentissage profond, nommé "DeepStream", facilitant la conception et la génération de modèles, jusqu'à l'inférence. Cet outil n'a pas été évalué, mais pourrait être un outil alternatif pour réentrainer un modèle.
\subsection{Évaluation}
L’évaluation des performances se décomposent des différents éléments suivants, et qui sont présentés dans le diagramme de la figure \ref{fig:metho_eval}: 
\begin{itemize}
    \item Tout d'abord, les indicateurs de performances.
    \begin{itemize}
        \item Les performances matérielles durant l'inférence seront évaluées grâce aux indicateurs fournies par les utilitaires 'Tegrastats' de NVIDIA, 'free' et 'iotop'. Ces utilitaires sont brièvement décrit dans le tableau \ref{table:table_sol_logiciel}.
        \item Les performances de la segmentation seront évaluées avec les indicateurs classiques\footnote{\url{https://ilmonteux.github.io/2019/05/10/segmentation-metrics.html}}: "IoU" ("Intersection Over Union", ou "Jaccard") et le "Z-Score" (ou "Dice").
    \end{itemize}
    \item Ensuite, différentes résolutions d'images et de vidéos seront utilisées pour déterminer lesquelles sont supportées par le modèle évalué. 
    \item Les images et vidéos qui sont à notre disposition pour être évaluée proviennent de différentes sources de données. Le jeu d'images de DeepScene est celui qui semble le plus approprié car il a été conçu pour détecter les chemins dans la forêt. De plus il existe une version du modèle qui a été entrainé avec ce jeu. Comme il possède un jeu d'images vérité terrain ("ground truth" GT), il sera bien utile pour évaluer la segmentation prédite et un gain de temps non négligeable dans le cadre d'un essai. Le jeu d'images de CityScape est très complet pour les scènes urbaines, mais comme il est moins spécialisé dans la détection de chemins ou de piste, son utilisation ne sera pas priorisé. Il contient toutefois des images vérité terrain de routes, ce qui est avantageux dans notre contexte et le favorise  par rapport au deux derniers que nous avons à notre disposition. En effet le jeu d'images et de vidéos de l'Association des piétons et cyclistes du pont Jacques-Cartier, et celui que j'ai monté, sont des jeux intéressants pour tester les résultats de la segmentation avec des images ou des vidéos qui viennent du site d'études, ou similaire, que celui de chemins forestiers. De plus ces images et vidéos sont loin des conditions parfaites (luminosité, qualité du sol, angle de vue, etc).
    \item Enfin plusieurs modèles ont été sélectionnés comme candidats intéressants pour l'évluation. Le premier modèle qui sera évalué sera celui de SegNet18 entrainé avec le jeu de données "DeepSCene", et fourni par NVIDIA. Le second de la liste, et qui est aussi déjà founi par NVIDIA, est celui de SegNet18 entrainé avec le jeu de donnée "CityScape". Les deux autres architectures, ResNet101 \& DeepScene et DeepLabV3 \& Deepscene, ne sont pas disponibles et devront être préparées et entrainées, mais elles sont attrayantes du point de vue de leur réputation et potentiel, et vouloir les adapter au contexte de l'essai semble logique. Une dernière boîte vide est disponible, afin de laisser une porte ouverte à une potentielle opportunité d'entrainer un modèle tout à fait personnalisé, comme par exemple une adapation de l'architecture de DeepLabV3 avec le jeu de données de l'Association des piétons et cyclistes du pont Jacques-Cartier.
\end{itemize} 
\label{metho_eval}
\begin{figure}[H]
    \centering
    \includegraphics[width=1.0\textwidth]{metho_eval_perf_round_glass_shadow2}
    \caption{Éléments pour l'évaluation des performances}
    \label{fig:metho_eval}
\end{figure}
\subsubsection{Stratégie de test de l'inférence}
\par L'objectif principal de l'essai est de déterminer la capacité et les limites du nano ordinateur d'inférer en temps réel des modèles de réseau de neurones à convolution entier pour la segmentation sémantique de vidéos. La stratégie qui sera appliquée sera de tester avec divers modèles et divers niveaux de qualité vidéos, en espérant trouver le compromis qui répond le mieux à cet objectif.
\begin{enumerate}
   \item \label{metho:testbaseinférence} Afin de s'assurer du bon fonctionnement du nano ordinateur et d'avoir des résultats de référence propre à notre environnement, l'inférence sera testée avec des modèles existants et pré entrainés pour la segmentation sémantique, avec les images et les vidéos provenant des références, et dont les caractéristiques et les résultats sont disponibles. 
   \item \label{metho:testbaseinférencesite} En espérant que les tests de l'étape \#\ref{metho:testbaseinférence} précédente donnent les résultats documentés dans les articles de références, ils seront repris avec les mêmes modèles, mais avec les images et les vidéos du site d'étude possédant la meilleure qualité acquise (1080p/i, 30FPS). Les données sources (images et vidéos) devront subir certains prétraitements à ce effet, afin de répondre aux requis des modèles.
   \item \label{metho:testdevinférencesite} Selon les résultats de l'étape \#\ref{metho:testbaseinférencesite}, les tests se concentreront sur l'inférence avec des vidéos, en réduisant progressivement la résolution (760p/i, 576p/i, 480p/i, 360p/i) et le nombre d'images par seconde (20FPS, 10FSP, 1FPS).
   \item Les étapes intermédiaires de l'étape \#\ref{metho:testdevinférencesite} précédente seront de 1) valider les résultats de l'inférence avec des images avant de tester avec les vidéos, et 2) évaluer si les modèles de réseaux de neurones à convolution entiers doivent et/ou peuvent être adaptés facilement, en tenant compte de l'échéancier de l'essai, et ce afin de répondre à l'objectif principal.
\end{enumerate}
\subsubsection{Stratégie de collecte des indicateurs de performance matériel}
\par La méthodologie de la collecte des indicateurs est la suivante\footnote{\url{https://vince7lf.github.io/2020/05/26/metrics.html}}: 
\begin{itemize}    
    \item La collecte est démarrée après un démarrage frais, manuellement, via un script shell, qui exécute chaque utilitaire, et attend l'interruption du test. 
    \item Chaque utilitaire qui est utilisé pour collecter les mesures, possède son propre fichier.
    \item La date et l'heure de chaque indicateur collecté sont précisées.
    \item Afin de faciliter la documentation et l'analyse du test, des points d'intérêt sont ajoutés dans un fichier séparé pour marquer un moment particulier du test, avec la date, l'heure et un libellé. Ce point d'intérêt est fait grâce à une commande "shell" qui vient ajouter une trace dans ce fichier.
    \item Chaque indicateur est collecté toutes les secondes.  
    \item Une fois le test complété, la collecte est arrêté manuellement. 
    \item Chaque fichier est ensuite transformé en fichier CSV, via des commandes shell.
    \item À partir des fichiers CSV un script Python génère les graphiques automatiquement. 
\end{itemize}
\par Chaque indicateur est une colonne du fichier CSV. Il existe le même nombre d'indicateurs à tout moment. La date et l'heure sont un champ. 
\par Avant tout début de tests, la collecte est démarrée sans activité autre que la collecte des indicateurs. Cela permet de prendre une base de référence sans aucune charge.
\par Ensuite les tests débutent. 
\par Les indicateurs collectés permettent de créer des graphiques qui montrent la progression de chacun.
\par Les performances matérielles du Jetson Nano sont évaluées grâce à différents utilitaires : "tegrastats" fournis par NVIDIA, "free" et "iotop".
\par Les performances de la segmentation sont évaluées grâce au IoU et au z-score pour la classe du chemin / route. Une fonction Python est utilisée. Les fonctions IoU et le z-score utilisent l'image prédite (généré par le modèle FCN) et l'image vérité terrain ("ground truth"). Les images originales sont donc présélectionnées selon leur intérêt et l'image vérité terrain ("ground truth") créée. L'image prédite et vérité terrain ("ground truth") doivent utiliser la même palette de couleurs et doivent être de la même résolution. Pour les images qui ne possèdent pas d'image vérité terrain ("ground truth"), cell-ci est créée à la main avec l'éditeur "Gimp". Comme la résolution de la segmentation de l'image prédite par le modèle de NVIDIA est très faible ("carrée"), l'image vérité terrain ("ground truth") ne sera pas précise au pixel prêt. Le besoin est d'évaluer et non d'entrainer, l'importance de la précision de la classification est moindre dans ce cas. 
\subsubsection{Segmentation avec des images}
\myparagraph{Préparation et post-traitement}
\par Afin de pouvoir mesurer les performances de la segmentation (IoU, z-score), les classes et la palette de couleur entre l'image vérité terrain ("ground truth" GT) et celles préditent doivent être les mêmes.
\par L'image vérité terrain ("ground truth (GT)") du jeu de donnée original DeepScene ne possède pas la même palette de couleur ni exactement les mêmes classes que celle du modèle.
\par Un travail d'uniformisation est nécessaire avant la segmentation, qui est résumé dans le tableau \ref{table:classes_palette_couleur}.
{
    \renewcommand*{\arraystretch}{1.4}
    \begin{table}[h]
    \centering
    \caption{Classes et palettes de couleur}\label{table:classes_palette_couleur}
    \vspace{0.3em} % Adjust the height of the space between caption and tabular
    \begin{tabular}{{@{}|p{4em}|p{6em}||p{4em}|p{6em}||p{4em}|p{6em}|@{}}}
        \hline
        \multicolumn{2}{|c||}{\textbf{DeepScene}} & \multicolumn{2}{c||}{\textbf{NVIDIA}} & \multicolumn{2}{c|}{\textbf{Consolidée}} \\
        \hline
        \multicolumn{1}{|l|}{\textbf{Classes}} & \multicolumn{1}{c||}{\textbf{RGB}} & \multicolumn{1}{l|}{\textbf{Classes}} & \multicolumn{1}{c||}{\textbf{RGB}} & \multicolumn{1}{l|}{\textbf{Classes}} & \multicolumn{1}{c|}{\textbf{RGB}} \\
        % \thead{Classes \\ DeepScene} & \thead{RGB \\ DeepScene} & \thead{Classes \\ NVIDIA} & \thead{RGB \\ NVIDIA} & \thead{Classes \\ consolidées} & \thead{RGB \\ consolidées} \\
        \hline
        \hline
        Road & 170-170-170 & Trail & 200-155-75 & Trail & \textcolor{red}{170-170-170}\\
        \hline
        Grass & 0-255-0 & Grass & 85-210-100 & Grass & \textcolor{red}{0-255-0}\\
        \hline
        Vegetation & 102-102-51 & Vegetation & 15-100-20 & Vegetation & \textcolor{red}{102-102-51}\\
        \hline
        Tree & 0-60-0 & - & - & \textcolor{red}{Vegetation} & \textcolor{red}{102-102-51}\\
        \hline
        Sky & 0-120-255 & Sky & 0-120-255 & Sky & \textcolor{red}{0-120-255}\\
        \hline
        Obstacle & 0-0-0 & Obstacle & 255-185-0 & Obstacle & \textcolor{red}{0-0-0}\\
        \hline
    \end{tabular}
    \end{table}
\par De plus, l'image segmentée prédite par le modèle ne possède pas précisément la même palette de couleur que celle qui est configurée, il y a quelques différences minimes dans les codes couleurs RGB (par exemple 0-119-255 au lieu de 0-120-255), mais qui doivent être arrangées afin de pouvoir être correctement évaluées. 
\par Un travail de traitement de l'image segmentée prédite est nécessaire avant l'évalution de la segmentation.
\myparagraph{Segmentation et évaluation}
\par Afin de tester la performance de la segmentation du modèle, deux images du jeu de données de DeepScene seront utilisées car ce jeu contient déjà les images vérités terrain ("ground truth" GT), un gain de temps non négligeable dans le cadre de l'essai. Uniquemement la classe "Trail" sera évaluée.
\par L'architecture fournit à l'utilitaire "segnet-console" est "fcn-resnet18-deepscene-576x320" \footnote{\url{segnet-console -{}-network=fcn-resnet18-deepscene -{}-visualize=mask -{}-alpha=10000 images/city_0.jpg output.jp}}. 
\par Un script Python\footnote{\url{https://gist.github.com/ilmonteux/8340df952722f3a1030a7d937e701b5a}} est utilisé afin de mesurer le\acrshort{iou}et le Z-Score de la classe de l'image prédite par le modèle.
\subsubsection{Segmentation avec des vidéos}
\myparagraph{Préparation et pré-traitement}
\par L'évaluation de la segmentation avec des vidéos va s'effectuer non pas avec la caméra, mais avec un matériel vidéo virtuel. En effet, il n'est pas réaliste de pouvoir travailler sur le terrain. La commande "segnet-camera" permet de fournir en option le matériel qui doit être utilisé, par exemple "/dev/video0" pour la caméra. Le module "v4l2loopback"\footnote{\url{https://github.com/umlaeute/v4l2loopback}} permet de créer un matériel vidéo virtuel "/dev/video1". Ce matériel permet de recevoir un flux vidéo, qui pourra alors alimenter l'utilitaire "segnet-camera", comme le ferait la caméra. Le flux vidéo sera produit par l'utilitaire "gstreamer" avec comme données d'entrées le fichier de la vidéo et dirigé vers le matériel vidéo virtuel "/dev/video1".
\par La difficulté réside dans le fait que le matériel vidéo virtuel et le flux vidéo doivent être compatibles avec ce que l'utilitaire "segnet-camera" s'attend, et qui a été conçu pour être compatible avec une caméra. 
\par Le diagramme de la figure \ref{fig:arch_segmentation_video} résume à haut niveau les relations entre ces éléments. Pour comparaison, le diagramme de la figure \ref{fig:arch_segmentation_camera} montre la segmentation avec la caméra. 
\begin{figure}[H]
    \centering
    \includegraphics[width=.65\textwidth]{arch_segmentation_video}
    \caption[Diagramme d'architecture de la segmentation d'une vidéo]{Diagramme d'architecture de la segmentation d'une vidéo}
    \label{fig:arch_segmentation_video}
\end{figure}
\begin{figure}[H]
    \centering
    \includegraphics[width=.5\textwidth]{arch_segmentation_camera}
    \caption[Diagramme d'architecture de la segmentation avec la caméra]{Diagramme d'architecture de la segmentation avec la caméra}
    \label{fig:arch_segmentation_camera}
\end{figure}
\myparagraph{Segmentation et évaluation}
\par Les tests de performance de la segmentation de vidéos se déroulent de la manière précisée dans la section "\ref{section:strategie_test_inference} \nameref{section:strategie_test_inference}". 
\par L'un des avantages de l'utilitaire "gstreamer" est de pouvoir contrôler la résolution et le nombre d'images par seconde (\acrshort{fps}) de la vidéo qui doit être segmentée. Les différentes résolutions et \acrshort{fps} qui désirent être exécutées sont préparées dans un script "shell" écrit pour l'occasion. Le script s'occupe de démarrer gstreamer avec les bons paramètres, et en parallèle de démarrer la segmentation avec "segnet-camera". Un jeu de résolution peut être testé unitairement\footnote{\url{https://github.com/vince7lf/gae724/blob/master/run_deepscene.sh}}, ou plusieurs en séquence\footnote{\url{https://github.com/vince7lf/gae724/blob/master/run_deepscene_batch.sh}}. 
\par Les résolutions et images par seconde qui ont été testées sont résumées dans le tableau \ref{table:resolutions_tested}. 
\par Deux vidéos ont été utilisées pour tester la segmentation. La première vidéo est utilisée pour tester l'inférence avec une vidéo du site d'étude, et qui a été fournie gracieusement par l'\acrshort{apcpontjc}. Cette première vidéo est intéressante, car elle est filmée en mouvement par un cycliste. Dans un interval de 30 secondes, l'angle de vue change rapidement. La piste cyclable est bordée d'un muret côté sud, et de la route avec les voitures qui circulent côté nord. Même si la journée est ensoleillée, la surface de la piste est aussi à un moment humide.
\par La seconde vidéo est utilisée pour tester la segmentation avec les différentes résolutions et images par seconde. C'est une vidéo d'une petite piste cyclable qui est dans mon quartier, et que j'ai prise en marchant avec mon téléphone intelligent. La vidéo est intéressante, car dans un interval de 30 secondes l'état de la piste passe d'une scène ensoleillée à ombragée, sèche à mi-sèche, avec un petit ou gros banc de neige en bordure, ou qui s'aventure un peu sur la piste, bordée d'herbe mouillée ou sèche.
\subsection{Adaptation}
{\color{red}
\todo{TODO}
\par 
\label{metho_adaptation}
\begin{figure}[H]
    \centering
    \includegraphics[width=0.75\textwidth]{metho_traitement_eval.png}
    \caption{Méthodologie du traitement et adaptation}
    \label{fig:metho_adaptation}
\end{figure}

% keep this, it creates space between section and avoid polluting the main document. 
\vspace{1\baselineskip}
}
\subsubsection{Choix du modèle de l'architecture FCN}
\par Le premier modèle qui sera évalué sera celui de SegNet18 entrainé avec le jeu de données "DeepSCene", et fourni par NVIDIA. Le second de la liste, et qui est aussi déjà founi par NVIDIA, est celui de SegNet18 entrainé avec le jeu de donnée "CityScape". Les deux autres architectures, ResNet101 \& DeepScene et DeepLabV3 \& Deepscene, ne sont pas disponibles et devront être préparées et entrainées, mais elles sont attrayantes du point de vue de leur réputation et potentiel, et vouloir les adapter au contexte de l'essai semble logique. Une dernière boîte vide est disponible, afin de laisser une porte ouverte à une potentielle opportunité d'entrainer un modèle tout à fait personnalisé, comme par exemple une adapation de l'architecture de DeepLabV3 avec le jeu de données de l'Association des piétons et cyclistes du pont Jacques-Cartier.
\label{section:choix_modele_architecture}
% \subsubsection{Adaptation du modèle}
% \par 
\vspace{1\baselineskip}
\par 
% \subsubsection{Ré-entrainement du modèle}
% \input{sections/methodologie/reentrainement_modele}
%----------------------------------------------------------------------------------------
\section{Résultats}
{
   \color{red}
   \todo{TODO}
   \par Voici le plan qui est utilisé pour rédiger les résultats.
   \begin{itemize}
      \item \label{resultat1}Pour chaque modèle et résolution utilisés, la segmentation sémantique de certaines images et vidéos sera présentée. La segmentation qui a réussi, celle qui est moins précise, et celle qui a échoué seront soulignées. Un résumé du \% de succès vs des échecs sera fait, selon les modèles et les résolutions. 
      \item En complément de la section précédente, les performances du Jetson nano pour les divers scénarios de test seront résumés avec différents indicateurs. Ceux qui ont échoué ou n'ont pas été possibles en raison des limitations du nano ordinateur seront indiqués. 
      \item Enfin les performances de l'inférence et des modèles de réseaux de neurones pour la segmentation sémantique seront listées. Des indicateurs de performance classiques et tirés de la littérature seront utilisés.
   \end{itemize}
}
\par Pour tester les performances de la micro-sd et du disque SDD interne M.2 NVMe, l'utilitaire hdparm peut être facilement utilisé. Les tests montrent que le SSD interne est plus de 8 fois plus efficace que la micro-sd pour la lecture de données. 
{
   \centering
   \vspace{0.3em} % Adjust the height of the space between caption and tabular
   \renewcommand*{\arraystretch}{1.4}
   \begin{longtable}[t]{@{}|p{5em}|p{2em}|p{2em}|p{3em}|@{}} % p{15em}p{35em} with landscape
      \caption{Comparaison des performances du "data read" entre un SDD M.2 NVMe et une micro-sd}\label{tab:Timing O_DIRECT disk reads}\\
      \hline
      \textbf{Disk reads} & \textbf{MB} & \textbf{sec} & \textbf{MB/sec}\\
      \hline
         SSD & 1004 & 3 & 334.15\\
      \hline
      micro-sd & 122 & 3.03 & 40.22\\
      \hline
   \end{longtable}
}
%----------------------------------------------------------------------------------------
\section{Interprétation et discussion des résultats}
\subsection{Performances matérielles}
\subsubsection{Stockage de données}
\noindent Les tests montrent que le \acrshort{ssd} interne est de 4 à 11 fois plus efficaces qu'une carte microSD, pour l'opération de lecture de données. 
\subsubsection{Performances système}
\myparagraph{Performances globales}
\vspace{\baselineskip}
\\
\noindent Concernant les performances globales du nano ordinateur, il est à noter que celui-ci est capable d'exécuter l'inférence en temps réel pour une durée prolongée (23 minutes dans ce cas), et rester réactif aux commandes. L'exemple qui le démontre est le démarrage du navigateur Chromium entre deux segmentations, et pendant la segmentation.
\myparagraph{Fréquence}
\vspace{\baselineskip}
\\
\noindent La commande "tegrastats" offre la fréquence des \acrshort{cpu}s (4 pour le nano ordinateur), le GR3D (\acrshort{gpu}) et EMC. On peut noter que l'inférence prend 100\% du GR3D pendant toute la durée. Les \acrshort{cpu}s sont tous utilisés équitablement pendant l'inférence, en dépassant rarement les 30\% d'utilisation. En fait la période qui montre une exploitation élevée des \acrshort{cpu}s est lors de l'utilisation de Chromium, ou l'ensemble des \acrshort{cpu}s sont employés entre 0\% et 90\%. 
\vspace{\baselineskip}
\\
\noindent Il faut donc rester vigilant quant à l'utilisation des \acrshort{cpu}s pendant l'inférence sur le long terme, au risque de perdre le système en raison d'un ralentissement progressif dû à un manque de ressources processeurs \acrshort{cpu}s.
\myparagraph{Mémoire}
\vspace{\baselineskip}
\\
\noindent La commande "free -m" offre l'utilisation mémoire du système en Mb. Le nano ordinateur au démarrage ne consomme qu'environ 1.5Gb de mémoire totale, et possède 4Gb de libres sur un total de 6 Gb(d'où proviennent les 6Gb du graphique ?\todo{TODO}). À la fin du test de 25 minutes, il ne reste qu'environs 3Gb de mémoire libre, un peu plus de 2Gb semble resté utilisé. De la mémoire swap a commencée à être consommée lors du démarrage de Chromium pendant la 3e segmentation, et ne semble jamais avoir été libérée. La mémoire tampon cachée est aussi sensiblement utilisée et revient un peu en dessous de son niveau original à la fin du test. 
\vspace{\baselineskip}
\\
\noindent De même que pour l'utilisation des processeurs, il semble être préférable de rester vigilant lors de l'utilisation opérationnelle du nano ordinateur, la segmentation consommant de la mémoire qui semble ne plus être disponible pour les autres ressources du système, comme le démontre l'état de la mémoire totale libre à la suite de l'arrêt de la 1re segmentation. 
\myparagraph{I/O}
\vspace{\baselineskip}
\\
\noindent La commande "iotop" offre les performances I/O du nano ordinateur pendant le test de 25 minutes. Le I/O de la segmentation est très raisonnable, de même que celle du système. Il n'y a quasiment pas d'opération visible en écriture, même la collecte des statistiques durant le test, aux secondes, n'apparait pas. Les opérations en lecture sont plus visibles, mais très ponctuelles. La période la plus occupée en lecture semble être due durant le démarrage de la segmentation la première fois: le système semble lire le modèle en mémoire, et le conserver en mémoire, car les opérations en lecture suivantes sont peu ou non visibles pendant le démarrage des segmentations suivantes.
\vspace{\baselineskip}
\\
\noindent Cela expliquerait l'augmentation de l'utilisation de la mémoire à la suite de la segmentation. 
\myparagraph{Température}
\vspace{\baselineskip}
\\
\noindent La commande "tegrastats" offre grâce à des capteurs intégrés à la carte mère la température de différents éléments matériels du nano ordinateur. La commande "sudo jetson\_clock" est démarrée manuellement dès que le système est démarré, permettant de profiter de la fréquence maximale d'utilisation supportée par le nano ordinateur. Le  succès de la commande est simple à vérifier: le ventilateur se met à ventiler aussitôt\footnote{\url{https://docs.nvidia.com/jetson/l4t/index.html#page/Tegra\%20Linux\%20Driver\%20Package\%20Development\%20Guide/power_management_nano.html}}.
\vspace{\baselineskip}
\\
\noindent La température dans la pièce au moment du test est de 27C. Au démarrage, on note que la température mesurée de la plupart des capteurs thermiques, sauf pour le AO ("Always on") est entre 33C et 36C. Le démarrage de la 1re segmentation fait graduellement monter la température, entre 37C et 39C, jusqu'au point d'arrêt de la segmentation, après 200 secondes approximativement, et qui diminue graduellement approximativement pendant 200 secondes vers son point d'origine lorsqu'elle est arrêtée. Le démarrage de Chromium pendant cette période semble ralentir un peu le refroidissement. L'observation lors de la seconde segmentation est identique à la première. La troisième segmentation est plus longue, 400 secondes, et voit la température se stabiliser entre 41C et 43C. L'arrêt de la segmentation voit la température baisser et revenir assez rapidement à sa température d'origine. 
\vspace{\baselineskip}
\\
\noindent Le capteur thermique AO ("Always on" ) est plus particulier, puisqu'il mesure une température de 10C supérieures aux autres capteurs. Selon le modérateur Trumany de NVIDIA\footnote{\url{https://forums.developer.nvidia.com/t/operating-temperature-range-on-jetson-nano/73555/10}}, "AO\_therm is used for a truly robust thermtrip and as an LP0 wake source, as other zones will cease to operate during LP0.". Mes compétences en la matière ne me permettent pas d'expliquer clairement ce renseignement, mais cela semble signifier que ce capteur est plus robuste que les autres et devient l'indicateur de référence pour gérer une surchauffe. 
\vspace{\baselineskip}
\\
\noindent Il est donc à noter que l'opérationnalisation constante de la segmentation aurait un impacte non négligeable sur la durée de vie du Jetson Nano. Selon la documentation de NVIDIA, une carte Jetson Xavier TX2i qui opère 24/7, selon certaines conditions, a une durée de vie théorique de 4,4 années\footnote{\url{https://docs.nvidia.com/jetson/l4t/index.html#page/Tegra\%2520Linux\%2520Driver\%2520Package\%2520Development\%2520Guide\%2Fjetson_module_support.html}}
\vspace{\baselineskip}
\\
\noindent Au besoin, plus d'informations peuvent être trouvées dans le guide de conception thermique du Jetson Nano \footnote{\url{https://developer.download.nvidia.com/assets/embedded/secure/jetson/Nano/docs/Jetson_Nano_Thermal_Design_Guide_TDG-09383-001_v1.3.pdf?2P65awpyl3RwXu6jWjsqFgresjNSqhO-N2uI3BPNH2Wcbp9LNh91GF3UtmC3JgEWd6MX2-BC5xoL80tY5Wpl5cEltIMR4IawEflJehkxKH3yDAgxV-HpXyOo5Ge8a32mdntMcfRzjRZZTP2-hsJlIuT5FB7G36zHkCva7uPS9ntgWDff-w1W0LBJLH5DvpE1qU-3yZM5hjSz9g9cpFM}}
\myparagraph{Consommation}
\vspace{\baselineskip}
\\
\noindent La commande "tegrastats" offre de visualiser la consommation du nano ordinateur, soit globale, pour les \acrshort{cpu}s et pour le \acrshort{gpu}. En mode opérationnel continue, cela peut avoir une importance sur le budget, car la consommation est clairement beaucoup plus élevée pendant la segmentation. Il peut être observé aussi qu'elle est beaucoup plus volatile avec Chromium démarrée. 
\subsection{Performances de la segmentation}
\subsubsection{Images}
La segmentation prédite pour la classe "Trail" est assez surprenante. Le \acrshort{iou} est de 89\% et de 69\% respectivement dans le cas des deux images évaluées, ce qui est très encourageant. Par contre les délimitations de la segmentation pour le chemin sont décevantes et questionnables, car le modèle retourne une résolution très faible, l'image est très grossièrement "pixelisée", de gros carrés sont utilisés pour délimiter chaque classe. C'est probablement dû au fait que l'architecture du modèle segnet18 n'utilise que 18 couches, et qu'il n'y a donc que peu de représentations possibles pour les classes. 
\subsubsection{Vidéos}
La segmentation des vidéos n'a pas pu être évaluée avec des indicateurs de performances. C'est donc subjectivement que l'on peut donner une appréciation. Comme les images se succèdent très rapidement, ce n'est pas non plus évident. Grossièrement je n'ai pas trouvé le résultat de la segmentation très bonne. Les vidéos proviennent d'un contexte autre que celui du jeu de données d'entrainement (la forêt), la surface de la piste est bien différente. 
\vspace{\baselineskip}
\\
\noindent Je ne sais pas trop comment la segmentation en temps réel de vidéos peut être utilisée, car il n'y a pas moyen d'évaluer la qualité de la segmentation sans avoir la vérité terrain (\acrshort{gt}).
%----------------------------------------------------------------------------------------
\section{Conclusion et recommandations}
\subsection{Objectif principal}
\noindent L'objectif principal de l'essai était d'évaluer la capacité du nano ordinateur "NVIDIA Jetson Nano" à exécuter, en temps réel, une architecture de réseau de neurones pleinement connectés (\acrshort{fcnn}) permettant la segmentation sémantique d'une vidéo d'une piste multifonctionnelle. Il faut découper en plusieurs faits cet objectif afin de bien pouvoir l'évaluer: 
\begin{itemize}
   \item Le nano ordinateur est capable de segmenter sémantiquement une vidéo représentant une piste cyclable grâce à une architecture \acrshort{fcn}. 
   \item La segmentation sémantique découlant de l'inférence n'a pas pu être mesurée, il n'y a aucun moyen qui m'est connu afin de récupérer un indicateur, un coefficient ou un score me permettant de juger si la segmentation d'une vidéo est bonne ou non, comme pour une image ou la mesure du \acrshort{iou} ou du F1 score est possible si l'image de la vérité terrain (\acrshort{gt}) est disponible. 
   \item L'image générée par l'architecture FCN SegNet18 a une résolution faible, de l'ordre de 19 x 10 pixels. La délimitation de la segmentation, entre chaque classe, est donc grossière.
   \item Le "temps réel" à été simulé, et n'est donc pas celui qui sera utilisé sur le terrain. 
   \item le nano ordinateur et l'architecture FCN supporte l'inférence d'une vidéo HD (résolution de 720x1280 = 720p) avec un nombre d'images par seconde de 60/1 \acrshort{fps}.
\end{itemize}
\vspace{\baselineskip}
\noindent D'un point de vue performance matérielle et logicielle, le nano ordinateur est capable de segmenter une vidéo avec une architecture \acrshort{fcn}. Par contre, d'un point de vue qualitatif, 1) la qualité de la segmentation ne peut pas être mesurée. De plus, 2) la segmentation prédite est imprécise.
\vspace{\baselineskip}
\\
\noindent La première contrainte qualitative semble être un défaut majeur. Mais si on replace l'objectif dans le contexte de la détection de la délimitation d'une piste cyclable, à partir d'un point de vue fixe, on peut s'interroger sur le besoin de faire de la télédétection en temps réel avec une vidéo en haute résolution.
\vspace{\baselineskip}
\\
\noindent La deuxième contrainte pourrait potentiellement être améliorée en utilisant un modèle dont l'architecture est plus performante, mais implicitement plus complexe, telle que l'architecture SegNet101 ou DeepLabV3, mais qui risque d'être aussi plus demandant en ressources matérielles, GPU, CPU et mémoire. Ce qui risque de remettre en question les performances matérielles et logicielles du nano ordinateur. C'est ainsi probables la raison pour laquelle NVIDIA procure uniquement des jeux de modèles pré entrainés de segmentation sémantique avec SegNet18 pour le nano ordinateur. 
\subsection{Limites}
\subsubsection{Limites matérielles}
\noindent Au sujet des limites matérielles, durant l'inférence, il n'y a aucune limite qui est ressortie lors des tests de performance. Par contre il a été lu qu'un mode opérationnel 24/7 n'offrait qu'une durée de vie de 4.4 années au nano ordinateur. 
\subsubsection{Limites applicatives}
\noindent Au sujet des limites applicatives, durant l'inférence, il n'y a aucune limite qui est ressortie lors des tests de performance. Par contre il a été observé durant l'essai que le nano ordinateur ne devrait pas être utilisé comme machine de développement, pour par exemple pour re entrainé une architecture. L'entrainement de l'architecture SegNet18 n'a pas fonctionné dans un environnement virtuel Python, ni dans un conteneur Docker sur le nano ordinateur, celui-ci arrête de fonctionner. Il n'y a pas eu d'investigation, mais il semble que le nano ordinateur atteins une limite mémoire qui le ralenti jusqu'à un arrêt de fonctionnement. DIGITS ne peut pas non plus être utilisé, car il n'est pas compatible avec l'architecture ARM du nano ordinateur. Si l'objectif est d'améliorer l'architecture en ré entrainant à la demande en mode opérationnel, l'entrainement et l'inférence ne peuvent cohabiter simultanément, cela me semble donc impossible aujourd'hui, à moins d'investiguer et de trouver un moyen d'optimiser les ressources.
\vspace{\baselineskip}
\\
\noindent Durant l'essai, il a aussi été observé que l'utilisation prolongée de Chromium peut impacter les performances du nano ordinateur en le ralentissant grandement. 
\subsection{Optimisation}
\subsubsection{Optimisation matérielle}
\noindent Plusieurs initiatives ont été tentées afin d'optimiser le matériel. L'optimisation requise est celle d'utiliser un adaptateur 5V 4Amp, recommandé et fiable, afin de fournir assez de puissance au nano ordinateur lorsque d'autres périphériques viennent s'y raccorder, comme une caméra et un ventilateur. Profiter du PoE de l'interface réseau n'a pas été testé, mais cela semble aussi être une option rapide et simple à mettre en place pour assister l'adaptateur. Enfin, forcer le démarrage du ventilateur dés le démarrage du nano ordinateur est une autre optimisation simple, mais efficace a appliqué. Par contre, je ne recommande pas l'utilisation d'un dongle ou adaptateur Wifi, celui-ci étant énergivore, peu efficace, non fiable, ni stable. Il prendrait de plus un pourcentage d'utilisation non négligeable du Hub USB 3.0. 
\vspace{\baselineskip}
\\
\noindent La seconde optimisation qui a été logiquement tentée est celle d'utiliser un \acrshort{ssd} à la place d'une microSD, car il y aurait, selon moi, beaucoup d'avantages. Pour des raisons de performances d'abord, le gain peut-être d'au moins 4 fois plus grand en opération de lecture I/O. Ensuite, en durée de vie, une carte microSD est fragile et ne peut être considérée comme un système fiable sur le long terme. D'un point de vue capacité de stockage, un \acrshort{ssd} peut offrir beaucoup mieux. Enfin, un \acrshort{ssd} est plus adapté à la gestion d'un système opérationnelle et la manipulation de petits fichiers. En contrepartie, un SSD va demander plus de puissance (Watt) au nano ordinateur, et générer plus de chaleur. Ma recommandation serait de trouver un disque \acrshort{ssd} interne au format NVMe, connecteur de type M.2, qui peut être facilement branché au port PCIe du nano ordinateur.
\vspace{\baselineskip}
\\
\noindent Une autre optimisation matérielle qui n'est pas à négliger est le boitier. Vu que le système a été conçu pour être en opération continue sur le terrain, un boitier bien conçu permet de le protéger sur le long terme. Il doit être bien adapté à ses périphériques, que sont la caméra et le ventilateur, et optionnellement un \acrshort{ssd} interne.
\subsubsection{Optimisation logicielle}
\noindent La version de l'architecture SegNet18 fournit par NVIDIA s'exécute avec fluidité, sans que l'on sente que le nano ordinateur puisse devenir non réactif. Au démarrage de l'inférence, il y a une brève période de 2-3 secondes ou le nano ordinateur ne répond plus. Mais sinon, il est tout à fait possible d'utiliser le nano ordinateur pendant l'inférence d'une vidéo ou avec la caméra, et même avec 5-6 onglets d'ouverts dans Chromium. Lorsque le nombre d'onglets, ou d'instances de Chromium, devient trop grand, il a été observé que le nano ordinateur devenait lent, limite non fonctionnel, jusqu'à la fermeture des onglets. Ceci est probablement dû à une limitation mémoire.
\vspace{\baselineskip}
\\
\noindent Autrement, certaines corrections au code C++ ont dû être apportées au code source original fourni par NVIDIA: l'image de la caméra est à l'envers (et je ne pouvais monter la caméra dans le sens opposé dans le boitier); le pipeline gstreamer interne de l'application est trop spécifique pour supporter un flux vidéo autre que celui provenant de la caméra; et la taille de la fenêtre XWindow qui s'ouvre pour afficher la segmentation de la vidéo est programmée pour prendre tout l'écran, nous faisant perdre ainsi l'accessibilité et visibilité aux autres fenêtres.
\myparagraph{Segmentation}
\vspace{\baselineskip}
\\
\noindent Comme observé durant les tests, la résolution de la segmentation avec l'architecture SegNet18 est faible, 19 x 10 pixels. Le désavantage majeur dans le contexte de cet essai est que les délimitations des classes sont approximatives, incluant celle du chemin. Même si le IoU et le F1 score sont pourtant acceptable pour cette classe. Il semble que ce serait l'élément prioritaire à améliorer. 
\myparagraph{Ré entrainement}
\vspace{\baselineskip}
\\
\noindent Même si la phase de ré entrainement a pu être initiée durant l'essai, elle n'a pas durée longtemps : re générer le même fichier interopérable \acrshort{onnx} avec le code source original a été laborieux. Il est vrai que NVIDIA propose, avec DIGITS, un environnement de re entrainement des architectures qu'ils offrent. Mais dans le contexte de cet essai, je n'avais à ma disposition que l'environnement de Compute Canada. Néanmoins je pense qu'il est important de pouvoir le faire tout en gardant le contrôle de son environnement, par exemple pour permettre d'adapter l'architecture de notre choix, plus performant, tel que SegNet101 ou DeepLabV3, entrainé avec le jeu de données DeepScene, et l'adapter à un jeu de données personnalisé. Le questionnement est de savoir comment le nano ordinateur réagit avec une architecture beaucoup plus grosse et complexe que SegNet18. Dans une autre perspective, il serait bon de considérer un modèle de nano ordinateur plus performant, tel que le Jetson Xavier AGX.
\subsection{Accès distant}
\noindent L'un des sous-objectifs était de permettre un accès à distance sécurisé au nano ordinateur. Pour des raisons de temps, aucune activité de recherche ni de test n'a été effectuée dans le cadre de cet essai. 
\subsection{Documentation}
\noindent Un gros effort de documentation du cheminement de l'essai a été fait. La documentation des activités, des procédures, des scripts, des modifications, des erreurs, des références, et bien d'autres éléments encore, est disponible publiquement dans le blogue sur GitHub\footnote{\url{https://github.com/vince7lf/vince7lf.github.io}}.
%----------------------------------------------------------------------------------------
\clearpage 
\newpage
{
   \color{red}
   \par Il faut uniformiser le style. Il faut que les références dans le texte soient "(year, lastname)" et non "[n]"\todo{TODO}
}
\printbibliography[title={\bibname\label{bib:references}}] 
%----------------------------------------------------------------------------------------
\clearpage 
\newpage
\section{Annexes}
\subsection{Exemples de nano ordinateurs qui supportent les \acrshort{sdk} pour l'\acrshort{ia}} \label{annexe:nano_computer_samples}
{
   \renewcommand*{\arraystretch}{1.4}
   \begin{table}[ht]
   \centering
   \caption{Comparaison des trois nano ordinateurs supportant les \acrshort{sdk} pour l'\acrshort{ia}}\label{table:compare_nano}
   \vspace{0.1em} % Adjust the height of the space between caption and tabular
   \begin{tabular}{{@{}|p{12em}|p{12em}|p{12em}|@{}}}
      \hline
      \textbf{NVIDIA Jetson Nano} & \textbf{NVIDIA Jetson Xavier AGX} & \textbf{Raspberry Pi 4B + Intel NCS2}\\
      \hline
      \centering 99USD & \centering 599USD & 134USD (55USD + 79USD) \\
      \hline
      45 x 69.6 mm, 250 gr, 5-10 W & 100 x 87 mm, 630 gr, 10-15-30 W & 56 x 85.60 mm + 27x72 mm, 45 gr + 18.1 gr, 15 W\\
      \hline
      128-core NVIDIA Maxwell GPU & 512-core NVIDIA Volta GPU with 64 Tensor Cores & Intel Movidius Myriad X VPU 16 SHAVE cores \\
      \hline
      Quad-Core ARM Cortex-A57 MPCore & 8-core NVIDIA Carmel Arm v8.2 64-bit CPU 8MB L2 + 4MB L3 & Quad-core ARM Cortex-A72 64-bit @ 1.5 GHz\\
      \hline
      4 GB 64-bit LPDDR4 & 32 GB 256-bit LPDDR4 & 4GB LPDDR4\\
      \hline
      0.47 TFLOPS@FP16 & 5.5-11.5 TFLOPS@FP16; 20-32 TOPS@INT8 & 4 FLOPS@FP16, 1 TOPS@INT8 \\
      \hline
   \end{tabular}
   \end{table}
}
\subsection{Communication avec l’\acrlong{apcpontjc}}
\noindent L'\acrlong{apcpontjc} (\acrshort{apcpontjc}) a été contacté afin de leur demander la permission d'utiliser leurs fichiers multimédias de la piste cyclable du pont Jacques-Cartier, tel que leurs images et leurs vidéos. Voici les détails de la communication et les conditions d'utilisation.
{
   \centering
   \label{pdf:courriel_autorisation_apc_pontjc}
   \includegraphics[page=1,width=\textwidth]{courriel_autorisation_apc_pontjc.pdf} \\
   \clearpage
   \includegraphics[page=2,width=\textwidth]{courriel_autorisation_apc_pontjc.pdf} \\
}

% do not used \includepdf, it breaks \printglossary. Prefer using \includegraphics, it works with pdf
% \includepdf[pages=1,width=\textwidth]{courriel_autorisation_apc_pontjc.pdf} \label{pdf:courriel_autorisation_apc_pontjc}
% \includepdf[pages=-,pagecommand={},width=\textwidth]{courriel_autorisation_apc_pontjc.pdf} \label{pdf:courriel_autorisation_apc_pontjc}

%----------------------------------------------------------------------------------------
\end{document}